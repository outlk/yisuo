\section{狐狸和葡萄}

饥饿的狐狸看见葡萄架上挂着一串串晶莹剔透的葡萄,口水直流,想要摘下来吃,但又摘不到。看了一会儿,无可奈何地走了,他边走边自己安慰自己说:“这葡萄没有熟,肯定是酸的。”

{\bfseries \color{red}这就是说,有些人能力小,做不成事,就借口说时机未成熟。}

\section{狼与鹭鸶}

狼误吞下了一块骨头,十分难受,四处奔走,寻访医生。他遇见了鹭鸶,谈定酬金请他取出骨头,鹭鸶把自己的头伸进狼的喉咙里,叼出了骨头,便向狼要定好的酬金。狼回答说:“喂,朋友,你能从狼嘴里平安无事地收回头来,难道还不满足,怎么还要讲报酬?”

{\bfseries \color{red}这故事说明,对坏人行善的报酬,就是认识坏人不讲信用的本质。}

\section{小男孩与蝎子}

有个小孩在城墙前捉蚱蜢,一会儿就捉了许多。忽然看见一只蝎子,他以为也是蚱蜢,便着两手去捕捉他。蝎子举起他的毒刺,说道:“来吧,如果你真敢这样做,就连你捉的蚱蜢也会统统失掉。”

{\bfseries \color{red}这故事告诫人们,要分辨清好人和坏人,区别对待他们。}

\section{掉在井里的狐狸和公山羊}

一只狐狸失足掉到了井里,不论他如何挣扎仍没法爬上去,只好呆在那里。公山羊觉得口渴极了,来到这井边,看见狐狸在井下,便问他井水好不好喝?狐狸觉得机会来了,心中暗喜,马上镇静下来,极力赞美井水好喝,说这水是天下第一泉,清甜爽口,并劝山羊赶快下来,与他痛饮。一心只想喝水信以为真的山羊,便不假思索地跳了下去,当他咕咚咕咚痛饮完后,就不得不与狐狸一起共商上井的办法。狐狸早有准备,他狡猾地说:“我倒有一个方法。你用前脚扒在井墙上,再把角竖直了,我从你后背跳上井去,再拉你上来,我们就都得救了。”公山羊同意了他的提议,狐狸踩着他的后脚,跳到他背上,然后再从角上用力一跳,跳出了井口。狐狸上去以后,准备独自逃离。公山羊指责狐狸不信守诺言。狐狸回过头对公山羊说:“喂,朋友,你的头脑如果像你的胡须那样完美,你就不至于在没看清出口之前就盲目地跳下去。”

{\bfseries \color{red}这故事说明,聪明的人应当事先考虑清楚事情的结果,然后才去做。}

\section{寡妇与母鸡}

有个寡妇养着一只母鸡,母鸡每天下一个蛋。她以为多给鸡喂些大麦,就会每天下两个蛋。于是,她就每天这样喂,结果母鸡长得越来越肥,每天连一个蛋也不下了。

{\bfseries \color{red}这故事说明,有些人因为贪婪,想得到更多的利益,结果连现有的都失掉了。}

\section{徒劳的寒鸦}

宙斯想要为鸟类立一个王,指定一个日期,要求众鸟全都按时出席,以便选他们之中最美丽的为王。众鸟都跑到河里去梳洗打扮。寒鸦知道自己没一处漂亮,便来到河边,捡起众鸟脱落下的羽毛,小心翼翼地全插在自己身上,再用胶粘住。指定的日期到了,所有的鸟都一齐来到宙斯面前。宙斯一眼就看见花花绿绿的寒鸦,在众鸟之中显得格外漂亮,准备立他为王。众鸟十分气愤,纷纷从寒鸦身上拔下本属于自己的羽毛。于是,寒鸦身上美丽的羽毛一下全没了,又变成了一只丑陋的寒鸦了。

{\bfseries \color{red}这故事是说,借助别人的东西可以得到美的假象,但那本不属于自己的东西被剥离时,就会原形毕露。}

\section{站在屋顶的小山羊与狼}

小山羊站在屋顶上,看见狼从底下走过,便谩骂他,嘲笑他。狼说道:“啊,伙计,骂我的不是你,而是你所处的地势。”

{\bfseries \color{red}这故事说明,地利与天机常常给人勇气去与强者抗争。}

\section{山震}

有一次,一座大山发生了大震动,震动发出的声音就像大声的呻吟和喧闹。许多人云集在山下观看,不知发生了什么事。当他们焦急地聚集在那里,担心看到什么不祥之兆时,仅看见从山里跑出一只老鼠。

{\bfseries \color{red}这是说庸人多自忧。}

\section{善与恶}

力量弱小的善,被恶赶走到了天上。善于是问宙斯,怎样才能回到人间去。宙斯告诉他,大家不要一起去,一个一个的去访问人间吧。恶与人很相近,所以接连不断地去找他们。善因为从天上下来,所以就来得很慢很慢。

{\bfseries \color{red}这就是说,人很不容易遇到善,却每日为恶所伤害。}

\section{老猎狗}

一条老猎狗年轻力壮时从未向森林中任何野兽屈服过,年老后,在一次狩猎中,遇到一头野猪,他勇敢地扑上去咬住野猪的耳朵。由于他的牙齿老化无力,不能牢牢地咬住,野猪逃跑了。主人跑过来后大失所望,痛骂他一顿。年老的猎狗抬起头来说:“主人啊!这不能怪我不行。我的勇敢精神和年轻时是一样的,但我不能抗拒自然规律。从前我的行为受到了你的称赞,现在也不应受到你的责备。”

{\bfseries \color{red}这是说,生老病死是不可抗拒的规律。}

\section{蚂蚁与屎壳郎}

夏天,别的动物都悠闲地生活,只有蚂蚁在田里跑来跑去,搜集小麦和大麦,给自己贮存冬季吃的食物。屎壳郎惊奇地问他为何这般勤劳。蚂蚁当时什么也没说。冬天来了,大雨冲掉了牛粪,饥饿的屎壳郎,走到蚂蚁那里乞食,蚂蚁对他说:“喂,伙计,如果当时在我劳动时,你不是批评我,而是也去做工,现在就不会忍饥挨饿了。”

{\bfseries \color{red}这是说,尽管风云变化万千,未雨绸缪的人都能避免灾难。}

\section{公鸡和宝玉}

一只公鸡在田野里为自己和母鸡们寻找食物。他发现了一块宝玉,便对宝玉说:“若不是我,而是你的主人找到了你,他会非常珍惜地把你捡起来;但我发现了你却毫无用处。我与其得到世界上一切宝玉,倒不如得到一颗麦子好。”

{\bfseries \color{red}这是说自己需要的东西才是真正珍贵的。}

\section{小鹿与他的父亲}

有一天,小鹿对公鹿说道,“父亲,你怎么还怕狗呢?你比他高大,比他跑得更快,而且还有很大的角用于自卫。”公鹿笑着说:“孩儿,你说得都对,可我只知道一点,一听到狗的叫声,我就会不由自主地立刻逃跑。”

{\bfseries \color{red}这故事说明,激励那些天生胆小、软弱的人毫无用处。}

\section{两只口袋}

普罗米修斯创造了人,又在他们每人脖子上挂了两只口袋,一只装别人的缺点,另一只装自己的。他把那只装别人缺点的口袋挂在胸前,另一只则挂在背后。因此人们总是能够很快地看见别人的缺点,而自己的却总看不见。

{\bfseries \color{red}这故事说明人们往往喜欢挑剔别人的缺点,却无视自身的缺点。}

\section{山鹰与狐狸}

山鹰与狐狸互相结为好友,为了彼此的友谊更加巩固,他们决定住在一起。于是鹰飞到一棵高树上面,筑起巢来孵育后代,狐狸则走进树下的灌木丛中间,生儿育女。有一天,狐狸出去觅食,鹰也正好断了炊,他便飞入灌木丛中,把幼小的狐狸抢走,与雏鹰一起饱餐一顿。狐狸回来后,知道这事是鹰所做,他为儿女的死悲痛,而最令他悲痛的是一时无法报仇,因为他是走兽,只能在地上跑,不能去追逐会飞的鸟。因此他只好远远地站着诅咒敌人,这是力量弱小者唯一可以做到的事情。不久,鹰的背信弃义的罪行也受到了严惩。有一次,一些人在野外杀羊祭神,鹰飞下去,从祭坛上抓起了带着火的羊肉,带回了自己的巢里。这时候一阵狂风吹了过来,巢里细小干枯的树枝马上燃起了猛烈的火焰。那些羽毛未丰的雏鹰都被烧死了,并从树上掉了下来。狐狸便跑了过去,在鹰的眼前,把那些小鹰全都吃了。

{\bfseries \color{red}这故事说明,对于背信弃义的人,即使受害者弱小,不能报复他,可神会惩治他。}

\section{马与马夫}

从前,有个马夫,他偷偷地把喂马的大麦卖掉了,但仍每天给马擦洗,用梳子梳理马毛。马对马夫说道:“如果你真心想要我长得美,就不要再卖掉喂我的大麦了。”

{\bfseries \color{red}这是说,那些虚情假意的人用花言巧语和小恩小惠去贿赂别人,却把别人最必需的东西夺走了。}

\section{农夫与蛇}

冬天,农夫发现一条蛇冻僵了,他很可怜它,便把蛇放在自己怀里。蛇温暖后,苏醒了过来,恢复了它的本性,咬了它的恩人一口,使他受到了致命的伤害。农夫临死前说:“我该死,我怜悯恶人,应该受恶报。”

{\bfseries \color{red}这故事说明,即使对恶人仁至义尽,他们的邪恶本性也是不会改变的。}

\section{吹xiao的渔夫}

有一个会吹xiao的渔夫,带着他心爱的箫和渔网来到了海边。他先站在一块突出的岩石上,吹起箫来,心想鱼听到这美妙音乐就会自己跳到他的前面来的。他聚精会神地吹了好久,毫无结果。他只好将箫放下,拿起网来,向水里撒去,结果捕到了许多的鱼。他将网中的鱼一条条地扔到岸上,并对乱蹦乱跳的鱼说:“喂,你们这些不识好歹的东西!我吹xiao时,你们不跳舞,现在我不吹了,你们倒跳了起来。”

{\bfseries \color{red}这故事适用于那些做事不择时机的人们。}

\section{人与森林之神}

传说,从前有个人与一个森林之神萨堤罗斯交朋友。冬天到了,天气变得十分寒冷,那人把手放到嘴边不断地呵起热气来。森林之神忙问这是什么原因,为什么要这样做。那人回答说:“天寒手冷,呵热气手可变暖和些。”后来,他们同桌一起吃饭,桌上的饭菜热气腾腾,烫得很,那人夹起一点放到嘴边。森林之神又问他这是为何。他说饭菜太烫,把它吹凉。森林之神对人说道:“喂,朋友!我只好同你绝交了,因为你这嘴能一会儿出热气,一会儿又出冷气。”

{\bfseries \color{red}这故事是说,切不可与那些反复无常的人交朋友。}

\section{苍蝇与蜜}



{\bfseries \color{red}}

\section{两只打架的公鸡}

为了争占母鸡,两只公鸡打了起来,其中一只把另一只打跑了。那只被打败的只好躲进有遮盖的地方,那只打胜的却飞到高墙上大喊大叫。这时一只鹰猛飞过来,将他抓了去。这以后,那只被打败的公鸡平平安安地zhan有了那些母鸡。

{\bfseries \color{red}这故事说明,傲慢给人带来危害,谦卑给人恩惠。}

\section{老鼠与青蛙}

老鼠不幸被青蛙所爱。青蛙愚蠢地把老鼠的脚绑在自己的脚上。开始,他们在地面上行走,走了走,一切正常,还可吃着谷子。当来到池塘边时,青蛙把老鼠带到了水里,他自己在水里嬉戏玩耍,高兴得呱呱叫。可怜的老鼠却被水灌饱,淹死了。不久,老鼠浮出水面,但他的脚仍和青蛙绑在一起。鹞子飞过这里,看见了老鼠,冲向水中,把他抓了起来,青蛙跟着被提出了水面,也成了鹞子的美食。

{\bfseries \color{red}这是说,与别人关系太亲密,在灾难降临时,往往会受到牵连。}

\section{叼着肉的狗}

狗叼着肉渡过一条河。他看见水中自己的倒影,还以为是另一条狗叼着一块更大的肉。想到这里,他决定要去抢那块更大的肉。于是,他扑到水中抢那块更大的。结果,他两块肉都没得到,水中那块本来就不存在,原有那块又被河水冲走了。

{\bfseries \color{red}这故事适用于贪婪的人。}

\section{公牛与车轴}

几头公牛正使劲拉着货车行走,车轴被压得发出吱吱的响声,牛回过头,不耐烦地对车轴说道:“喂,朋友,我们无声无息负担着全部重量,你叫唤什么?”

{\bfseries \color{red}这故事是说,那些叫唤得特别响的人往往干活少,而那些不作声的人往往承担着全部重量。}

\section{狼与小羊}



{\bfseries \color{red}}

\section{熊与狐狸}

有一头熊大肆吹嘘,说他很爱人类,因为他从不吃死人。一只狐狸对他说:“但愿你把死人撕得粉碎,而不要危害那些活着的人。”

{\bfseries \color{red}这故事适用于生活中那些假装善良的恶人。}

\section{田鼠与家鼠}

田鼠与家鼠是好朋友,家鼠应田鼠所约,去乡下赴宴。他一边吃着大麦与谷子,一边对田鼠说:“朋友,你知道,你这是过着蚂蚁一般的生活,我那里有很多好东西,去与我一起享受吧!”田鼠跟随家鼠来到城里,家鼠给田鼠看豆子和谷子,还有红枣、干酪、蜂蜜、果子。田鼠看得目瞪口呆,大为惊讶,称赞不已,并悲叹自己的命运。他们正要开始吃,有人打开了门,胆小的家鼠一听声响,害怕得赶紧钻进了鼠洞。当家鼠再想拿干酪时,有人又进屋里拿什么东西。他一见到有人,立刻又钻回了洞里。这时,田鼠也顾不上饥饿,颤颤惊惊地对家鼠说:“朋友,再见吧!你自己尽情地去吃,担惊受怕地享受这些好吃的东西吧。可怜的我还是去啃那些大麦和谷子,平平安安地去过你看不起的普通生活。”

{\bfseries \color{red}这故事说明,人们宁愿过简单平稳的生活,而不愿享受那充满恐怖的欢乐生活。}

\section{狗、公鸡和狐狸}

狗与公鸡结交为朋友,他们一同赶路。到了晚上,公鸡一跃跳到树上,在树枝上栖息,狗就在下面树洞里过夜。黎明到来时,公鸡像往常一样啼叫起来。有只狐狸听见鸡叫,想要吃鸡肉,便跑来站在树下,恭敬地请鸡下来,并说:“多么美的嗓音啊!太悦耳动听了,我真想拥抱你。快下来,让我们一起唱支小夜曲吧。”鸡回答说:“请你去叫醒树洞里的那个看门守夜的,他一开门,我就可以下来。”狐狸立刻去叫门,狗突然跳了起来,把他咬住撕碎了。

{\bfseries \color{red}这故事说明,聪明的人临危不乱,巧妙而轻易地击败敌人。}

\section{狮子与报恩的老鼠}

狮子睡着了,有只老鼠跳到了他身上。狮子猛然站起来,把他抓住,准备吃掉。老鼠请求饶命,并说如果保住性命,必将报恩,狮子轻蔑地笑了笑,便把他放走了。不久,狮子真的被老鼠救了性命。原来狮子被一个猎人抓获,并用绳索把他捆在一棵树上。老鼠听到了他的哀嚎,走过去咬断绳索,放走了狮子,并说:“你当时嘲笑我,不相信能得到我的报答,现在可清楚了,老鼠也能报恩。”

{\bfseries \color{red}这故事说明,时运交替变更,强者也会有需要弱者的时候。}

\section{海鸥和鸢}

一只海鸥吞吃了一条很大的鱼,胀破了他的肚子,躺在海滩上等死。一只鸢看见后说:“你真是自作孽啊!你本是空中飞的鸟,不该到海里去找食物。”

{\bfseries \color{red}这是说每个人都应该安分守己。}

\section{卖神像的人}

有人雕刻了一个赫耳墨斯的木像,拿到市场去卖。因为没有一个买主上前,他便大声叫喊,想招揽生意,说有赐福招财的神出售。这时旁边有一个人对他说道:“喂,朋友,既然这样,你自己应该享受他的好处,为什么还要卖掉他呢?”他回答说:“我要的是现在马上能兑现利益,这神的利益却来得很慢。”

{\bfseries \color{red}这故事正是说那种不择手段地求利,连神也不尊敬的人。}

\section{牛和蛙}

一头牛到水潭边去喝水,踩着了一群小蛙,并踩死了其中一只。小蛙妈妈回来后,见到少了一个儿子,便问他的兄弟们,他到哪里去了。一只小蛙说:“亲爱的妈妈,他死了。刚才有一头巨大的四足兽来到潭边,用他的蹄子踩死了我们的兄弟。”蛙妈妈一边尽力鼓气,一边问道:“那野兽是不是这个样子,这般大小呢?”小蛙说:“妈妈,您别再鼓气了。我想您不可能和那怪物一样大小,再鼓气就会把肚子胀破。”

{\bfseries \color{red}这是说,渺小无论如何也不能与伟大相比。}

\section{众树与荆棘}

石榴树、苹果树、橄榄树相互为谁的果实最好而争吵不休。正当他们激烈争闹时,篱笆边的荆棘听到了,便说:“朋友们,我们不要再争吵。”

{\bfseries \color{red}这是说,有些微不足道的人,在强者相互争斗中,也自不量力地极想表现一番。}

\section{乌龟与鹰}

乌龟看见鹰在空中飞翔,便请求鹰教他飞行。鹰劝告他,说他不能飞行。可乌龟再三恳求,鹰便抓住他,飞到高空,然后将他松开。乌龟落在岩石上,被摔得粉身碎骨。

{\bfseries \color{red}这故事说明,那些好高鹜远,不切实际的人必将失败。}

\section{骡子}

有匹吃大麦长大的骡子很强壮。每当他跳跃时,总是自言自语说:“我父亲一定是一匹能奔善跑的马,我非常像他。”有一天,因为需要,骡子不得不被拉去不停地跑路。回来后,他才愁眉苦脸地想起自己的父亲是驴子。

{\bfseries \color{red}这故事说明,人们如遇好运出了名,也千万不要忘记自己的本性,因为生活如同潮起潮落,前途难以预测。}

\section{乌龟与兔}

乌龟与兔为他们俩谁跑得快而争论不休。于是,他们定好了比赛的时间和地点。比赛一开始,兔觉得自己是天生的飞毛腿,跑得快,对比赛掉以轻心,躺在路旁睡着了。乌龟深知自己走得慢,毫不气馁,不停地朝前奔跑。结果,乌龟超过了睡熟了的兔子,夺得了胜利的奖品。

{\bfseries \color{red}这故事说明,奋发图强的弱者也能战胜骄傲自满的强者。}

\section{猫和鸡}

有一天,猫不怀好意地、假惺惺地举办生日宴会,请来许多鸡赴宴。鸡刚一到齐,猫立刻就关上大门,把他们统统吃掉了。

{\bfseries \color{red}这是说,对于敌人不要抱有任何美好的希望,否则将遭受更大的不幸。}

\section{说谎的放羊娃}

有个放羊娃赶着他的羊群到村外很远的地方去放牧。他老是喜欢说谎,开玩笑,时常大声向村里人呼救,谎称有狼来袭击他的羊群。开始两三回,村里人都惊慌得立刻跑来,被他嘲笑后,没趣地走了回去。后来,有一天,狼真的来了,窜入羊群,大肆咬杀。牧羊娃对着村里拼命呼喊救命,村里人却认为他又在像往常一样说谎,开玩笑,没有人再理他。结果,他的羊群全被狼吃掉了。

{\bfseries \color{red}这故事说明,那些常常说谎话的人,即使再说真话也无人相信。}

\section{病鹿}

有只生病的鹿躺在草地上。众多野兽前去看望他,并吃光了那附近的草。鹿病好后,因找不到草,缺少食物而体弱至死。

{\bfseries \color{red}这故事是说,过多地结交毫无益处的朋友是有害无益的。}

\section{老太婆和酒瓶}

一个老太婆找到一个不久前曾装过最好陈酒的空酒瓶。这酒瓶仍带着浓浓的酒香,她多次把酒瓶放在鼻尖下,不断摇晃,贪婪地吮吸酒香,并说:“啊,多么甜美!装过酒的空瓶都留下这样甘美难忘的香味,那酒真不知有多么美味芬香。”

{\bfseries \color{red}这是说美好的事物留下深远的影响,使人们永远难以忘记。}

\section{月亮和她妈妈}



{\bfseries \color{red}}

\section{驴子与蝉}



{\bfseries \color{red}}

\section{狐狸和樵夫}

狐狸为躲避猎人们追赶而逃窜,恰巧遇见了一个樵夫,便请求让他躲藏起来,樵夫叫狐狸去他的小屋里躲着。一会儿,许多猎人赶来,向樵夫打听狐狸的下落,他嘴里一边大声说不知道,又一边做手势,告诉他们狐狸躲藏的地方。猎人们相信了他的话,并没留意他的手势。狐狸见猎人们都走远了,便从小屋出来,什么都没说就走。樵夫责备狐狸,说自己救了他一命,一点谢意都不表示。狐狸回答说:“如果你的手势与你的语言是一致的,我就该好好地感谢你了。”

{\bfseries \color{red}这故事适用于那些嘴里说要做好事,而行为上却作恶的人。}

\section{狼与逃进神庙的小羊}

一只小羊被狼追赶,逃进了一个神庙里。狼对小羊说,如不赶快出来,祭司会抓住你,把你献给神。小羊回答说:“我宁愿献给神,也比被你吃掉好。”

{\bfseries \color{red}这故事说明,对要死的人来说,应选择有价值的死。}

\section{口渴的乌鸦}

乌鸦口渴得要命,飞到一只大水罐旁,水罐里没有很多水,他想尽了办法,仍喝不到。于是,他就使出全身力气去推,想把罐推dao,倒出水来,而大水罐却推也推不动。这时,乌鸦想起了他曾经使用的办法,用口叼着石子投到水罐里,随着石子的增多,罐里的水也就逐渐地升高了。最后,乌鸦高兴地喝到了水,解了口渴。

{\bfseries \color{red}这故事说明,智慧往往胜过力气。}

\section{小蟹与母蟹}

母蟹对小蟹说:“你不要横爬,为什么不直着走?”他答道:“妈妈,请您亲自教我怎样直走,我将照着你的样子走。”可母蟹根本不会直走,于是小蟹说她笨。

{\bfseries \color{red}这是说,教育者自己必须正直地生活,正直地走,才能去教导别人。}

\section{骆驼与宙斯}

骆驼见牛炫耀自己漂亮的角,羡慕不已,自己也想要长两只角。于是,他来到宙斯那里,请求给他加上一对角。宙斯因为骆驼不满足已有庞大的身体和强大的力气,还要妄想得到更多的东西,气愤异常,不仅没让他长角,还把他的耳朵砍掉一大截。

{\bfseries \color{red}这故事说明,许多人因为贪得无厌,一见别人的东西就眼红,不知不觉连自己已具有的东西也失去了。}

\section{一只眼睛的鹿}

有头瞎了一只眼的鹿,来到海边吃草,他用那只好的眼睛注视着陆地,防备猎人的攻击,而用瞎了的那只眼对着大海,他认为海那边不会发生什么危险。不料有人乘船从海上经过这里,看见了这头鹿,一箭就把他射倒了。他将要咽气的时候,自言自语地说:“我真是不幸,我防范着陆地那面,而我所信赖的海这面却给我带来了灾难。”

{\bfseries \color{red}这故事是说,事实常常与我们的预料相反,以为是危险的事情却倒很安全,相信是安全的却更危险。}

\section{朋友与熊}

两个平常非常要好的朋友一道上路。途中,突然遇到一头大熊,其中的一个立即闪电般地抢先爬上了树,躲了起来,而另一个眼见逃生无望,便灵机一动马上躺倒在地上,紧紧地屏住呼吸,假装死了。据说,熊从来不吃死人。熊走到他跟前,用鼻子在他脸上嗅了嗅,转身就走了。躲在树上的人下来后,问熊在他耳边说了些什么。那人委婉地回答说:“熊告诉我,今后千万注意,别和那些不能共患难的朋友一起同行。”

{\bfseries \color{red}这故事说明,不能共患难的人不是真正的朋友。}

\section{牛栏里的鹿}

一只鹿被猎狗追赶得很急,跑进一个农家院子里,恐惧不安地混在牛群里躲藏起来。一头牛好意地告诫他说:“喂!不幸的家伙!你为什么要这样做,你将自己交到敌人手中,这不是自投罗网吗?”鹿回答说:“朋友,只要你允许我躲在这里,我便会寻找机会逃走的。”到了傍晚,牧人来喂牲口,他们并未发现鹿。管家和几个长工经过牛栏时,也没注意牛栏里有鹿。鹿庆幸自己安全,便向那头好意劝告过他的牛表示衷心的感谢。另一头牛说:“我们固然想保护你,但现在还不能完全放心。另外还有一个人要经过牛栏,他对于一切都十分留心。只要他经过后,你的性命就有了保证。”这时,主人进来了,一边埋怨牛饲料分配得不好,一边走到草架旁大声说:“怎么搞的,只有这么一点点草料?牛栏垫的草也不够一半。这些懒虫连蜘蛛网也没打扫干净。”当他在牛栏里走来走去检查每样东西时,发现鹿角露出在草料上面,便叫来人捉住这只鹿,把他杀掉了。

{\bfseries \color{red}这是说,在逃避一种危险时,不要忽视另一种危险。}

\section{烧炭人与漂布人}

烧炭人在一所房子里经营,看见有一个漂布人搬迁到他的旁边来住时,满怀高兴地走上去劝他与自己同住,并解释说这样彼此更亲密,更方便,还更省钱。漂布人却回答说:“也许你说的是真话,但完全不可能办到,因为凡我所漂白的,都将被你弄黑。”

{\bfseries \color{red}这故事说明,不同类的人难相处。}

\section{狮子、驴子与狐狸}

狮子和驴子以及狐狸商量好一起联合去打猎,他们捕获了许多野兽,狮子命令驴子把猎物分一分。驴子平均分成三份,请狮子自己挑选,狮子勃然大怒,猛扑过去把驴子吃了。狮子又命令狐狸来分。狐狸把所有的猎物都堆在一起,仅留一点点给他自己,然后请狮子来拿。狮子问他,是谁教他这样分的,狐狸回答说:“是驴子的不幸。”

{\bfseries \color{red}这故事说明,应该从别人的不幸中吸取经验和教训。}

\section{驴子与小狗}

有人养着一只狗和一头驴子,主人常同狗一起嬉戏。有一天,他外出吃饭,带回一些食物,扔给狗吃。狗高兴得摇着尾巴迎了上去。驴子非常羡慕,也蹦蹦跳跳跑了过去,结果踢了主人一脚。主人十分气愤,痛打了驴子一顿,并把它拴在马槽边。

{\bfseries \color{red}这故事说明,同样的事情不一定适合于所有的人。}

\section{风与太阳}

北风与太阳两方为谁的能量大相互争论不休。他们决定,谁能使得行人脱下衣服,谁就胜利了。北风一开始就猛烈地刮,路上的行人紧紧裹住自己的衣服,风见此,刮得更猛。行人冷得发抖,便添加更多衣服。风刮疲倦了,便让位给太阳。太阳最初把温和的阳光洒向行人,行人脱掉了添加的衣服,太阳接着把强烈阳光射向大地,行人们开始汗流浃背,渐渐地忍受不了,脱guang了衣服,跳到了旁边的河里去洗澡。

{\bfseries \color{red}这故事说明,劝说往往比强迫更为有效。}

\section{树和斧子}

一个人来到森林里,请求树给他一根木做斧子柄。树答应了他的请求,给他一根小树枝。他用小树枝做成了斧子柄,完好的装在斧子上,接着抡起斧子砍起树来。他很快就砍倒了森林中最贵重的大树。一棵老橡树悲伤地看着同伴被砍毁,无能为力,他对身旁的柏树说:“我们是自己先葬送了自己。如果我们不给他那根小树枝,他就无法砍伐我们,也许我们能永久地站立在这里。”

{\bfseries \color{red}这是说不能帮助对自己造成威胁的对象,那怕是一个小小的帮助。}

\section{兔子和猎狗}



{\bfseries \color{red}}

\section{恋爱的狮子与农夫}

狮子爱上了农夫的女儿,向她求婚。农夫不忍将女儿许配给野兽,但又惧怕狮子,一时无法拒绝,于是他急中生智,心生一计。狮子再次来请求农夫时,他便说,他认为狮子娶自己的女儿很适合,但狮子必须先拔去牙齿,剁掉爪子,否则不能把女儿嫁给他,因为姑娘惧怕这些东西。狮子利令智昏,色迷心窍,很轻易地接受了农夫的要求。从此,那农夫就瞧不起狮子,毫不惧怕他。狮子再来时,农夫就用棍子打他,把他绑起来。

{\bfseries \color{red}这故事说明,有些人轻易相信别人的话,抛弃自己特有的长处,结果,轻而易举地被原来恐惧他们的人击败了。}

\section{金枪鱼与海豚}

金枪鱼被海豚追逐,发出阵阵声音,在眼看要被海豚捉住的时候,金枪鱼猛然一跳,不料跳得太远,搁浅在岸边了。紧追不舍的海豚也跟着金枪鱼一跳,同样也搁浅在岸边。这时,金枪鱼回过头去,看着奄奄一息的海豚说:“现在我对死已无所畏惧了,因为我看见那造成这种结果的家伙也与我一同死。”

{\bfseries \color{red}这故事说明,有些人看见那些造成别人不幸的人,同样也给自己带来不幸,便容易忍受不幸带来的痛苦。}

\section{狼与羊群}

狼一心想吃掉羊群,但因有狗守护他们,不能得逞,心想非智取不可。于是,他派使者去拜访羊群,说狗才是他们俩之间的敌人,若能把狗赶出来,他们之间就能和平共处了。羊根本没有认清狼的险恶用心,不假思索地将狗赶出去。没有了狗的保护,狼便轻而易举地把羊都吃掉了。

{\bfseries \color{red}这是说,人们如失去保护自己的人,很快就会被敌人征服。}

\section{瞎子和小野兽}

一个瞎子精于用手触摸各种动物,什么动物只要他一摸,便能分辨出来。有个人带来一条小狼,请他摸一摸,说出是什么东西。他摸了摸这个小野兽后说:“这是一条狐狸,还是一条狼,我不大清楚。不过有一点我却十分明白,让这种动物进羊栏总是不安全的。”

{\bfseries \color{red}这故事是说恶劣的习性在年小时便可得知。}

\section{胃与脚}



{\bfseries \color{red}}

\section{大力神和车夫}

一名车夫赶着货车沿着乡间小路行进。途中车轮陷入了很深的车辙中,再也无法前进。这时,愚蠢的车夫吓得茫然失措,一筹莫展,痴呆呆地站在那里,凝视着货车,不断地高声喊叫,求大力神来助他一把。大力神来到后,对他说:“朋友,用你的肩膀扛起车轮,再抽打拉车的马。你自己不自力更生,尽力解决,仅靠祈求我,怎么行呢?”

{\bfseries \color{red}这是说自力更生,自助自立是克服困难的最好办法。}

\section{断尾的狐狸}

一只狐狸被捕兽器把尾巴夹断了。受了这种耻辱以后,他觉得自己脸上无光,生活很不好过,所以他决定劝说其他狐狸也去掉尾巴,大家都一样了,他的缺点就可以掩饰过去了。于是他召集了所有狐狸,劝说他们割去尾巴,他信口雌黄地说尾巴既不雅观,又使我们拖着一件笨重的东西,是多余的负担。有一只狐狸站起来说:“喂,朋友,如果这不是于你有利,你就不会这样煞费苦心地来劝说我们了。”

{\bfseries \color{red}这故事适用于那些不是出于好意,而是为了自己利益而劝告他人的人。}

\section{灯}

用橄榄油的灯能发出很亮的光。灯洋洋得意,以为自己的光比太阳还要亮得多。一阵风吹来,灯马上被吹灭了。有人来再点燃,并对他说:“灯啊,好好地亮着,别啰嗦!星星的光是永不会灭的。”

{\bfseries \color{red}这是说,人们不要因名声与荣誉而盲目自大,沾沾自喜。}

\section{兔与青蛙}

有一次,众多兔子聚集在一起,为自己的胆小无能而难过,互相悲叹他们的生活中充满着危险和恐惧,还常常被人、狗和鹰以及别的许多动物屠杀。他们都觉得,与其一生心惊胆战,还不如一死了之的好。于是就这样决定了,他们一齐奔向池塘,想要投水自尽。这时许多青蛙围着池塘边蹲着,听到了那急促的跑步声后,立刻纷纷跳下池塘。有一只较聪明的兔子,见到青蛙都跳到水中,似乎明白了什么。“朋友们,快停下,我们不必吓得去寻死了!。你们看,这里还有些比我们更胆小的动物呢!”

{\bfseries \color{red}这故事说明,那些不幸的人们往往会以他人的更大的不幸来聊以自慰。}

\section{母狮与狐狸}

狐狸取笑母狮无能,说她每胎仅能生一子。母狮回答说,“可我生下的毕竟是一头狮子。”

{\bfseries \color{red}这是说,贵重的价值在于质,而不在量。}

\section{渔夫与小梭鱼}

渔夫把网撒到海里,捕到了一条小梭鱼。那可怜的小鱼求渔夫把它放了,说他还太小了。他又许愿说:“待我长大后,再捉住我,将对你更有好处。”渔夫说:“现在我若放弃手中的小利,而去追求那希望渺茫的大利,那我岂不成了傻子么?”

{\bfseries \color{red}这故事说明,愚蠢的人才会放弃已到手的小利,而去追求那种虚无的大利。}

\section{农夫与他的儿子们}

有个农夫快要辞别人世时,想要把自己耕作经验传给儿子,便叫他们来说:“孩子们,我即将离开这个世界了,你们都去寻找我埋藏在葡萄园里的东西,把它们统统都找出来吧!”儿子们以为那里埋藏了金银财宝。父亲去世之后,他们把那葡萄园的地全都翻了一遍,什么宝物都没找到,却使葡萄园的地很好地耕作了一番,所以这年比以往结了更多的葡萄。

{\bfseries \color{red}这故事说明,劳动是最好的宝物。}

\section{农夫和鹳}

农夫在刚刚播种的田里布下许多网,许多来吃种子的鹤都被捉住了,并捉到一只鹳,鹳的腿被网折断了,它哀求农夫说:“饶了我吧,可怜可怜我吧。我又不是鹤,而是一只鹳,我是性情优美的鸟。你瞧,我多么孝顺父母,为他们辛勤劳作,再仔细看看我的羽毛,与鹤也完全不同。”农夫大笑说:“你说的话也许不错;但我只知道,你和这些偷吃种子的鹤一起被捉到,那末你就得和他们一起死。”

{\bfseries \color{red}这是说人们切莫与坏人交朋友。}

\section{鼹鼠}

传说鼹鼠的眼睛是瞎的,可小鼹鼠却对妈妈说他能看得见。妈想试验他一下,便拿来一小块香喷喷的食物,放在他面前,并问他是什么。他说是一颗小石头。母亲说:“啊,不幸的孩子,你不但眼睛看不见,连鼻子也没用了。”

{\bfseries \color{red}这故事是说,那些爱吹牛说大话的人,常常夸海口能做大事,却在一些微不足道的事情上暴露了本质。}

\section{老太婆与医生}

有位患了眼病的老太婆,请一位医生给她治病,并谈定了治疗费。那医生每次来给她上药治疗时,总是乘她闭着眼睛,顺手牵羊地偷走一些家具。老太婆的病终于治愈了,可她家里的东西几乎被偷光了。医生便向老太婆要商定好的治疗费,老太婆不肯付钱,便被带到法官那里。她说她许诺过要付给医生治疗费,条件是把她的眼病治好,可是经过医治后,她的眼睛却比以前更糟了。她说:“以前我还能看见家里的所有物品,现在却都看不见了。”

{\bfseries \color{red}这故事是说,贪得无厌的人,总会不知不觉地留下自己的罪证。}

\section{燕子与乌鸦}

燕子与乌鸦争吵谁最美丽。乌鸦对燕子说:“春天才能看到你美丽的外貌,我的身体却可以抵御冬季的严寒。”

{\bfseries \color{red}这是说,健康的身体是最漂亮的外貌。}

\section{狼与老太婆}

一只饥饿的狼四处寻觅食物。当他来到一家农舍时,突然听见小孩的哭声。一老太婆吓唬小孩说:“快别哭了,不然我马上把你丢出去喂狼。”狼听见了,信以为真,便站在门外等着。天渐渐地黑了,他又听见老太婆逗哄那小孩说:“好宝宝,如果有狼来,我就杀了他。”狼听了这话后,一边跑一边说:“这老太婆怎么说的是一套,做的又是另一套。”

{\bfseries \color{red}这故事说的是那些言行不一,表里不一的人。}

\section{主人和他的狗}

一个人打点好了行装准备出发。这时,他看见他的狗仍站在门口打呵欠,便严厉地对它说:“为什么你还站在那里打呵欠?一切准备妥当,只等你了,赶快跟我走吧!”狗摇着尾巴回答说:“主人!我早就准备好了,我等你等得都打呵欠了。”

{\bfseries \color{red}这是说有些人不检点自己,还常常把过失归咎于别人。}

\section{猴子与海豚}

出海航行的人喜欢带着一些动物,以供旅行中消遣。有个海员带着一只猴子航海,当到达雅典阿提卡的苏尼翁海峡时,一场风暴突然袭来,船被狂风巨浪打翻了,大家都纷纷跳入水中逃生,猴子也机灵地跳入水中。海豚看见了它,以为是人,立即钻到它底下,把它托起来,安全地送往岸边。到达雅典海港珀赖欧斯时,海豚问那猴子是不是雅典人。他回答说:“是的,我祖先都是名人显贵。”海琢接着又问他知不知道珀赖欧斯。猴子以为海豚所说的也是个人,所以答道:“他是我非常要好的朋友。”海豚对猴子的谎话十分气愤,便不再托住猴子,让他淹死于海水中。

{\bfseries \color{red}这故事是说那些信口雌黄的人。}

\section{受伤的狼与羊}

狼被狗所咬,伤势很严重,痛苦地躺在地上,不能外出觅食。这时,他看见一只羊,便请求他到附近的小河里为他取一点水来。他还说:“你给我一点水解渴,我就能自己去寻找食物了。”羊回答说:“如果我给你送水喝,那么我就会成为你的食物。”

{\bfseries \color{red}这故事告诉我们千万别上那些伪善的恶人的当。}

\section{农夫与争吵的儿子们}

有个农夫的儿子们常常互相争斗不休。他多次语重心长地劝说他们,尽管他苦口婆心,仍无济于事。他认为应该用事实来教育他们,便叫儿子们去拿一捆木棒来。木棒拿来后,他先把整捆木棒交给他们,叫他们折断。儿子们一个个竭尽了全力都无法将它折断。随后他解开了那捆木棒,给他们每人一根。他们都毫不费力地将木棒折为两段。这时,农夫说:“孩子们,你们要像木棒一样,团结一致,齐心协力,就不会被敌人征服;可你们互相争斗不休,便很容易被敌人打垮。”

{\bfseries \color{red}这故事说明,团结就是不可征服的力量,而内讧却只能耗损自己。}

\section{老太婆和羊}

从前,有一个贫穷的老太婆,养着一只羊。到了剪羊毛的季节,她想剪羊毛,但又不愿花钱雇请他人,就自己动手剪羊毛。她的剪毛技术不熟练,竟连毛带肉都剪下来了。那只羊痛得挣扎着说:“主人,你为什么这般伤害我?我的血和肉能增加多少羊毛重量呀?如果要我的肉,屠夫立刻就能杀死我;如果要我的毛,剪毛匠会很好的剪下我的毛,而不使我痛苦。”

{\bfseries \color{red}这故事是说最少的费用并不一定会获得最大的利益。}

\section{人与同行的狮子}

有一天,狮子与人同行赶路,他们互相吹嘘自己。在路上,他们看见一块石碑,石碑上刻着一个人征服几头狮子的图画。那人一边指给狮子看,一边说:“你看,事实证明我们比你们强得多了吧。”狮子笑着说道:“如果狮子们会雕刻,那么你就会看见众多人倒在狮子脚下。”

{\bfseries \color{red}这故事是说,那些自己毫无本事的人却喜欢常常在别人面前炫耀自己。}

\section{被狗咬的人}



{\bfseries \color{red}}

\section{马和鹿}

从前有一匹马独占一片草原。有一天,一只鹿闯入了他的领地,想与他分享草原。马对鹿的闯入十分仇视,一心想报复,便向人请求帮助惩罚鹿。人回答说,如果愿意把一块马口铁含在嘴里,并答应让人骑在马背上,他就拿出最有效的武器为马去驱逐鹿。马同意了人的要求,允许人骑在他身上。从这以后,马才知道,还没有对鹿进行报复,自己却成了人的奴隶。

{\bfseries \color{red}这是说不假思索地答应别人提出的条件,往往不但达不到自己的目的,反而会失去更多。}

\section{捕鸟人和冠雀}

捕鸟人装好了网,准备捕鸟。冠雀老远就看见了,便问他在干什么。他说正在建造一座漂亮的城市,说完就跑到远处躲藏起来。冠雀信以为真,毫不迟疑地飞进网内,结果被捉住了。捕鸟人跑来捉冠雀时,冠雀说:“喂,朋友,你建造这样的城市,决不会有更多的居民。”

{\bfseries \color{red}这故事说明,残暴的统治者会使人们宁愿舍弃城市和家园。}

\section{挂铃的狗}



{\bfseries \color{red}}

\section{行人与梧桐树}

夏季中午时候,十分炎热,头顶烈日的几个行人疲倦极了,看见一棵梧桐树,便走到树底下,躺在树荫下休息。他们仰望着阔大的树叶,彼此大发议论:“这树不能结果,对人没有什么好处。”树回答说:“不知好歹的人们!你们现在正享受着我的恩惠,还说我是不结果的无用之树。”

{\bfseries \color{red}这是说,有些人不知好歹,享受了别人的帮助,还要贬低别人。}

\section{牧人和丢失的牛}

牧人在树林中放牛,不幸丢失了一头离群的小牛犊。他在树林中到处寻找,一无所获。他发誓,只要能发现偷小牛犊的贼,他愿意供奉树林守护神一只羊。过了一会儿,当他走上小山丘时,忽然看见山下有只狮子正在吃他的小牛犊。他吓得举起双手,仰望着天空,向天祈求说:“我刚发誓,如果捉到偷牛犊的贼,我愿供奉一只羊给树林守护神。现在那贼已发现,我愿意失去那只小牛犊,并再添上一只大牛,只要我自己能安全逃离狮子。”

{\bfseries \color{red}这是说,有些人在强大的敌人面前吓破了胆,忘掉了自己的誓言。}

\section{蝮蛇和锉刀}

有条蝮蛇爬进铁匠铺里,要各种工具接济他。从他们那里得到救济之后,又爬到锉刀那里,请他也给东西。锉刀说:“你若想从我这里得到点东西去,那你真是太傻了,我历来取之于人,却从不施舍。”

{\bfseries \color{red}这故事说明,想从守财奴那里得到利益是十分愚蠢的。}

\section{芦苇与橡树}

芦苇与橡树为他们的耐力、力量和冷静争吵不休,谁也不肯认输。橡树指责芦苇说他没有力量,无论哪方的风都能轻易地把它吹倒,芦苇没有回答。过了一会儿,一阵猛烈的强风吹了过来,芦苇弯下腰,顺风仰倒,幸免于连根拔起。而橡树却硬迎着风,尽力抵抗,结果被连根拔掉了。

{\bfseries \color{red}这故事说明,有时候不要硬与比自己强大的人去抗争,或许对自己更为有利。}

\section{宙斯与众神}

宙斯与普罗米修斯与雅典娜创造万物时,宙斯创造了牛,普罗米修斯创造了人,雅典娜创造了房子。他们选举莫摩斯来评判他们的杰作。莫摩斯却嫉妒他们的创造物,便说宙斯犯了错误,应该把牛的眼睛放在角上,让牛能看见撞到什么地方。接着,他又说普罗米修斯也做错了,没有把人的心挂在体外,好让各人心里的所有想法都能表露出来,使坏人无法伪装。最后他说雅典娜应该把房屋装上轮子,若有坏人作邻居,便很容易搬迁。宙斯对莫摩斯无端的诽谤十分气愤,便把他轰出了奥林匹斯山。

{\bfseries \color{red}这故事说明,世界上没有十全十美、完美无缺的东西。}

\section{樵夫与赫耳墨斯}

有个樵夫在河边砍柴,不小心把斧子掉到河里,被河水冲走了。他坐在河岸上失声痛哭。赫耳墨斯知道了此事,很可怜他,走来问明原因后,便下到河里,捞起一把金斧子来,问是否是他的,他说不是;接着赫耳墨斯又捞起一把银斧子来问是不是他掉下去的,他仍说不是;赫耳墨斯第三次下去,捞起樵夫自己的斧子来时,樵夫说这才是自己所失掉的那一把。赫耳墨斯很赞赏樵夫为人诚实,便把金斧、银斧都作为礼物送给他。樵夫带着三把斧子回到家里,把事情经过详细地告诉了朋友们。其中有一个人十分眼红,决定也去碰碰运气,跑到河边,故意把自己的斧子丢到急流中,然后坐在那儿痛哭起来。赫耳墨斯来到在他面前,问明了他痛哭的原因,便下河捞起一把金斧子来,问是不是他所丢失的。那人高兴地说:“呀,正是;正是!”然而他那贪婪和不诚实的样子却遭到了赫耳墨斯的痛恨,不但没赏给他那把金斧子,就连他自己的那把斧子也没给他。

{\bfseries \color{red}这故事说明,诚实人会得到人们帮助,狡诈的人必遭到人们唾弃。}

\section{鹅与鹤}

鹅与鹤一起在田野上觅食。突然猎人们来了,轻盈的鹤很快飞走了。身体沉重的鹅,没来得及飞,就被捉住了。

{\bfseries \color{red}这故事是说,一无所有的人无牵无挂一身轻;而那些拥有万贯家财的人财富却成了他们的负担。}

\section{蜜蜂与宙斯}

蜜蜂不愿把自己的蜂蜜给人类,便飞到宙斯面前,请求给他强大的力量,可以用针刺死那些接近蜂窝的人。宙斯对他的恶意十分气愤,便使蜜蜂只要刺一回人,蜂针就断了,自己也随之死了。

{\bfseries \color{red}这故事适用于不怀好意、自食恶果的人。}

\section{狮子与驴子合作打猎}

狮子与驴子联合,一起外出打猎。他们来到野羊居住的山洞。狮子守在洞口监视着,驴子则跑进洞里,乱喊乱跳,吓唬野羊,把他们赶出去,守候在洞口的狮子捕捉了许多野羊。之后,驴子跑出洞来,问狮子他是否很勇敢,野羊都被轰赶出来了。狮子答道:“是呀!如果我不知道你是野驴子,我也许会害怕你。”

{\bfseries \color{red}这是说,那些在能人和行家面前自吹自擂的人,自然会被世人讥笑。}

\section{山羊与牧羊人}

很多山羊被牧羊人赶到羊圈里。有一只山羊不知在吃什么好东西,单独落在后面。牧羊人拿起一块石头扔了过去,正巧打断了山羊的一只角。牧羊人吓得请求山羊不要告诉主人,山羊说:“即使我不说,又怎能隐瞒下去呢?我的角已断了,这是十分明显的事实。”

{\bfseries \color{red}这故事说明,明显的罪状是无法隐瞒的。}

\section{挤牛奶的姑娘}

一个农家挤奶姑娘头顶着一桶牛奶,从田野里走回农庄。她忽然想入非非:“这桶牛奶卖得的钱,至少可以买回三百个鸡蛋。除去意外损失,这些鸡蛋可以孵得二百五十只小鸡。到鸡价涨得最高时,便可以拿这些小鸡到市场中去卖。那么这样一年到头,我便可分得很多赏钱,用这些钱足够买一条漂亮的新裙子。圣诞节晚宴上,我穿上漂亮迷人的新裙子,年青的小伙子们都会向我求婚,而我却要摇摇头拒绝他们。”想到这里,她真的摇起头来,头顶的牛奶倒在地上。她的美妙幻想也随之消失了。

{\bfseries \color{red}这是说,想入非非不会给自己带来任何实惠。}

\section{牛和屠夫}

有一天,许多牛想杀死宰牛的屠夫,因为屠夫从事屠杀他们的职业。他们聚集在一起,商讨办法,磨砺他们的角,准备战斗。有一头耕过许多田地的老牛说:“屠夫们确实宰杀我们,但他们是用精巧的手艺来杀我们,减少了我们的痛苦。如果没有这些手艺高明的屠夫,而让其他人来宰杀,我们便更加痛苦了。你们要知道,虽然屠夫可以杀死,但人们总是要吃牛肉的。”

{\bfseries \color{red}这是说如果灾难和死亡是不可避免时,就要勇敢地面对它,与其痛苦而死,不如痛快而死。}

\section{偷东西的小孩与他母亲}

有个小孩在学校里偷了同学一块写字石板,拿回家交给母亲。母亲不但没批评,反而还夸他能干。第二次他偷回家一件大衣,交给母亲,母亲很满意,更加夸奖他。随着岁月的流逝,小孩长大成小伙子了,便开始去偷更大的东西。有一次,他被当场捉住,反绑着双手,被押送到刽子手那里。他母亲跟在后面,捶胸痛哭。这时,小偷说,他想和母亲贴耳说一句话。他母亲马上走了上去,儿子一下猛地用力咬住她的耳朵,并撕了下来。母亲骂他不孝,犯杀头之罪还不够,还要使母亲致残。儿子说道:“我初次偷石板交给你时,如果你能打我一顿,今天我何至于落到这种可悲的结局,被押去处死呢?”

{\bfseries \color{red}这故事说明,小错起初不惩治,必将酿成大错。}

\section{猫和鼠}

从前,有一户人,家里有许多老鼠。猫知道后,便跑到那里去,毫不留情地抓住一只消灭一只。老鼠因为不断被杀,都躲入鼠洞里。猫再也不能抓到他们,就想出妙计来引鼠出洞。猫爬在一把木橛上,吊在上面装死。有只老鼠出来窥探一下,见到猫的情形,说,“呵,伙计,你哪怕变成一只皮袋,我也决不到你的跟前去呢。”

{\bfseries \color{red}这故事说明,聪明人吃一亏,长一智,不会再受伪装的欺骗了。}

\section{太阳结婚}

夏天,太阳举行了婚礼。所有的动物都高高兴兴,青蛙也欢天喜地。有一只青蛙却说:“傻子们,你们为什么还高兴呢?一个太阳都能把烂泥晒干,现在他又结了婚,如再生下一个与他一样的儿子,那我们不知还要吃怎样的苦呢?”

{\bfseries \color{red}这就是说,许多缺乏思想的人,只会跟随他人瞎起哄。}

\section{蚊子与公牛}

蚊子飞到公牛角上,休息了很久。他在要飞走时,问公牛是不是希望他离开。公牛回答说:“你来时我一点儿都不知道,你离去我也未必会在意。”

{\bfseries \color{red}这是说,对于那些既软弱又无知的人,存在与否,人们都觉得无关紧要。}

\section{被射伤的鹰}

鹰站立在岩石上,想要去捕捉一只兔子。有人一箭射中了他,那箭扎入他的身上,带着鹰毛的箭翎却留在鹰的眼前。他望着羽翎说:“我自己的羽毛害死了我,这种痛苦更难以忍受。”

{\bfseries \color{red}这就是说,因自己的原因而受害,那痛苦更令人难受。}

\section{马槽中的狗}

一条狗躺在马槽中,不停地叫,不让马吃干草。一匹马对同伴说:“这条狗太自私了!他自己不会吃干草,还不让会吃的去吃。”

{\bfseries \color{red}这故事是说那些总是不愿别人得到好处的人。}

\section{老鼠开会}

很久很久以前,老鼠们因深受猫的侵袭,感到十分苦恼。于是,他们在一起开会,商量用什么办法对付猫的骚扰,以求平安。会上,各有各的主张,但都被否决了。最后一只小老鼠站起来提议,他说在猫的脖子上挂个铃铛,只要听到铃铛一响,我们就知道猫来了,便可马上逃跑。大家对他的建议报以热烈的掌声,并一致通过。有一只年老的老鼠坐在一旁,始终一声没吭。这时,他站起来说:“小鼠想出的这个办法是非常绝妙的,也是十分稳妥的;但还有一个小问题需要解决,那就是派谁去把铃铛挂在猫的脖子上?”

{\bfseries \color{red}这故事是说,想出一个好主意也许不难,实现主意就不那么容易了。}

\section{狮子、熊和狐狸}

狮子和熊同时抓到一只小羊羔。他们俩为争夺小羊凶狠地打了起来。经过一场苦斗,双方都受了重伤,有气无力地躺在地上。狐狸早已躲在远处坐山观虎斗,一见他们两败俱伤,都直挺挺地躺在地上,便跑过去,把躺在他们俩之间的羊羔抢了去。伤势严重的狮子和熊眼睁睁地看着狐狸抢走了羊,却毫无办法。他们唉声叹气地说:“我们都错了,我们俩斗得你死我活,让狐狸得到了好处。”

{\bfseries \color{red}这故事正如俗话所说:“鹬蚌相争,渔人得利。”双方相争让第三者得了利。}

\section{狐狸和刺猬}

一只狐狸渡过湍急的河水时,被冲到一个深谷中。他遍体鳞伤,躺在地上一动也不能动。一群饥饿的吸血蚊蝇叮满了他的全身。这时,一个刺猬走了过来,十分可怜他的痛苦,问需不需要赶开这些害他的蚊蝇。狐狸回答说:“不用啦,请你不要打扰他们。”刺猬感到奇怪:“为什么不要把他们赶跑呢?”狐狸回答说:“千万不要,你所见到的这些蚊蝇已吸足了我的血,不再叮咬我了。你若替我赶跑他们,那另一些更饥饿的就会来将我所剩的血吸干。”

{\bfseries \color{red}这是说,与其忍受两次折磨,不如将一次折磨忍受到底。}

\section{生金蛋的鹅}

有个人养有一只鹅,鹅生下美丽的金蛋。那人以为鹅肚里一定有金块,便把它杀了,结果发现它与别的鹅完全一样。他贪婪地希望得到更多的财富,却把微小的利益也失去了。

{\bfseries \color{red}这故事说明,人们应该满足于现有的东西,切不可贪得无厌,杀鸡取卵。}

\section{狮子和海豚}

狮子在海滩上游荡,看见海豚跃出水面,便劝他与自己结为同盟,说他们是一对最好的搭挡,因为一个是海中动物之王,一个是陆地兽中之王。海豚立即高兴地答应了。不久,狮子和野牛展开了一场生死搏杀,他请求海豚助他一臂之力。尽管海豚想出海助战,却办不到。狮子指责他背信弃义。他回答说:“不要责备我,去责备自然吧!因为它让我成为海里的动物,不许上陆地呀!”

{\bfseries \color{red}这是说,我们结交盟友,应当选择那些能共患难的人。}

\section{号兵}

从前,有个号兵,正当他吹集合号时,被敌人抓获了,他大声叫道:“各位,请不要无缘无故地杀我。因为我没有杀害你方任何人,我仅有这把铜号,没有任何武器。”敌人却对他说道:“正因为这样,你就更应当被杀死,你自己虽没打仗,可你召集所有士兵来攻打我们。”

{\bfseries \color{red}这故事说明,人们更痛恨那些怂恿他人作恶的人。}

\section{夜莺与鹞子}

夜莺站在一棵大树上,像往常一样歌唱着。饥饿的鹞子看见她后,便猛飞过去将她抓住。夜莺临死时,请求鹞饶了她,说她难以充满鹞的肚子,如要彻底解决饥饿,应当去抓捕更大的鸟。鹞子却回答说:“若我放弃了手中现成的食物,再去追求看不见的东西,那我岂不是傻瓜了么。”

{\bfseries \color{red}这故事是说,那些为贪图更大的利益,而放弃已到手的东西的人,是愚蠢的人。}

\section{夜莺与燕子}



{\bfseries \color{red}}

\section{作客的狗}

一天,有个人大摆酒席,设宴招待亲朋好友。这人家里的狗也高兴地跑去请另一只狗,说:“朋友,快走,请你和我一起去赴宴。”那狗兴高采烈地跑来,见到如此丰盛的筵席,他心里暗暗地说,“太好啦!真想不到天底下还有这么多好吃的!让我饱吃一顿,明天都不会肚子饿。”他独自暗暗地窃喜,不停地摇着尾巴,十分信任地看着他的朋友。正在这时,厨师看见狗尾巴在那里四处乱摇,立刻抓住他的腿,从窗口丢到外边去了。那狗摔得大声叫唤,惊慌地跑了回去。路上别的狗遇见他时,都问他:“朋友,宴会怎么样呀?”他回答道:“我喝得太多了,已经醉了,所以我记不请回去的路了。”

{\bfseries \color{red}这故事说明,对那些慷他人之慨的人不可信任。}

\section{青蛙求王}

青蛙们没有国王,大为不快。于是,他们派代表去拜见宙斯,请求给他们一个国王。宙斯看到他们如此蠢笨,便将一块木头扔到了池塘里。青蛙最初听到木头落下的声音吓了一大跳,立刻都潜到池塘的底下去了。后来当木头浮在水面一点不动时,他们又游出水面来,终于发现木头没什么了不起的,大家爬上木块,坐在它的上面,把开始时的害怕忘得一干二净。他们觉得有这样一个国王很没面子,又去见宙斯,请求给他们换一个国王,说第一个国王太迟钝了。宙斯感到十分生气,便派一条水蛇到他们那里去。结果青蛙都被水蛇抓去吃了。

{\bfseries \color{red}这故事说明,迷信统治者,不相信自己的力量,只能受制于人,招致灾难。}

\section{白松与荆棘}

白松与荆棘互相争吵。白松自傲地说:“我质地优良,躯干粗壮,既可以做庙宇的屋顶,又可建造船只,你能做什么呢?”荆棘说:“如果你一想到劈你的斧头和锯你的锯子,你恐怕还是愿意做荆棘吧。”

{\bfseries \color{red}这故事是说,平淡无奇的生活也许比富有离奇的生活更无痛苦和危险些。}

\section{百灵鸟和小鸟}

早春时节,一只百灵鸟飞到嫩绿色的麦田做巢。小百灵们的羽毛慢慢地丰满了,力气也渐渐地长足了。有一天,麦田的主人见到已成熟的麦子,便说:“收割的时候到了,我一定要去请所有的邻居来帮助收获。”一只小百灵鸟听到这话后,便赶忙告诉她妈妈,并问该搬到什么地方去住。百灵鸟说:“孩子,他并不是真的急切要收获,只是想请他的邻居来帮他的忙。”几天过后,那主人又来了,看到麦子熟透得掉了下来,急切地说:“明天我自己带上家里的帮工和可能雇到的人来收获。”百灵鸟听到这些话后,便向小鸟们说:“现在我们该搬家了,因为主人这一次真的急起来了。他不再依赖邻居,而要自己亲自动手干。”

{\bfseries \color{red}这是说不寄希望于外力,而是自己亲自动手干,这才是真正下决心了。}

\section{击水的渔夫}

渔夫在河里拦河张网捕鱼,用麻绳缠住石块,再不停地打击河水,吓得鱼群仓皇逃窜,都钻进了他的网中。附近的一个人见到后,指责他这样把河水弄浑,大家都没清水喝了。渔夫回答说:“若不搅浑河水,我就非饿死不可。”

{\bfseries \color{red}这故事是说,有些人如同这个渔夫一样,为了自己的私利,不惜把事情搞混搞乱,再从中渔利。}

\section{贼和看家狗}

一个宁静的夜晚,一个贼悄悄地溜入一户人家的院子。为了防止狗吠叫喊醒主人和追咬自己,贼特意随身带了几块肉。当他把肉给狗吃的时候,狗说:“你若想这样来堵住我的嘴,那就大错特错了。你这样无缘无故、突如其来地送给我肉,一定是别有用心,不怀好意的,肯定是为了你自己的利益,想伤害我的主人。”

{\bfseries \color{red}这是说忠心的狗不受肉的贿赂,每个人都应忠于职守,抵制诱惑。}

\section{驴子与农夫}

驴子给农夫干繁重的活,却吃得很少。他跑去请求宙斯,让他脱离农夫,卖到别的主人那里去。于是宙斯把他卖给一个陶工,陶工让他搬运沉重的粘土和陶器,比以前更劳累。他又请求宙斯再给他换一个主人,宙斯又把他卖给了一个皮匠。他一到皮匠那里,看到要干的活,后悔不已地说:“我真不幸!留在以前的那些主人那里该多好啊!现在连我的皮都得交给这个主人了。”

{\bfseries \color{red}这故事说明,许多人总是抱怨自己的生活不好,却并不了解别人的生活同样也有不如意的地方。}

\section{老人与死神}

有一天,有个老人砍了不少柴,十分吃力地挑着走了很远的路。一路上他累极了,实在挑不动了,便将担子放下,叫喊起死神来。死神来后,问他为什么叫喊他,老人说:“尽管我已精疲力竭了,但还是请你把那担子放在我肩上。”

{\bfseries \color{red}这故事说明,即使生活不幸,人们仍爱惜生命。}

\section{医生与病人}

一个医生给病人治病。病人最终还是死了,医生对那些送葬的人们说:“如果病人生前戒了酒,洗了肠,就不至于丧命。”在场的另一个人回答说:“高明的大夫,事到如今,你说这些话,已是毫无用处的了,你应该在病人生前患病的时候,用这些话去劝告他。”

{\bfseries \color{red}这故事说明,当朋友处于困难的时候,应及时给与帮助,而不应该在事后去说一些毫无用处的空话。}

\section{鸟、兽和蝙蝠}

鸟与野兽宣战,双方各有胜负。蝙蝠总是依附强的一方。当鸟和兽宣告停战和平时,交战双方明白了蝙蝠的欺骗行为。因此,双方都裁定他为奸诈罪,并把他赶出日光之外。从此以后,蝙蝠总是躲藏在黑暗的地方,只是在晚上才独自飞出来。

{\bfseries \color{red}这故事是说那些两面三刀的人,最终不会有好下场。}

\section{两个锅}

河中漂流着一个瓦锅和一个铜锅。瓦锅对铜锅说:“请离我远一些,不要靠近我。那怕是我自己不小心碰到你,我也会碰碎。”

{\bfseries \color{red}这是说,与强硬的人相伴是很不安全的。}

\section{猫和生病的鸡}

猫听说有只鸡生病了,便装扮成医生,带着医疗用品前去看望。他站在鸡窝前面,耐心地询问鸡哪里不舒服。鸡回答说:“很好,只要你离开这儿,我就不会死。”

{\bfseries \color{red}这故事说明,坏人即使装出十分善良的样子,聪明的人也会知道他们是口蜜腹剑的人。}

\section{狼与母山羊}

母山羊在陡峭的山崖上吃草,狼无法捉到他,便劝说她赶快下来,免得一不小心掉进山谷里,还说在自己身边的草地好得多,青草茂盛鲜嫩,还有许多花。母山羊回答说,“你不是真心喊我去吃草,而是让我去填饱你的肚子。”

{\bfseries \color{red}这是说,尽管坏人老奸巨滑,但在聪明人面前,他们的诡计仍是枉费心机。}

\section{骆驼和阿拉伯人}

一个阿拉伯的骆驼夫把货满载在骆驼背上后,问骆驼是喜欢上山还是喜欢下山。骆驼振振有词地说:“你为什么这样问我?难道经过沙漠的平坦大道都关闭了吗?”

{\bfseries \color{red}这是说,不了解事物的特性就不可能正确使用它。}

\section{狼与牧羊人}

狼老老实实地跟随着羊群,一点坏事也没干。牧羊人开始一直把他当作敌人一样小心防范,提心吊胆,十分警惕地看护着羊。狼却一声不吭地跟着走,丝毫没有想抢羊的迹象。后来牧羊人不再提防狼,反而认为这是一头老实的护羊犬。一次,牧羊人因事须进城一趟,便把羊留下交给狼守护。于是,狼乘此机会,咬死了大部分的羊。牧羊人回来,看见羊群被咬死了,十分后悔,并说道:“我真活该,我为什么把羊群托付给狼呢?”

{\bfseries \color{red}这是说,把财物托付给不应托付的人,自然会上当。}

\section{行人与斧头}

两个人一起赶路。一个人拾到一把斧头,另一个人对他说:“我们拾到了一把斧子。”那人说:“不能说‘我们拾到了’,而是‘我拾到了’。”过了一会儿,那丢斧头的人追上了他们,斧子被要了回去。拾到斧子的人对同伴说:“我们完了。”另一个说:“你不要说‘我们完了’,而要说‘我完了’,因为在拾到那斧子时,并没有将它作为我们共有的东西呀。”

{\bfseries \color{red}这故事说明,那些有福不愿与人同享的人,有祸也没人与他同担。}

\section{驴子、狐狸与狮子}

驴子与狐狸俩合伙去打猎。他们突然遇见了狮子,狐狸见大事不妙,立即跑到狮子面前,许诺把驴子交给他,只要自己免于危险。狮子答应可以,狐狸便引诱驴子掉进了一个陷阱里。狮子见驴子已不能再逃跑,便立即先抓住狐狸吃了,然后再去吃驴子。

{\bfseries \color{red}这是说,那些出卖朋友,背叛友谊的人也得不到好下场。}

\section{欠债的雅典人}

在雅典有一个欠债的人,债主三番五次来催索拖欠的钱,他总推说没钱,请求延期。因为债主不答应,他迫不得已把家中仅有的母猪赶出来,当场出卖。买主走上前来,问这猪还能下小猪崽么。他回答说,她不但能下小猪,而且还下得很多,不同寻常,因为她在土地女神节会生下些小母猪,而在雅典娜节会生下些小公猪。买主听了这话大为吃惊。债主说道:“这毫不奇怪,因为她在酒神节还会为你生些小山羊哩。”

{\bfseries \color{red}这故事说明,有些人为了自己的利益,会不惜为不可能的事情作伪证。}

\section{狮子和野驴}

狮子与野驴一起外出打猎,狮子力气大,野驴跑得快。他们抓获了许多野兽。狮子把猎物分开,堆成三份,说道:“这第一份,该我拿,因为我是王。第二份也该是我的,把它算作我和你一起合作的报酬。至于第三份呢?如果你不准备逃走,也许会对你有大害。”

{\bfseries \color{red}这是说,人们对自己的力量和能力须实事求是,正确估量,不要去与比自己强大得多的人交际和合作。}

\section{驴子和驴夫}

驴夫赶着驴子上路,刚走一会儿,就离开了平坦的大道,沿着陡峭的山路走去。当驴子将要滑下悬崖时,驴夫一把抓住他的尾巴,想要把他拉上来。可驴子拼命挣扎,驴夫便放开了他,说道:“让你得胜吧!但那是个悲惨的胜利。”

{\bfseries \color{red}这故事说明,事事争胜好斗不会有好下场。}

\section{老鼠与黄鼠狼}

老鼠与黄鼠狼开战。老鼠每次总被打败,他们聚在一起商议,认为是因为没有将帅所以屡次失败,于是他们举手表决选举了几只老鼠作将帅。这些将帅想要使自己显得与众不同,做了些角绑在头上。战事又起,老鼠再次被打败。别的老鼠很容易地逃进了洞里,而那些将帅因头上有角不能钻进去,全都被黄鼠狼吃掉了。

{\bfseries \color{red}这是说,对于许多人来说,虚荣为不幸的根源。}

\section{鹿与葡萄藤}

有只鹿为逃避猎人的追捕,躲藏在葡萄藤底下。猎人刚刚从旁边走过去不远,鹿就以为躲过了危险,便毫无顾及地开始吃那茂盛的葡萄叶子。叶子沙沙地抖动着,猎人们马上掉回头来,觉得叶子底下一定躲着什么动物,一箭就把鹿射中了。鹿在临死前说:“我真是该死,因为我不应该去伤害救我的葡萄藤。”

{\bfseries \color{red}这故事说明,那些恩将仇报的人将会被神惩罚。}

\section{篱笆与葡萄园}

一个愚蠢的年轻人继承了父亲的家业。他砍掉葡萄园四周所有的篱笆,因为篱笆不能结葡萄。篱笆砍掉以后,人和野兽都能随意侵入葡萄园。没过多久,所有的葡萄树全都被毁坏了。那蠢家伙见到如此情景,才恍然大悟:虽然篱笆结不出一颗葡萄,但它们能保护葡萄园,它和葡萄树一样同等重要。

{\bfseries \color{red}这故事是说,红花虽好,还要绿叶扶持。}

\section{狐狸与面具}

狐狸走进演员的家里,仔细察看他所有的家当后,发现了一个制作精巧的妖怪面具,便连忙把它拿在手里说:“喂,这是谁的头,真可惜没有脑子!”

{\bfseries \color{red}这故事是说那身体魁伟而缺乏思想的人。}

\section{父亲与女儿}

父亲有两个女儿,一个嫁了菜农,另一个嫁给了陶工。过了些日子,父亲来到菜农家里,问女儿情况如何,他们的生活过得怎么样。女儿说一切都很好,只是有一事须祈祷神明,那就是请求多下雨,好好地浇灌那些蔬菜。不久之后,他又来到陶工家里,问女儿过得如何。女儿说什么都不缺,只祷告一件事,请求天气晴朗,阳光充足,使陶器更快地干燥。父亲对她说道:“你望出太阳,你的妹妹却盼下雨,那么我又为谁祈求呢?”

{\bfseries \color{red}这故事是说,那些同时想做两件截然不同的事的人,必然任何一件事都干不成。}

\section{马与驴子}

从前,有个人赶着一匹马和一头驴子上路。路途中,驴子对马说:“你若能救我一命,就请帮我分担一点我的负担吧。”马不愿意,驴子终因精疲力竭,倒下死了。于是,主人把所有的货物,包括那张驴子皮,都放在马背上。这时,马悲伤地说:“我真倒霉!我怎么会受这么大的苦呢?这全因不愿分担一点驴的负担,现在不但驮上全部的货物,还多加了一张驴皮。”

{\bfseries \color{red}这故事说明,强者与弱者应相互帮助,共同合作,各自才能更好地生存。}

\section{老狮子与狐狸}

有一头年老的狮子,已不能凭借力量去抢夺食物了,心想只能用智取的办法才能获得更多的食物。于是,他钻进一个山洞里,躺在地上假装生病,等其他小动物走过来窥探,就把他们抓住吃了。这样,不少的动物都被狮子吃掉了。狐狸识破了狮子的诡计,远远地站在洞外,问狮子身体现在如何。狮子回答说:“很不好。”反问狐狸为什么不进洞里来。狐狸说道:“如果我没发现只有进去的脚印,没有一个出来的脚印,我也许会进洞去。”

{\bfseries \color{red}这是说,聪明的人常常能审时度势,根据迹象预见到危险,避免不幸。}

\section{山羊与驴}

有个人饲养着山羊和驴子。主人总是给驴子喂充足的饲料,嫉妒心很重的山羊便对驴子说,你一会儿要推磨,一会儿又要驮沉重的货物,十分辛苦,不如装病,摔倒在地上,便可以得到休息。驴子听从了山羊的劝告,摔得遍体鳞伤。主人请来医生,为他治疗。医生说要将山羊的心肺熬汤作药给他喝,才可以治好。于是,主人马上杀掉山羊去为驴子治病。

{\bfseries \color{red}这故事说,凡是策划作恶的人,将自食其果。}

\section{鹰与乌鸦}

鹰从高岩直飞而下,把一只羊羔抓走了。一只乌鸦见到后,非常羡慕,很想仿效。于是,他呼啦啦地猛扑到一只公羊背上,狠命地想把他带走,然而他的脚爪却被羊毛缠住了,拔也拔不出来。尽管他不断地使劲拍打着翅膀,但仍飞不起来。牧羊人见到后,跑过去将他一把抓住,剪去他翅膀上的羽毛。傍晚,他带着乌鸦回家,交给了他的孩子们。孩子们问这是什么鸟,他回答说,“这确确实实是乌鸦,可他自己硬要充当老鹰。”

{\bfseries \color{red}这故事是说,仿效别人却做自己力所不能及的事,不仅得不到什么益处,还会给自己带来不幸,并受世人的嘲笑。}

\section{口渴的鸽子}

有只鸽子口渴得很难受,看见画板上画着一个水瓶,以为是真的。他立刻呼呼地猛飞过去,不料一头碰撞在画板上,折断了翅膀,摔在地上,被人轻易地捉住了。

{\bfseries \color{red}这是说,有些人想急于得到所需的东西,一时冲动,草率从事,就会身遭不幸。}

\section{小母牛与公牛}

小母牛看见公牛在辛苦地干活,十分可怜他。可是祭祀时,主人家不用公牛,却捉住小母牛去宰杀。这时,公牛笑着对她说:“喂,小母牛,正因为你要作祭品,所以你就什么活都不必干。”

{\bfseries \color{red}这故事说明,危险等着那些游手好闲的人。}

\section{秃头武士}

有个秃头的武士,头戴假发,骑着马奔跑去打猎。突然,一阵风把他的假发吹跑了,他的同伴都情不自禁地放声大笑。秃子勒住马说:“这假发本来就不是我的,从我头上飞去,又有什么奇怪呢?这假发不也是早已离开了那生长它的原主人吗?”

{\bfseries \color{red}这是说,人们不必为突然失去的东西所苦恼。原本不是你的东西想留也留不住,是你的就永远跑不掉。}

\section{狐狸和鹤}

狐狸请鹤来吃晚饭。然而他并没有真心真意地准备什么饭菜来款待客人,仅仅只用豆子做了一点汤,并把汤倒在一个很平很平的石盘子中,鹤每喝一口汤,汤便从他的长嘴中流出来,怎么也吃不到。鹤十分气恼,狐狸却十分开心。后来,鹤回请狐狸吃晚饭,他在狐狸面前,摆了一只长颈小口的瓶子,自己很容易地把头颈伸进去,从容地吃到瓶里的饭菜,而狐狸却一口都尝不到。狐狸受到了应得的回报。

{\bfseries \color{red}这是说若要他人尊重自己,自己首先须尊重他人;同时,告诉我们,对待那些不尊重他人的人,最好的办法是以其人之道还治其人之身。}

\section{斑鸠与人}

有个人捕捉到一只斑鸠,要杀死他。斑鸠请求赦免,并说:“请饶恕我吧,我会为你捉到更多斑鸠。”那人说道:“你更要被杀,不然你的亲戚朋友将会遭受你的陷害。”

{\bfseries \color{red}这故事是说,那些用阴谋诡计加害亲人的人,必将先得到正义的惩罚。}

\section{牧人与野山羊}

牧人把羊群赶到牧场去放牧,看见有几只野山羊混杂在羊群里。傍晚,他将所有的羊都赶进羊圈。第二天,暴风雨大作,不能到牧场去放牧,只好在羊圈里饲养。他丢给自己的羊一点点食料,仅只限于不至饿死,而为了想把外来的那几只野山羊留下,成为自己的,他却给他们很多食料。雨停后,牧人把所有的羊都赶向牧场,来到山下时,那些野山羊全都逃跑了。牧人指责他们忘恩负义,得到了特殊照顾,却仍要逃走。野山羊回过头来说:“正因如此,我们更要小心谨慎了。因为你特殊照顾我们这些昨天刚来的,而过于冷淡你以前一直饲养的。显而易见,今后再有其他的野山羊来,你一定又会冷落我们去偏爱他们。”

{\bfseries \color{red}这故事说明,喜新厌旧的人的友谊是不可信的。因为即使同他相交很久,他一有新交,便会冷落旧交。}

\section{遇难的人与海}

有个在海上遇难的人被冲上海岸,他躺在地上,疲劳地睡着了。不一会儿,他坐起来,看着大海,指责大海总是以平静、温和的外表引诱人们。当人们上当后,大海就变得凶暴和残忍,最终把人们毁灭了。这时,海变成一个女人对他说:“喂,朋友,你别责怪我,应该责怪风!我本是非常平静的,是风忽然猛刮过来,掀起了惊涛骇浪,使我变得残暴了。”

{\bfseries \color{red}这是说,有些人惯于找借口,推却自己的责任。}

\section{运神像的驴子}

有个人把神像放在驴子背上,赶着进城去。凡是遇见他们的人都对着神像顶礼膜拜。驴子以为人们是向它致敬,便洋洋得意,大喊大叫,再也不肯往前走了。驴夫见到这情形,明白了是什么回事,立刻狠狠地给他一棍,并骂道:“喂,这蠢东西,人们给驴子鞠躬的时候还早得很哩!”

{\bfseries \color{red}这故事说明,那些依靠别人获得尊敬的人太不自量了。}

\section{小猪与羊群}

有头小猪混进了羊群里,和羊一起吃食料。后来,牧人发现了,捉住了他,他竭力嚎叫,又拚命挣扎。羊群指责他大喊大叫,说:“我们常常被牧人捉,可从来不这样叫喊。”小猪对他们说:“我被捉与你们被捉是两件不同的事,捉你们仅为了毛或奶,捉我却是为了吃我的肉呀。”

{\bfseries \color{red}这故事说明,真正的危险不是关系钱财,而是关系生命。}

\section{猎狗与野兔}

猎狗抓住了一只野兔,一会儿咬他,一会儿舔他的嘴唇,不断地作弄他玩。野兔竭力抗拒,并对猎狗说:“喂,你这家伙,请你不要又咬又亲,我难以判断,你究竟是我的敌人,还是朋友。”

{\bfseries \color{red}这故事适用于态度暧mei的人。}

\section{小孩与栗子}

一个小孩把手伸进装满栗子的瓶中,他想尽可能地抓一大把。但当他想要抻出手来时,手却被瓶口卡住了。他既不愿放弃一部分栗子,又不能拿出手来,只好痛哭流涕。一个行人对他说:“你还是知足些吧,只要少拿一半,你的手就能很容易地拿出来了。”

{\bfseries \color{red}这是说一次不要贪多,人一定要知足。}

\section{小山羊与吹xiao的狼}

小山羊落在羊群后面,被狼所追赶。他回过头来,对狼说:“狼啊,我知道我将成为你口中之食,不要让我默默无闻而死吧,请你吹xiao,我来跳一回舞。”于是,狼吹着箫,小山羊跳起舞来,狗听到后跑来追赶狼。狼回过头来对小山羊说:“我真活该,我本是拿屠刀的,不该学着吹xiao呀。”

{\bfseries \color{red}这故事说,有些人不守本分,最终往往会失败。}

\section{青蛙庸医}

一只青蛙从洼地里潮湿的家蹦了出来,大声对所有的野兽宣称:“我是一个医术高明、能治百病、博学多才的医生!”一只狐狸问他:“你连自己跛足的姿势和起皱的皮都不能治,怎么还吹牛说能给别人治病呢?”

{\bfseries \color{red}这故事是说判断人们的知识和才能需要听其言,观其行,不要被他的花言巧语所迷惑。}

\section{蚂蚁与鸽子}

口渴的蚂蚁,爬到泉水旁去喝水,不幸被急流冲走。快要淹死时,鸽子看见他,连忙折断一根树枝,扔到水里,蚂蚁赶紧爬了上去,脱离了危险。后来,一个捕鸟人走来,用粘竿捕捉那只鸽子。蚂蚁见了,便咬了捕鸟人的脚一口。捕鸟人痛得丢下粘竿,鸽子立即惊跑了。

{\bfseries \color{red}这故事说明,人们应该知恩图报。}

\section{披着狮皮的驴}

驴子披着狮子的皮,到处走动,吓唬别的动物。动物们都以为他真是狮子,吓得四处逃跑。突然一阵风刮来,把驴子披着的狮皮吹走了,驴子原形毕露。这时,动物们一见又都跑回来,用木板和棍棒狠狠地打他。

{\bfseries \color{red}这故事说明,那些狐假虎威,仗势欺人的人必将遭到世人痛恨,自取灭亡。}

\section{伊索在造船厂}

闲暇的时候,善讲故事的伊索来到了造船厂。有些造船工人与他开玩笑,逗他说话。伊索说在古时候到处是一片混沌和水,但宙斯想要土出现,便叫土分三次喝干海水。土第一口喝下去,最先奇迹般地出现了山峰,第二口喝下去时,一片原野展现在眼前。伊索接着又说:“他若再喝第三口,那么你们这点技艺就将毫无用处了。”

{\bfseries \color{red}这故事说明,嘲弄比自己高明的人,往往会自讨没趣。}

\section{洗澡的小男孩}

有一天,有个小男孩在河里洗澡,遇到了危险,眼看要被淹死时,看见有人路过,连忙大声呼救。然而,那人却责备小男孩太鲁莽和太冒险。小孩回答:“请你还是先把我救起来,再责备吧。”

{\bfseries \color{red}这故事是说,该说的时候说,该做的时候做。}

\section{农夫和狗}

农夫持续被风暴困在自己家里,粮草都没有了,又无法出去为自己弄食物,迫于无奈,他吃掉了绵羊。可是风暴仍继续不停,山羊也被他吃掉了。后来,风暴丝毫没减弱,他又吃掉了那耕田的牛。那些狗看见主人的所作所为,互相说道:“我们得赶紧离开这里,主人连帮他一起辛勤耕作的牛都宰了,又怎能放过我们呢?”

{\bfseries \color{red}这故事说明,那些对于家人都要伤害的人,须要特别警惕。}

\section{狮子与农夫}

有只狮子闯到农夫家的畜圈里,农夫想要捉住他,马上把院子的大门紧紧关上了。狮子因跑不出去,便先咬死了一些羊,随后又朝那些牛冲去。农夫害怕自身难保,便将门打开,让狮子出去。狮子逃走之后,妻子对悲叹不已的丈夫说:“你活该!人们都远离可怕的狮子,你为什么还要把他关起来呢?”

{\bfseries \color{red}这是说,去激怒比自己更强大的人,必然会自讨苦吃。}

\section{马与驴子}

驴子看到主人精心照料马,并给他丰富的饲料,想到自己连糠麸都不够吃,还要做十分繁重的工作,便悲伤地对马说:“你真幸福!”当战事爆发时,全副武装的战士骑着马,奔驰于战场,不顾枪林弹雨,冲锋陷阵。马不幸受伤倒下,驴子见到后,不再觉得马比自己幸福,反而觉得马比自己更可怜。

{\bfseries \color{red}这故事说明,不要随便羡慕别人,各人都有自己的生活,都有自己的幸福与不幸。}

\section{铁匠与小狗}

铁匠家有一条狗,他打铁时,狗就睡觉,吃饭时狗便立刻跑到铁匠的身旁摇头摆尾,讨好主人。铁匠扔给狗一块骨头,并说道:“你这家伙,总是贪睡。为什么那沉重的打铁声丝毫没影响你的睡眠,而我们吃饭时轻微的响动声却使你惊醒?”

{\bfseries \color{red}这故事是说,那些唯利是图的人,对于自己有利的事专心致志,对自己无利的事则不闻不问。}

\section{丑陋的女仆与阿佛洛狄忒}

从前有一家主人爱上丑陋心险的女仆。她用很多金钱把自己打扮得很华丽,去同女主人比美。她还不断地祭奉阿佛洛狄忒,请求将自己变得漂亮些。在睡梦中的阿佛洛狄忒对女仆说不能为了她的祭品而赐给她美貌,并对以她为美的人感到气愤。

{\bfseries \color{red}这就是说,凡是用卑鄙的手段致富的人,不要得意忘形,别忘了自己是什么身份,别忘了自己的本来面目。}

\section{狼与狮子}

有一次,狼从羊群中抢走一只羊,正叼着羊往回走,碰上了狮子。狮子立刻从狼口里抢走羊。狼远远地站着,自言自语地说:“你抢我的东西太不正当了。”那狮子笑着说:“那么,这东西是朋友正当地赠送给你的吗?”

{\bfseries \color{red}这故事是说,窃贼和强盗都是一丘之貉,没有好坏之分。}

\section{渔夫与大鱼和小鱼}

渔夫从海里拉起鱼网来,他立即抓住网里的大鱼,扔到岸上,而那些小鱼却从网眼中逃回到海里。

{\bfseries \color{red}这就是说,小人物容易得救,而那些名声大的却难以躲过危险。俗话说,人怕出名,猪怕壮。}

\section{孩子和青蛙}

几个孩子在水潭边玩耍,看见水中有许多青蛙,便用石头去打他们。几只青蛙被他们打死了。这时,一只青蛙从水中伸出头来说:“孩子们,请你们不要再打了。这对于你们来说是在做游戏,而对于我们却有性命之忧啊。”

{\bfseries \color{red}这是说,不要把自己的快乐建筑在别人的痛苦上。}

\section{公鸡与野鸡}

有个人在家里喂养了些公鸡。有一天,他在市场上遇见了一只驯化的野鸡,便买下带回家来,与公鸡关养在一起。那些公鸡都追啄他,野鸡原以为他是外来的关系,才遭受欺负。过了几天后,他见到那些公鸡相互斗架,彼此斗得头破血流,还不肯停下来。于是,他自言自语地说:“现在,我再不怨恨被他们所欺了,因为我亲眼看到他们自己也不能互相宽容。”

{\bfseries \color{red}这故事说明,聪明的人若看见邻里与家人还不能互相宽容,那他们得不到这些人的宽容就不足为奇了。}

\section{驴子、公鸡与狮子}

有一天,公鸡和驴子生活在一起。饥饿的狮子来侵害驴子,公鸡一叫,狮子害怕鸡叫,转身逃之夭夭。驴子见狮子连鸡叫都害怕,心想狮子没有什么了不起,便立即跑去追赶狮子。他追到远处,公鸡的叫声听不到了,狮子猛然转过身来,把他吃了。驴子临死时叹道:“我真是不幸啊!我真愚蠢啊!我并不是竞争对手,为什么还要去参加战斗呢?”

{\bfseries \color{red}这故事说明,不要轻敌,要知道自己有多大能耐,不要在强大的对手面前逞能。}

\section{河流与海}

河流入海中,并抱怨海说:“我们的水本是甘甜可口的,但你们却将我们变成咸得不可饮用的水了。”海知道他们是有意来责难他,便说:“请你们别再流到我这里面来,你们也就不会变咸了。”

{\bfseries \color{red}这是说,要看到事物发展的总趋势,不要纠缠在枝节问题上。}

\section{运盐的驴子}

有只驴子驮着盐过河。他的脚一滑,跌倒在河水中,盐在水中都溶化了。他站起来时顿感一身轻松了许多,他很高兴。后来,有一天,他驮着海绵过河,心想再跌倒下去,站起来时定会更轻松。于是,他故意地摔了下去,他没想到海绵是吸水的,因此再也站不起来了,淹死在河里了。

{\bfseries \color{red}这是说,有些人聪明反被聪明误,自己害了自己。}

\section{狮子和他的三个顾问}

狮子把羊叫来,问他能不能闻到自己嘴里发出的臭味。羊说:“能闻到。”狮子咬掉了这个傻瓜蛋的头。接着,他又把狼召来,用同样的问题问狼。狼说:“闻不到。”狮子把这个阿谀奉承的家伙咬得鲜血淋漓。最后,狐狸被召来了,狮子也用同样的问题问他,狐狸看看周围的情形,说:“大王,我患了感冒,闻不到什么味。”

{\bfseries \color{red}这故事是说,模棱两可,暧mei含糊可以让人抓不着把柄。}

\section{黑人}

有人买了一个黑奴,以为他的肤色是由于原来的主人的大意而为,带回家后,用许多肥皂和水想要把他洗干净。可是黑奴肤色丝毫没有一点变化,他自己却因辛苦大病一场。

{\bfseries \color{red}这故事说明,生来就有的东西始终会保留着原始的样子。}

\section{渔夫与金枪鱼}



{\bfseries \color{red}}

\section{狐狸和豹}

狐狸和豹互相为吹嘘自己的美貌而争吵不休。豹总夸耀他身上五颜六色的斑纹,狐狸却说:“我要比你美得多,我的美并不体现在表面,而是灵活的大脑。”

{\bfseries \color{red}这故事说明,智慧的美胜于形体之美。}

\section{猴子与渔夫}

有只猴子坐在一棵大树上,看见渔夫在河里撒网,便仔细看他们的动作。一会儿,渔夫们收起了网,吃饭去了。猴子便连忙从树上爬下来,想要去模仿渔夫捕鱼。但他一拿起网,反把自己套住了,差一点被淹死。猴子自言自语地说:“我真是活该!我没有学会撒网,还抓什么鱼呢?”

{\bfseries \color{red}这故事说明,不要不假思索地模仿不适合自己的行为。}

\section{鹰与屎壳郎}

鹰正在奋力追逐一只兔子。兔子一时无处求助,只好拼命地奔跑。这时,正巧看见一只屎壳郎,兔子便向他求救。屎壳郎一边安慰兔子,一边向鹰恳求不要抓走向他求救的兔子。而鹰却没有把小小的屎壳郎放在眼里,还是在他的眼前把兔子吃掉了。屎壳郎极为遗憾,深感受到侮辱。从此以后,他便不断地盯着鹰巢,只要鹰生了蛋,他就高高地飞上去,把鹰蛋推滚下来,将它摔得粉碎。鹰四处躲避,后来竟飞到宙斯那里去,请求给她一个安全的地方生儿育女。宙斯容许她在自己的膝上来生。屎壳郎知道后,就滚了一个大粪团,高高地飞到宙斯的上面,把它扔到他膝上。宙斯立即起身抖落粪团,无意间把鹰的蛋都砸了下来。据说从那以后,屎壳郎出现的时节,鹰就不孵化小鹰。

{\bfseries \color{red}这故事告诉人们,不要看不起任何人,因为没有人弱小到连自己受了侮辱都不能报复。}

\section{白发男人与他的情人们}

一个头发斑白的男人有两个情人,一个年轻,一个年老。那个年老的女人认为,与比自己年轻的男人交往,怕被别人取笑,只要他来找她,就得不断地拔去他的黑头发。那个年轻的为隐瞒她有一个年老情人,又不断地拔去他的那些白头发。这样,两人轮流地拔,他终于变成了秃头。

{\bfseries \color{red}这故事是说,不相称的事总是有害的。}

\section{母山羊与葡萄树}

葡萄藤刚刚长出嫩绿的新芽,母山羊就非常粗暴地去吃它的嫩叶。葡萄树对母山羊说:“你太残忍了,为什么要伤害我刚刚长出的新芽?难道地上没有青草了?吃了我的叶子,你仍会被宰了拿去祭祀,那时我将把酿成的酒洒在你身上。”

{\bfseries \color{red}这故事是说,那些连嫩新芽都不知爱护的家伙只配承受责骂。}

\section{病鸢}

一只病得快要死的鸢对他妈妈说:“妈妈,请您不要悲伤!还是赶快祈求神明,让他们保佑我的性命吧。”妈妈回答说:“唉!我的孩儿,你想有哪位神会可怜你?几乎每一位神明都被你惹怒了,你总是从他们的祭坛上把人们献给神的祭品偷走。”

{\bfseries \color{red}这故事是说,若要在患难中得到朋友的帮助,就必须在平时缔结友谊。}

\section{小孩和苎麻}

一个小孩不小心被苎麻刺了,他急忙跑回家,告诉妈妈说:“我只轻轻地碰它一下,它就刺得我很痛。”妈妈说:“正因为如此,它才会刺你。下次你如果再碰到苎麻,要勇敢地一把抓住它,它就会在你的手中变得柔软如丝,不再会刺伤你了。”

{\bfseries \color{red}这是说,许多人都是服硬不服软的。}

\section{捕到石头的渔夫}

渔夫们拉网时,觉得很沉重,他们高兴得手舞足蹈,以为这一下子捕到了许多的鱼。哪知把网拉到岸边,网里却满是石头和别的东西,没有一条鱼。他们十分懊丧,没捕到鱼倒也罢,难受的是事实与他们所预想的正好相反。他们中一个年老的渔夫说道:“朋友们,别难过,快乐总与痛苦在一起,她们如同一对姐妹。我们预先快乐过了,现在不得不忍受到一点点痛苦。”

{\bfseries \color{red}这故事是说,人生变化万千,正如有时晴朗的天空会突然发生风暴,不要因挫折而苦恼。}

\section{三个手艺人}

一座大城被敌军围困了,城中的居民们聚在一起,共同商议对抗敌人的办法。一个砌匠挺身而出,主张用砖块作为抵御材料;一个木匠毅然提议用木头来抗敌是最佳的方法;一个皮匠站起来说:“先生们,我不同意你们的意见。我认为作为抵御材料,没有一样东西比皮更好。”

{\bfseries \color{red}这是说,人们都习惯于从自身角度考虑问题,总认为自己所熟悉的东西是最好的。}

\section{驴子和他的影子}

一个旅客雇了一头驴,骑着它到远处去。那天天气很热,赤日炎炎。他停下来休息,躲避在驴子的影子下,求个荫凉,避免暴晒。驴子的影子仅够遮蔽一个人,于是旅客和驴子的主人为了遮荫激烈地争起来,谁都认为自己才有这个权利。驴子的主人坚持说他仅出租驴子本身,不出租驴子的影子。那旅客说他雇的驴子包括驴子本身和影子。他们争论不休,以至互相打了起来。当他们打架时,驴子逃跑了。

{\bfseries \color{red}这是说,人们往往为小事争吵不休,从而失去了最重要的东西。}

\section{饥饿的狗}

几只饥饿的狗,看见河里浸泡着一张兽皮,他们使劲够也够不着。于是,便互相商定,大家一起喝干河水,就可以得到那张兽皮了。结果,还没等到去拿兽皮时,他们的肚皮都被河水涨破了。

{\bfseries \color{red}这是说,不量力而行,辛辛苦苦追求希望渺茫的利益,结果,不但所希望的东西没得到,反而会付出惨重的代价。}

\section{狮子与公牛}

狮子准备杀害一头大公牛,他打算施展狡计来智取。于是,狮子对公牛说:“朋友,你若愿意,我杀一头羊,设宴招待你。”他想趁公牛躺下来吃的时候把他杀死。公牛走到狮子那儿,只见有许多的铜盆和许多大铁叉,根本没看见羊,他一声不吭地走了。狮子责问他,为什么无缘无故一声不响地走了。他回答说:“我这么做是有一定道理的,因为我看出那些所准备的器具,并不是为了吃羊,而是为了吃牛的。”

{\bfseries \color{red}这故事说明,那些聪明的人能从蛛丝马迹中识破坏人的阴谋诡计。}

\section{翠鸟}

居住在海上的翠鸟喜欢僻静的地方,传说为了逃避人类的捕猎,她常在海岸边的岩石上筑巢。有一次在孵卵的季节,一只翠鸟走到一处海岬,看中了临海的一块岩石,便在那里筑起了鸟巢。一天,她出外觅食,忽然海上狂风大作,掀起的波浪一浪高过一浪,汹涌的浪涛冲到岩石上,把鸟巢卷走了,小鸟也无踪无影了。翠鸟回来后,见到这般凄惨的景象,痛苦地说道:“我真是不幸啊,我小心防备了陆上的捕猎,才逃到这里来,谁知海更靠不住。”

{\bfseries \color{red}这就是说,十分小心谨慎的防备敌人,却不知道有时会落在比敌人更厉害的友人手里。}

\section{牧人与海}

有个牧羊人在海边的草地上放牧羊群,看见海很宁静而温顺,便想去航海做生意。于是,他卖掉了羊群,买了些枣子,装船出发了。不料海上刮起了大风暴,船将要沉下去,他只得忍痛把所装的货物全都抛到海里,才乘坐着空船幸免于难。很久之后,有人路过海边,偶遇海面很宁静,大为赞美。牧羊人却对他说:“好朋友,大海又在想要枣子了,所以才显得如此宁静。”

{\bfseries \color{red}这故事说明,人们从患难中能得到学问。}

\section{燕子与蟒蛇}

有只在法院里做窝的燕子飞出去了。蟒蛇趁机爬进燕子窝里,把小燕子都吞吃了。燕子回来发现窝空了,极其悲痛。另一只燕子飞来劝慰她,并说她不是唯一丢失孩子的妈妈。她回答说:“我这样悲痛,并不仅仅是为了丢失孩子,而是因在这受害的地方本是所有受害者都能求得帮助的地方。”

{\bfseries \color{red}这故事说明,灾难来自最意想不到的地方时,最使人悲伤。}

\section{女主人与侍女们}

有个女主人很勤劳,她雇了几名侍女。夜里每当公鸡一打鸣,她就叫她们起来去干活。侍女们每天日夜劳作,累得精疲力尽,她们十分痛恨那只公鸡,决定要弄死它,她们以为是那公鸡不到天亮叫醒女主人,才使她们受苦受难。然而在她们把公鸡弄死之后,反而比以前更为不幸。那女主人不知道鸡叫的时间,总是在黑夜里更早地把她们叫起来去干活。

{\bfseries \color{red}这故事是说,许多人的不幸往往是自己造成的。}

\section{守财奴}

有个守财奴变卖了他所有的家产,换回了金块,并秘密地埋在一个地方。他每天走去看看他的宝藏。有个在附近放羊的牧人留心观察,知道了真情,趁他走后,挖出金块拿走了。守财奴再来时,发现洞中的金块没有了,便捶胸痛哭。有个人见他如此悲痛,问明原因后,说道:“喂,朋友,别再难过了,那块金子虽是你买来的,但并不是你真正拥有的。去拿一块石头来,代替金块放在洞里,只要你心里想着那是块金子,你就会很高兴。这样与你拥有真正的金块效果没什么不同。依我之见,你拥有那金块时,也从没用过。”

{\bfseries \color{red}这故事说明,一切财物如不使用等于没有。}

\section{鬣狗与狐狸}

传说鬣狗每年要变换他们的性别,有时是雄的,有时是雌的。有条鬣狗看见狐狸,便指责他,说自己想要和他交朋友,狐狸却不理睬。狐狸回答道:“你不要指责我,而应该指责你自己,因为我不知道把你当女朋友好呢,还是当男朋友好。”

{\bfseries \color{red}这是说那些做事态度暧mei的人。}

\section{迈安特洛斯河边的狐狸}

有一天,众多狐狸聚集在迈安特洛斯河边,想要喝河里的水。但因河水水流很急,他们彼此只是说说而已,不敢跳下河去。其中有一只狐狸,嘲笑同伴胆小,为显示自己比他们勇敢,他壮着胆子跳入河中。湍急的河水一下就把他冲到了河心,站在河边的狐狸对他说:“请不要离开我们,快回来,告诉我们从哪里可以安全下去喝水吧。”被水冲走的狐狸却回答说:“我想把一封寄往米利都的信送到那里去,回来后我再告诉你们吧。”

{\bfseries \color{red}这就是说,那些喜欢卖弄自己、自我吹嘘的人常常给自己招来不幸。}

\section{吹牛的运动员}

有个运动员因平常参加比赛时缺乏勇气,被人们指责,只好出外去旅行。过了些日子,他回来后,大肆吹嘘说,他在别的很多城市多次参加竞赛,勇气超人,在罗德岛曾跳得很远,连奥林匹克的冠军都不能与他抗衡。他还说那些当时在场观看的人们若能到这里来,就可以给他作证。这时,旁边的一个人对他说:“喂,朋友,如果这一切是真的,根本不需要什么证明人。你把这里当作是罗德岛,你跳吧!”

{\bfseries \color{red}这故事说明,用事实容易就近证明的事,说得再多都是多余的。}

\section{狼与马}

狼路过一处田地,看到地里有许多大麦。虽然黄澄澄的招人喜爱,但狼不吃大麦,只好走开了。刚走不远,就遇见一匹马,他把马领到田里,告诉马这些大麦他自己舍不得吃,特意给马留着,因为喜欢听马吃草时牙齿发出的美妙声音。马回答说:“喂,朋友,你若能以大麦为食料,你就未必喜欢听我吃草的声音,而不顾你的肚子了。”

{\bfseries \color{red}这故事说明,那些本性恶劣的人,尽管向人们报告最好的消息,也是别有用心的,人们不会相信。}

\section{老狮子}

一头年老体衰的狮子病得有气无力,奄奄一息地躺在地上。一头野猪冲到他身旁,狠狠地咬他,报复狮子以前对他的伤害。一会儿,一头野牛也用角来顶他,把狮子视为可恨的仇敌。当驴子看到可以对这庞大的野兽为所欲为时,也用他的蹄子用力去踢狮子的头部。这头快要断气的狮子说:“我已勉强忍受了勇者的施暴,但还得含羞忍受你这个小丑的侮辱,真是死不瞑目。”

{\bfseries \color{red}这是说,无论过去多么辉煌,都难以避免辉煌失去后别人的不敬与报复。}

\section{肚胀的狐狸}

饥饿的狐狸四处寻食,他看见树上的洞穴里有牧人遗留的面包和肉,就立即钻进去吃。肚子吃得胀鼓鼓的,他费了九牛二虎之力,却怎么也钻不出来,便在树洞里唉声叹气。另一只狐狸恰巧经过那里,听到他的呻吟,便过去问他原因。当他听明白缘由后,便对他说道:“你就老老实实呆在里边吧,等到恢复你钻进去的样子时,就很容易出来了。”

{\bfseries \color{red}这故事说明,时间能解决许多困难问题。}

\section{赫耳墨斯与雕刻家}

赫耳墨斯想要知道人们对他有多尊重,便化作一个凡人,来到一个雕刻家的店里。他看见宙斯的像,便问要多少钱。雕刻家回答说一块银元。他笑着又问赫拉的像要多少钱。雕刻家说那要贵些。当他看见了自己的像时,心想自己身为神的使者,又是招财进福的神,应该标出高价吧。赫耳墨斯便指着自己的像,问需要多少钱,雕刻家答道:“假如你买了那两个,我便把这个做零头,白送给你吧。”

{\bfseries \color{red}这故事说明,那些爱慕虚荣的人,往往被别人看不起。}

\section{天文学家}

有位天文学家习惯每天晚上出去观察星象。有一天,他来到郊外聚精会神地观察天空,一不小心掉进一口井里。他大声叫喊起来。附近的人听到呼叫声后,走过来弄清楚了情况,便对他说:“喂,朋友,你用心观察天上的东西,却没有看地上的事情。”

{\bfseries \color{red}这故事是说,人首先要做好地上的最普通的事,才谈得上天上的高深的事。}

\section{磨坊主和儿子与驴子}

磨坊主和他的儿子一起赶着他们的驴子,到邻近的市场上去卖。他们没走多远,遇见了一些妇女聚集在井边,谈笑风生。其中有一个说:“瞧,你们看见过这种人吗,放着驴子不骑,却要走路。”老人听到此话,立刻叫儿子骑上驴去。又走了一会,他们遇到了一些正在争吵的老头,其中一个说:“看看,这正证明了我刚说的那些话。现在这种社会时尚,根本谈不上什么敬老尊贤。你们看看那懒惰的孩子骑在驴上,而他年迈的父亲却在下面行走。下来,你这小东西!还不让你年老的父亲歇歇他疲乏的腿。”老人便叫儿子下来,自己骑了上去。他们没走多远,又遇到一群妇女和孩子。有几个人立刻大喊道:“你这无用的老头,你怎么可以骑在驴子上,而让那可怜的孩子跑得一点力气都没啦?”老实的磨坊主,立刻又叫他儿子来坐在他后面。快到市场时,一个市民看见了他们便问:“朋友,请问,这驴子是你们自己的吗?”老人说:“是的。”那人说:“人们还真想不到,依你们一起骑驴的情形看来,你们两个人抬驴子,也许比骑驴子好得多。”老人说:“不妨照你的意见试一下。”于是,他和儿子一起跳下驴子,将驴子的腿捆在一起,用一根木棍将驴子抬上肩向前走。经过市场口的桥时,很多人围过来看这种有趣的事,大家都取笑他们父子俩。吵闹声和这种奇怪的摆弄使驴子很不高兴,它用力挣断了绳索和棍子,掉到河里去了。这时,老人又气愤又羞愧,赶忙从小路逃回家去。

{\bfseries \color{red}这是说,任何事物都不可能使人人满意,想使人人满意,反而会谁也不满意。}

\section{争论神的人}

两个人,为忒修斯的能耐大还是赫拉克勒斯的能耐大,争吵得面红耳赤。但是两位神却对他们十分生气,没收了他们中一人的土地。

{\bfseries \color{red}这就是说,对主人评头论足会带来坏处。}

\section{鹿与洞里的狮子}

有只鹿拼命地逃避猎人的追捕,跑到一个住着狮子的洞里。他刚一进去就被狮子抓获。鹿临死之前说:“我真是倒霉,逃避了猎人,却将自己送给了最凶猛的野兽。”

{\bfseries \color{red}这故事是说,有些人为了躲避较小的危险,反而陷入到更大的危险里去。}

\section{海豚、鲸与白杨鱼}

海豚与鲸互相交战。他们争斗了很久,并越打越猛烈。这时,有一条白杨鱼游过来,劝告他们停止争斗。海豚却说:“我们宁可争斗到同归于尽,也比让你来调解要好受得多。”

{\bfseries \color{red}这是说,有些人本来无足轻重,遇着乱世,自以为是地称起英雄来。}

\section{泉边的鹿与狮子}

一只鹿非常口渴,连忙跑到泉水边去。他喝着甘甜的泉水,望着水里自己的影子,见自己修长而美丽的双角,得意洋洋,当见到自己细小的腿,又郁郁不乐。正当他看得入神时,有头狮子疾奔而来。他转身拼命地逃跑,一下就把狮子远远地甩在身后,因为鹿的力量是在腿上,狮子的力量是在心脏上。在空旷的平原上,鹿总能跑在前头,保住性命。但当他进入到树林中时,美丽的双角被树枝挂住了,再也无法奔跑了,结果被跟踪而来的狮子捉住了。鹿临死之前对自己说:“我真不幸呢!被我所不喜欢的救了命,却被我所最信赖和宠爱的东西断送了生命。”

{\bfseries \color{red}这故事是说,美丽的东西不一定有用,甚至还会坏事,不美的东西却在关键时刻有实用。}

\section{狐狸和鳄鱼}



{\bfseries \color{red}}

\section{狐狸和狗}

狐狸偷偷地溜进羊群里,抱起一只羔羊,假惺惺抚mo着他。狗问狐狸在干什么,他说:“我在逗他,与他玩耍呢。”狗又说:“现在你若不放下这小羊,我将叫你尝尝狗的抚mo。”

{\bfseries \color{red}这故事适用于恶汉和笨贼。}

\section{胆小的士兵与乌鸦}

有个胆小的士兵出去打仗,乌鸦大叫一声,他立刻放下武器,一动也不敢动。过了一会儿,他拿起武器再往前走,乌鸦又大叫了起来。他停下来说:“你们尽力气去大叫吧,只是别来吃我的肉呀!”

{\bfseries \color{red}这故事适用于非常胆小的人。}

\section{丈夫与怪癖的妻子}

某人的妻子脾气非常怪癖,她与家里的所有人都难以相处。丈夫想知道她与她娘家的人是否也是如此,便找了一个很好的借口把她送回了娘家。刚过几天,她就回来了,丈夫问妻子娘家的人待她怎么样。她回答说:“那些放牛和牧羊的人都不给我好脸色看。”丈夫对她说道:“啊,亲爱的,若那些早出晚归的牧人都不能与你很好地相处的话,那么整天和你在一起生活的人又会对你怎么样呢。”

{\bfseries \color{red}这故事是说,事情常常可以由小见大,由表及里。}

\section{农夫与杀死他儿子的蛇}

一条毒蛇趁人不备爬进农夫家,咬死了农夫的儿子。农夫非常悲痛,抓起一把斧头,气冲冲地跑到蛇洞外站着,只要蛇一出洞就砍死他。不久,蛇刚从洞里出来,农夫立即一斧头砍去,可惜没砍到蛇,却把洞旁的一块石头劈成了两半。农夫担忧后患,便恳求蛇与他和解。蛇说:“我一见那劈开的石头,就不可能对你产生好感;同样,你一见到儿子的坟墓也不会原谅我。”

{\bfseries \color{red}这故事说明,深仇大恨难以和解。}

\section{狐狸和为王的猴子}

有一次,猴子在野兽的集会上跳舞,赢得了大家的好感,被选立为王。狐狸十分嫉妒,当他发现一个捕兽夹子里放着肉,便把猴子领到那里去,说他发现一个宝物,自己没敢动用,留给王室作贡品,并劝他亲自去取。猴子轻率地跑了上去,结果被夹子夹住了。他斥责狐狸陷害他,狐狸却说:“猴子,凭你这点小小的本事,你这笨蛋还想做兽中之王吗?”

{\bfseries \color{red}这故事说明,凡事不要轻率。不然,就会给自己带来不幸,并被世人嘲笑。}

\section{狐狸和狮子}

从前,狐狸和狮子住在一起。狐狸总是充当狮子的奴仆,常常去森林把野兽赶出来,然后再由狮子去捕捉。他们俩总是按功劳大小来分配猎物。然而,狐狸嫉妒狮子分得太多了,不愿意再帮狮子追赶野兽,自己独自去林中捕捉猎物。当他正准备捕捉一只羊时,却先被猎人抓住了。

{\bfseries \color{red}这故事说明,平平安安地做百姓比胆颤心惊做国王好得多。}

\section{狐狸和关在笼里的狮子}

有头狮子被关在笼子里,狐狸看见了,便毫不畏惧地走过来大声地谩骂狮子。狮子对他说:“骂我的不是你,而是我所遇的不幸。”

{\bfseries \color{red}这故事说明,身遭不幸的强者往往会受到地位低下的小人的蔑视。}

\section{狐狸和猴子争论家世}

狐狸与猴子同行,一路互相争吵他们谁的家世高贵。他们各自夸耀一番后,来到了一处墓地。猴子转过头去,放声大哭。狐狸不知其原因,忙问他为什么哭,猴子指着那些墓碑说:“我看到这些为我祖先所解放和奴役过的奴隶墓碑,怎能不伤心呢?”狐狸说:“你就使劲的吹牛骗人吧,他们之中没有谁能站起来反驳你。”

{\bfseries \color{red}这就是说,在没有人反驳时,说谎话的人尤为自吹自擂。}

\section{农夫和毛驴}

有个年老的农夫一直住在乡下,从来没进过城,他要求家人带他进城去看看。家人让他坐在两头毛驴拉的车上,并对他说:“你只要赶着毛驴,它们就会把你送到城里。”走到半路,风暴突起,天昏地暗,毛驴迷了路,走到了悬崖边上。老人眼看十分危险,便说:“宙斯啊!我冒犯过你吗?你要罚我摔死,并且不让我死在光荣的马或高贵的骡子手下,却要我死在小毛驴手下!”

{\bfseries \color{red}这故事是说,死就要死得光荣、壮烈。}

\section{还不了愿的人}

有个穷人生了病,病情越来越严重,医生们都说毫无希望了。但他仍对众神祷告,如果他的病能好,一定献上一百个牛头作祭品,还要献匾给庙里。站在他旁边的妻子问:“从哪来的这笔钱?”那人答道:“你以为神是为钱向我要这些东西,才医治好的病吗?”

{\bfseries \color{red}这故事说明,信口许愿的人往往还不了愿。}

\section{杀人凶手}

一个杀人犯,被受害者的亲人们穷追猛赶,逃到尼罗河边时,迎头遇见一条狼,他惊恐地爬到河边的一棵树上,躲在上面。但他又看见树上有一条大蛇朝他爬来,他吓得跳到了河里。在河里有一条鳄鱼,正在等着他,就把他吃了。

{\bfseries \color{red}这故事说明,对于有罪的恶人,无论在地上,在空中或在水里,都不会安全。}

\section{农夫与命运女神}

有个农夫耕地时,发现了一块金子,认为一定是土地女神所赐。于是,他每日给土地的女神祭奉。命运女神来到他面前,说:“喂,朋友,那块金子是我送给你的发财礼物,你为什么把它看成是土地女神的恩惠呢?若时运不同,这块金子也许会落入别人的手里,那时候你一定又要怨我命运女神了。”

{\bfseries \color{red}这故事说明,人应当认清恩人,报答他的恩惠。}

\section{狡猾的人}

一个狡猾的人与另一个人打赌,约定要向他证明德尔斐的神示是假的。到了约定日期,他手里拿了一只小麻雀,并将它藏在外衣里。他走到庙里,站在神前面,问神他手中拿着的东西是活的还是死。他想,若神说是死的,他便把活麻雀拿出来;若神说是活的,他就捏死麻雀,再拿出来。神识破了他卑鄙的诡计,说道:“小伙子,收起你那一套吧!你手里的东西,是死是活,还不在乎你!”

{\bfseries \color{red}这故事说明,神不能亵du。}

\section{农夫和狐狸}

有个心肠很坏的农夫十分嫉妒邻居农田里的庄稼长得好,一心想毁掉这些庄稼。于是,有一天,他趁捉狐狸的机会,偷偷地把烧燃的木柴放在邻居的地里。正好路过此地的狐狸拿起那块木柴,按照神明的指示,扔到这个农夫的地里,把他的庄稼烧得精光。

{\bfseries \color{red}这是说,害人必害己,神决不会放走任何一个坏人。}

\section{农夫和树}

有个农夫的田里有一棵树,这棵树并不粗壮,只能作那些麻雀和吵闹的蝉的栖息地。农夫觉得这树没有什么大用处,想把它砍掉,便拿起斧头,朝树砍了一下。那些蝉和麻雀请求农夫不要砍倒他们的家,允许树生长在田地里,他们将在树上为他歌唱,使他高兴。农夫没理睬他们,接着又砍了第二斧和第三斧,直至树上砍出了一个洞。这时,他发现树洞里有蜜蜂窝和蜜,他尝了尝蜜后,连忙抛下斧头,不但不再砍伐,而且对这树加以小心保护。

{\bfseries \color{red}这故事是说,重利轻义是某些人的本性。}

\section{遇难的人}

有位雅典的富商与别人一起去航海。一天,海面上风暴骤起,狂风巨浪把船打翻了。这时,别人都在使劲游泳逃命,唯有雅典人不停地祷告雅典娜女神,许愿如果能得救,一定献上很多祭品。有一个共同遇难的人游到他身旁,对他说道:“雅典娜保佑你,你也得动动你的手吧!”

{\bfseries \color{red}这故事是说,在请求神帮助时,自己也得积极想办法去做点事。}

\section{发现金狮子的人}

从前,有一个既胆小又贪心的富人发现了一只金狮子。他自言自语地说:“我不知道这事到底该怎么办,我心里很乱,无法打定主意。我一方面贪心爱财,一方面又胆小怕事。这样的运气是怎么回事?金狮子是哪位神明造出来的?这件事使我心中矛盾重重。爱金子吧,又害怕金子做成的狮子;心中的yu望催我去拿,性格却劝我退后。唉!好运来了,可我又不敢接受。这宝物并没有使我快乐。神给予我的恩惠,可望而不可及!这是怎么回事?我要怎么办,要采用什么方法呢?我得回去把家人带来,凭借他们许多人的力量来捉住它,我自己则站和远远地观望!”

{\bfseries \color{red}这故事适用于那些既贪图财富又害怕灾祸的人。}

\section{农夫与狼}

农夫替牛解下犁套,牵着它去喝水。这时,有只穷凶极恶的饿狼正出来觅食,看见那犁,开始仅仅只舔舔那牛的犁套,觉得有牛肉味,便不知不觉地将脖子慢慢地伸了进去,结果再无法拔出来,只好拉着犁在田里耕起田来。那农夫回来后,看见了它,便说:“啊,可恶的东西!但愿你从今弃恶从善,回来种田吧。”

{\bfseries \color{red}这故事是说,尽管有些恶人做了一点善事,但这并非他的本意,而是出于无奈。}

\section{骗子}

有个人卧床不起,病情十分严重,他绝望地祷告众神,说若能使他病愈,他一定奉献一百头牛。众神想试验他一下,便用灵丹妙药,使他康复了。他病好下床后,没用真正的牛来酬谢众神,而用面团做成了一百头牛,放在祭坛上烧了,并念念有词地祷告说:“诸位神明,请接受我所许下的承诺吧。”这时,众神们认为他用骗术亵du了神灵,便在晚上托梦告诉他,要他到海边去,说在那里可以找到一千块雅典钱。他醒来后,高兴极了,直往海边跑去。结果在那里遇到海盗,被他们抓去卖了,卖了一千块钱。

{\bfseries \color{red}这故事适用于说谎话的人。}

\section{青蛙邻居}

两只青蛙相邻而居。一只住在远离大路的深水池塘里,另一只却住在大路上小水坑中。住在池塘里的青蛙友好地劝住水坑的邻居搬到他那里去,说那将会生活得更好、更安全,可是邻居却说舍不得离开习惯了的地方,不想搬来搬去。结果,被过路的车子压死了。

{\bfseries \color{red}这故事是说,习惯于环境不图变迁,不但过不上好日子,还会为旧环境所困扰,有生命之忧。}

\section{人与宙斯}

传说神最先创造的是动物,并赏赐给他们的有的是力量,有的是速度,有的是翅膀。而在创造人时,人却裸露着身体一无所有。人对神说:“你就不给我一点什么赏赐吗?”宙斯说:“你难道没见到赐于你的礼物?那才是最大的礼物。因为你有思想,思想不论赋于神或人都是有力的,而且将比所有力量更有力,比最快的速度更快呢。”人这才感觉到神赐于自己礼物是最珍贵的,并向神们表示敬意,很感激地走去。

{\bfseries \color{red}这是说,思想是神给人最大的赏赐,最特别的礼物,思想也是人有别于动物的最显著的标志。}

\section{人与狐狸}

有人与狐狸为敌,因为狐狸经常危害他。有一天,他抓到了一只狐狸,想要狠狠地报复一下。他把油浸在麻皮上,绑在狐狸尾巴上,然后点上火。神灵却将狐狸引进那人的田地里。那时正当收获的季节,这人一边赶狐狸一边痛哭,因为田里将什么都收获不到了。

{\bfseries \color{red}这故事是说,人们在极度生气时,往往会毫无理智地处理事情,从而招来更大的灾祸。}

\section{三只公牛与狮子}

三只公牛住在一起生活。有只狮子一心想要吃掉他们,可他们团结一致,狮子一直没能得逞。狮子便进行挑拨离间,使得他们相互冲突,随后狮子趁三头牛单独居住时,轻而易举地将他们一个个地吃掉了。

{\bfseries \color{red}这故事说明,人们不要相信敌人的花言巧语,要相信你的朋友,保持团结。}

\section{女人与酗酒的丈夫}

从前,有一个女人,她丈夫喜欢好酒贪杯。她想帮丈夫戒掉这不良恶习,便想出了一个办法。一次,她丈夫大醉如泥,像死人似地不省人事,她就把他背出去,放到墓穴里,然后回家了。估计丈夫快清醒时,她便来到墓地,敲墓穴的门。墓里的人问:“谁在敲门?”她答道:“我是给死人送吃的来的。”他说:“喂,好朋友,请你不要送吃的,还是先送点喝的来吧。没有喝的,真让我难受。”女人捶胸顿足,伤心地说:“啊,我多么的不幸呀!我费尽心机,一点效果都没有。老公呀,你不但没有改好,反而变本加利,你的嗜好已成了一种恶劣的习惯了。”

{\bfseries \color{red}这故事说明,人不能沉湎于不良的嗜好中,即使你并非有意,可是习惯成自然,要戒除就不容易了。}

\section{女巫}

有个女巫声称自己能念咒语,使众神息怒。她经常四处招摇撞骗,因此得到了不少酬金。但后来有人控告她破坏神道,把她抓到法庭,判处了死刑。有人见到女巫被押赴刑场时,对她说:“喂,女人,你不是自称能平息神灵的忿怒吗,现在怎么连凡人的忿怒也不能平息了呢?”

{\bfseries \color{red}这故事是说,有些口口声声称自己能办大事,可一点小事也办不到。}

\section{胆小的猎人与樵夫}

有个猎人搜寻狮子的足迹。他问一个樵夫,有没有发现狮子的足迹。樵夫说:“只看到狮子本身。”猎人吓得脸色惨白,全身哆嗦地说:“我仅搜寻它足迹,并不要找狮子本身。”

{\bfseries \color{red}这故事是说,有些人的勇敢,仅停留在口头上,而不是表现在行动中。}

\section{金丝雀与蝙蝠}

挂在窗口笼里的金丝雀,在夜里歌唱。蝙蝠听到后,飞过来问她为什么白天默默无声,在夜间却放声歌唱。金丝雀回答说,她这样是有道理的,因为他是在白天唱歌时被捉住的,从此他变得谨慎了。蝙蝠说:“你现在才懂得谨慎已没用了,你若在被捉住之前就懂得,那该多好呀!”

{\bfseries \color{red}这故事说明,不幸的事发生之后,后悔是徒然的。}

\section{黄鼠狼与爱神}

黄鼠狼爱上一个漂亮的青年,请求爱神将自己变为女人。爱神同情她的热情,将她变成了一个美丽多姿的少女。于是,那青年人一见就爱上了她,带着她回自己家里去了。当他们喜气洋洋地走进洞房时,爱神想要知道,黄鼠狼改变了外形后,习性会不会改变,因此她把一只老鼠放进了房子里。那女人忘记了自己的身份,立刻跳下床,去追老鼠,想要吃掉它。爱神见此,十分气愤,又将黄鼠狼变回原来的模样。

{\bfseries \color{red}这故事是说,本性恶劣的人,即使外形变了,本性仍难改。}

\section{黄鼠狼与锉刀}

黄鼠狼钻进一家铁匠的作坊,看见一把锋利的锉刀放在那里,他就去舔它。结果舌头被刮破,鲜血直流。但他还以为舔下了一些铁,非常兴奋,终于把舌头全舔掉了。

{\bfseries \color{red}这故事是说,那些好斗的人最终害了自己。}

\section{演说家}

有一天,演说家得马得斯在雅典演说,没有一个人认真地听,他便请大家允许他讲一则伊索寓言。人们一致同意,他开始说:“得墨忒耳和燕子、鳗一起同行。他们来到了一条河边,燕子飞走了,鳗潜入水中。”讲到这里,他便再不讲了。人们问他:“那么得墨忒耳怎么了?”他回答说:“她正生你们的气呢,因为你们对国家大事毫无兴趣,而只喜欢听伊索寓言。”

{\bfseries \color{red}这故事是说,不务正业,只图安乐的人是十分愚蠢的。}

\section{第欧根尼与秃子}

古希腊哲学家第欧根尼遭一个秃子谩骂后,说道:“我决不会回击。我倒欣赏你的头发,他早已离开了你那可恶的头颅而去了。”

{\bfseries \color{red}这是说,幽默与讽刺是最好的回击。}

\section{旅行的第欧根尼}

古希腊哲学家第欧根尼外出旅行,走到一条洪水泛滥的河边,站在岸上无法过河。有个经常背人过河的人,见他在那里为难,便走过来把他搁在肩上,很友好地背他渡过了河。他很感激这个人,站在河岸上抱怨自己贫穷,无法报答行善的人。当他正思索这事的时候,看见那人又在背别的人过河。第欧根尼走上前说:“对于刚才的事我不必再感谢你了。我现在知道,你不加选择的这样做,只是一种怪癖。”

{\bfseries \color{red}这故事说明,有些人对于任何人都不加审慎地行善,他们得到的不是赞誉,而是愚蠢的骂名。}

\section{农夫与鹰}

农夫发现一只鹰被捕兽夹夹住了,他见鹰十分美丽,惊讶不已,于是便把鹰放了,鹰表示永不忘他的恩德。有一天,鹰看见农夫坐在将要倒塌的墙下,就立刻朝下飞去,用脚爪抓起他头上的头巾。农夫站起来去追,鹰立即把头巾丢还给他。农夫拾起头巾后,回过头来一看,却发现在他刚坐过的地方,墙已倒塌了。他对鹰的报恩十分感动。

{\bfseries \color{red}这故事是说,人们一定要知恩图报,做了好事也一定会得到好报。}

\section{橡树与宙斯}

橡树指责宙斯说:“我们生存着毫无意义,所有的植物中我们被砍伐得最多。”宙斯说:“招来不幸原因全在你们自己,假如你们不能做斧柄,对木匠和农夫毫无用处,那么斧头也不会来砍你们了。”

{\bfseries \color{red}这故事是说,有些人把自己所引起的不幸,毫无道理地归咎于神。}

\section{樵夫与橡树}

有个樵夫在砍橡树。他先用橡树枝做成楔子,再用楔子很容易地就劈开了树身。橡树说:“我不怨恨那斧子,而更恨那从我身上生出来的楔子。”

{\bfseries \color{red}这故事是说,一个人被家人所害,比被别人所害更痛苦,更伤心。}

\section{赫耳墨斯与地神}

宙斯创造了男人和女人,吩咐赫耳墨斯领他们到地里去,指点他们开荒种地,生产粮食。赫耳墨斯奉命而行。刚一开始地神就要阻挠。赫耳墨斯就强迫她,说这是宙斯的命令。地神说道:“那么就让他们随心所欲地去开垦吧,反正他们要哭泣着来偿还的。”

{\bfseries \color{red}这故事适用于那些轻易借债,却辛苦偿还的人。}

\section{赫耳墨斯与忒瑞西阿斯}

赫耳墨斯想要试验一下忒瑞西阿斯的预言灵不灵,便从牧场偷走他两头牛,再化作凡人样子,进城去找他,来到他家作客。忒瑞西阿斯得知牛被偷,便带赫耳墨斯来到郊外,观察有关偷盗的征兆,并对赫耳墨斯说,如看见了什么鸟就赶紧告诉他。当赫耳墨斯看见一只鹰从左边飞到右边去,便马上报告他。忒瑞西阿斯却说,这毫不相干。随后赫耳墨斯又看见一只乌鸦飞到一棵树上,时而往上看,时而低头向下看,又跑去报告他。忒瑞西阿斯于是说:“乌鸦向天地神发誓说,只要你愿意,我的牛就可以找回来。”

{\bfseries \color{red}这故事可讲给偷窃的人听。}

\section{赫耳墨斯与手艺人}

宙斯吩咐赫耳墨斯去给手艺人身上全都撒上说谎话的药。药研制好后,他给每个手艺人身上平均地撒上。最后,仅剩下了皮匠,但仍留下很多药,他便拿起剩下的药全部撒在了皮匠身上。从此以后,手艺人都说谎,特别是皮匠更为厉害。

{\bfseries \color{red}这故事适用说谎的人。}

\section{赫耳墨斯的车子与亚剌伯人}

有一天,赫耳墨斯赶着一辆满载说谎、欺骗、讹诈的车子,到世界各地去旅行,每到一处便将车上所载的东西分给众人。据说,当他走到亚剌伯人的国家时,那辆车突然坏了。亚剌伯人以为车上载着贵重物品,就抢光了车里的所有东西,赫耳墨斯便不能再到别的地方去分发这些东西了。

{\bfseries \color{red}这故事是说,亚剌伯人是最会说谎的人,他们的嘴里没有一句真话。}

\section{太监与祭司}



{\bfseries \color{red}}

\section{宙斯与狐狸}

宙斯赏识狐狸的聪明和狡诈,赐他做兽类之王。宙斯想知道狐狸随着身份的变化,他贪婪的本性会不会有所收敛。当狐狸坐轿子走过来时,宙斯扔下一只屎壳郎。屎壳郎围绕着轿子不停地飞,狐狸再也忍耐不住,立即跳下轿子,想捉住他。宙斯十分气愤,便将狐狸贬回到原来的地位。

{\bfseries \color{red}这故事说明,即使穿上了最华丽的服装,坏人也不会改变他的本性。}

\section{宙斯与人}

宙斯创造了人,并吩咐赫耳墨斯把智慧灌入到他们的体内。他给每人准备好了相等的智慧,然后再分别给他们灌上。身材矮小的人,就灌满了,成为了聪明人;可那些身材高大的,智慧仅灌到膝盖,根本不够,便比别人愚蠢些。

{\bfseries \color{red}这故事适用于那些身体魁梧而缺乏头脑的人。}

\section{宙斯与阿波罗}

宙斯与阿波罗争论谁的射术高明。阿波罗用力张弓,射出一支箭,而宙斯却只迈一大步就跨到了箭所落在的远处。

{\bfseries \color{red}这故事说明,强中更有强中手。}

\section{宙斯与蛇}

宙斯结婚时,所有的动物都尽自己的所能送来礼物。一条蛇嘴里衔着一朵玫瑰花,爬来送礼。宙斯见了他说:“别的一切动物的礼物我都接受,可是从你的嘴里来的东西我是万不能收的。”

{\bfseries \color{red}这故事说明,坏人的恩惠是令人生畏的。}

\section{宙斯与善}

宙斯把所有的善都密封在一只缸里,存放在一个人那里。那人出于好奇,想要知道缸里是什么东西,便揭开了盖子。于是,所有的善又都飞到众神那里去了。

{\bfseries \color{red}这是说,人间所有的善只是对善的希望罢了。}

\section{宙斯与乌龟}

宙斯结婚时,举行盛大宴会,招待所有的动物。只有乌龟没有出席,宙斯不知道他什么原因没来。第二天,他便问乌龟为什么不来赴宴。乌龟回答说:“还是家里舒服,我爱自己的家。”宙斯气愤至极,就罚乌龟永远驮着他的家行走。

{\bfseries \color{red}这故事说明,再豪华的宴会也没有自己的家舒服。}

\section{宙斯做判官}

宙斯命令赫耳墨斯把人间的罪恶都记录在贝壳上,放到他旁边的箱子里,让他来裁决对每人的惩罚。然而箱子里的贝壳都相互混杂在一起,所以宙斯拿到的贝壳有先有后,但总归是要拿到的。

{\bfseries \color{red}这故事是说,人们不必因那些恶事和坏人没有及时的受到惩罚而不快。古人云,善有善报,恶有恶报,不是不报,只是时间未到。}

\section{赫拉克勒斯与雅典娜}

赫拉克勒斯经过一条狭窄的路时,见到地上有一个很像苹果的东西。他用脚想去踩碎它,突然觉得那东西变大了两倍,于是他更加用力去踩,到后来用大木棒去打。结果那东西越胀越大,把路都堵塞了。他扔下木棒,不知所措地站在那里。这时,雅典娜来到他面前,说:“兄弟,住手吧,不要与人争斗和对抗。如果你不去理那东西,它就会平平安安地停放在那里,仅只有苹果一样大。若与它争斗和对抗,它就会膨胀得巨大。”

{\bfseries \color{red}这故事是说,生活中需要和平共处,争斗和对抗往往会带来更大的危害。}

\section{赫拉克勒斯与财神}

赫拉克勒斯被承认为神以后,宙斯为他设宴庆贺。宴会上,赫拉克勒斯热情友好地向众神们一一问好。最后,当财神进来时,他却转过身去,背对着财神,低着头看地板。宙斯对此觉得非常奇怪,便问他为什么与其他神都高高兴兴地打招呼,唯独对财神却另眼相看。他回答说:“我对他另眼相看,是因为在人间,总见到他与坏人在一起。”

{\bfseries \color{red}这故事是说,许多有钱人其财富往往是不义之财。}

\section{英雄}

有人在家里供奉着英雄,经常不断地把昂贵的物品祭献给英雄,所用的祭品花去了他许多钱财。英雄在夜里对他说:“喂,朋友,不要再浪费你的钱财了。你若都花完了,就会变为穷人,那时你就会怨恨我。”

{\bfseries \color{red}这故事是说,许多人由于自己的无知遇到了不幸,却把原因归咎于神。}

\section{宙斯和猴子}

宙斯通知林中所有野兽,许诺给评选出来的拥有最漂亮孩子的野兽发奖。猴子与其他野兽一起来到宙斯那里,她以慈祥的母爱,带着一只扁鼻无毛、相貌丑陋的小猴子,前来参加评奖。当她把小猴子带给大家看时,引得一阵哄堂大笑。但她坚定地说:“我不知道宙斯会不会将奖品给我儿子。但至少有一点我十分清楚,在他母亲眼里,这小猴子是最可爱的、最漂亮的、最活泼的。”

{\bfseries \color{red}这是说,不管孩子漂亮还是丑陋,优秀还是平庸,在母亲眼中总是最好的。}

\section{哲学家、蚂蚁和赫耳墨斯}

一个哲学家在海边看见一艘船遇难,船上的水手和乘客全部淹死了。他便抱怨上帝不公,为了一个罪恶的人偶尔乘这艘船,竟让全船无辜的人都死去。正当他深深的沉思时,他觉得自己被一大群蚂蚁围住了。原来哲学家站在蚂蚁窝旁了。有一只蚂蚁爬到他脚上,咬了他一口。他立刻用脚将他们全踩死了。这时,赫耳墨斯出来了,他用棍子敲打着哲学家说:“你自己也和上帝一样,如此对待众多可怜的蚂蚁。你又怎么能做判断天道的人呢?”

{\bfseries \color{red}这故事是说,人不可苛求别人,因为自己也难免犯别人同样的错误。}

\section{赫耳墨斯神像与木匠}

一个很贫穷的木匠供奉着一个木雕的财神赫耳墨斯神像,请求发财。尽管他不断的祈求,日子却愈来愈穷。最后,一气之下他将神像从祭台上拿下来朝墙头摔去。神像的头被摔断下来,一道金泉一涌而出,木匠连忙拾起神像,并说:“我想你简直是无理得使我无所适从。尊敬你,供奉你,却得不到好处;对你不好,倒使我发了横财。”

{\bfseries \color{red}这故事是说,有些人敬酒不吃吃罚酒,对这样的人只有采取强硬的对策。}

\section{孔雀和天后赫拉}

孔雀向赫拉诉说夜莺以悠扬、动听的歌声,深深地打动了人们的心,使大家十分喜爱她。而她一开口唱歌,便遭到听众们的嘲笑。天后赫拉安慰她说:“但你的外表和身材是出类拔萃的。绿宝石的光辉闪耀在脖子上,开屏时,羽毛更是华丽富贵,光彩照人。”孔雀说:“既然在歌唱上我远远不及他人,这种无言的美丽,对我又有什么用呢?”赫拉回答说:“各人有各人的命运,这是由命运之神所注定的。他注定了你的美丽,老鹰的力量,夜莺的歌唱,乌鸦的凶征。所有鸟类都满意神所赋与他们的东西。”

{\bfseries \color{red}这故事是说,人要接受自己的优点,也要接受自己的缺点,任何事和任何人都不可能十全十美。}

\section{众神保护下的树}



{\bfseries \color{red}}

\section{两个仇人}

从前,有两个仇人同乘一艘船去航海,一个坐在船尾,另一个坐在船头。海上风暴大作,船眼看就要沉了,船尾的那人问舵工,船的哪一部分会先沉去。舵工说:“是船头。”那人说:“现在我死无遗憾了,我将能看到我的仇人死在我的前头。”

{\bfseries \color{red}这故事说明,有些人,报复仇人的愿望比保护自己生命的愿望更强烈。}

\section{宙斯与受气的蛇}

有条蛇常被人们所践踏,便跑去向宙斯告状。宙斯对他说:“你若咬了第一个践踏你的人,就不会再有第二个敢这样做的人了。”

{\bfseries \color{red}这故事说明,抵抗住第一个侵略者,其他的侵略者就会望而生畏,不敢来犯。}

\section{蝮蛇和狐狸}

盘缠在一捆荆棘上的蝮蛇,顺着河水漂流。狐狸在河边看见后说:“这船主与船倒很匹配。”

{\bfseries \color{red}这故事说的是想做坏事的恶人。}

\section{蝮蛇和水蛇}

蝮蛇常常到泉水边饮水。住在那里的水蛇,对蝮蛇不满足自己的领地,却要跑到别人的领地里来喝水,十分生气,出来阻拦他。他们俩的争吵愈演愈烈,并约定互相交战,谁赢了,就把这水陆领地全都交给谁。交战的日期决定之后,那些仇恨水蛇的青蛙,跑到蝮蛇那里激励他,并且答应为他助一臂之力。战斗开始了,蝮蛇向水蛇猛攻,可青蛙除叫唤外,什么也不能做。蝮蛇胜利后,责怪青蛙,虽许诺给他助战,却不但不帮一把,只会唱个不停。青蛙对他说道:“啊,朋友,你知道,我们不是用手助战,而是用声音。”

{\bfseries \color{red}这故事说明,在必要用手帮忙的时候,用再好的言语也是毫无用处的。}

\section{鹞子与蛇}

鹞子抓住一条蛇飞到空中,蛇回过头来,猛咬了他一口。他俩便从高空中摔了下来,鹞子被摔死了。蛇对他说:“你为什么这么狠毒,去危害没有做坏事的人呢?你是罪有应得。”

{\bfseries \color{red}这故事是说,有些人贪得无厌,四处作恶,及至遇到了强者,受到了应有的报应。}

\section{蛇的尾巴与身体}

有一天,蛇的尾巴拼命争吵着要由他领路。蛇的其他部分说:“你没有眼睛鼻子,怎么能指引我们向前走?”尾巴却什么道理也不听。于是,他便来领路,拖着全身乱冲乱撞,结果掉进一个石洞里,蛇的全身都被摔坏了。尾巴摇摆着乞求蛇头,说,“救救我们吧,我的争吵真是太无聊了!”

{\bfseries \color{red}这故事是说那些好胜而不自量力的人。}

\section{蛇、黄鼠狼与老鼠}

蛇和黄鼠狼在一所房子里打架。同住在房里的老鼠常常被他们吃掉,现在一见他们在打架,便纷纷跑出来。然而,他们双方一见到老鼠,便立刻停止了互相的厮杀,一齐朝老鼠扑过去。

{\bfseries \color{red}这是说,那些自行卷入政客们互相争权夺利的人,不知不觉地成了政客们的牺牲品。}

\section{蛇与蟹}

蛇与蟹住在一起。蟹总是真诚和友好地与蛇相处,而蛇却阴险卑鄙。蟹经常劝告蛇要真诚、正直,而蛇却全当耳边风。因此蟹十分气愤,忍无可忍,趁蛇睡着时,把蛇掐死了。蟹看着蛇僵直地躺在地下,说道:“喂,朋友,现在你死了,也用不着真诚、正直了,你要是能听我的劝告,就不至于被杀。”

{\bfseries \color{red}这故事是说,对有些人来说,也许他死了对人们还好一些。}

\section{蛇和鹰}

蛇和鹰互相交战,斗得难解难分。蛇紧紧地缠住了鹰,农夫看见了,便帮鹰解开了蛇,使鹰获得了自由。蛇因此十分气愤,便在农夫的水杯里放了毒药。当不知情的农夫端起杯子正准备喝水时,鹰猛扑过来撞掉了农夫手中的水杯。

{\bfseries \color{red}这故事说明,善有善报,好人一定能得到好报。}

\section{庸医}

从前,有一个庸医。他给一个病人看病,其他的医生都说这病人没有什么危险,仅需要一段时间,就能康复。他却叫病人准备后事,并说:“你已经活不过明天了。”过了些日子,这病人的病情略有好转,脸色苍白地出外散步。那医生遇见他,说道:“你好,地下的人们怎么样?”他回答说:“喝了忘河的水,很安静的。但不久以前,死神和冥王因医生们没让病人死去,大肆威胁和恐吓他们,并把他们的名字都一一记下。本来你的名字也要被记下,但是我跪在死神和冥王面前,苦苦哀求,并发誓说你不是真正的医生,而是被别人误认为的。”

{\bfseries \color{red}这故事揭露了那些既无知识和医术,又要吹牛行骗的江湖庸医。}

\section{嘶叫的鹞子}

最初鹞子能发出一种动听的尖叫声。当他听见马嘶叫后,觉得非常好听,十分喜欢,便不断使劲地去学马那样的嘶叫声。最终不但一点没有学会,而且连自己原来的叫声也不会了。

{\bfseries \color{red}这故事是说,那些好高骛远的人总想要他本性以外的东西,到头来得不偿失,连他自己本来具有的东西都丧失了。}

\section{捕鸟人与眼镜蛇}

捕鸟人拿着粘鸟胶与粘竿外出捕鸟。他看到一只鸟栖息在一棵大树上,就想要去捕捉它。于是,他接长了粘竿,仰着头全神贯注地盯着高空中的那只鸟。正当他这样聚精会神时,不知不觉地踩着了一条躺在他脚前的眼镜蛇。蛇马上回过头来,狠咬了他一口。他中了蛇毒,临死之前,自言自语地说:“我真倒霉,光想去捉别人,不料自己反遭其害,丢了性命。”

{\bfseries \color{red}这故事是说,那些想阴谋陷害别人的人自己会先遇到灾难。}

\section{捕鸟人、野鸽和家鸽}

捕鸟人布上网,把几只家鸽拴在网里,然后躲在远处看着。有些野鸽飞到家鸽旁边去,一下就被兜在网里。当捕鸟人跑去捉住野鸽时,野鸽责骂家鸽,说同他们原本是同族,却不把这诡计预先告诉他们。家鸽回答说:“对我们来说,维护主人的利益比照顾自己的亲族更重要呀。”

{\bfseries \color{red}这是说,不必指责那些为了爱护自己的主人,而背弃亲族情谊的奴仆。}

\section{捕鸟人和鹳}

捕鸟人布下捕鹤的网,躲藏在远处等候飞来的猎物。一只鹳鸟和几只鹤一起飞进了网里,捕鸟人马上跑过去,把他们全都捉住了。鹳鸟请求把他放了,说他对人有益无害,他能捕杀蛇和别的害虫。捕鸟人回答说:“即使你并不算坏人,但你与坏人们在一起,也应该受到惩罚。”

{\bfseries \color{red}这是说,人们应该避免与坏人交往,以免被怀疑与他们所干的坏事有关系。}

\section{捕鸟人和斑鸠}

有个客人很晚来到捕鸟人家,捕鸟人没有食物招待客人,便跑去捉了那只驯养的斑鸠,想要杀了它招待客人。斑鸠痛斥他忘恩负义,说自己曾帮他招引来了许多如同自己一样的斑鸠,使他得到很大的利益,现在却要被杀掉。捕鸟人说道:“这样就更应该杀了你,因为你连同类也不放过呀。”

{\bfseries \color{red}这故事说明,那些背叛亲人的人,不但为亲人们所憎恨,也为其主子所厌恶。}

\section{母鸡与燕子}

母鸡发现了一个蛇蛋,小心翼翼地孵化,细心地给它啄开蛋壳。燕子看见后,说:“傻瓜,你为什么要孵化这坏蛋呢?它一长大首先就会伤害你。”

{\bfseries \color{red}这故事是说,尽管人们仁至义尽,本性恶劣的人也不会变好。}

\section{老马}

一匹老马被人卖了去拉磨。当他被套上轭时,悲伤地说:“我从跑马场冲到了这样一个终点。”

{\bfseries \color{red}这是说,人也许到老年时还会遇到艰辛。}

\section{马、牛、狗与人}

宙斯创造了人,没给人长寿,却给了人聪明才智。在冬天,人给自己建造好了房屋,舒适地住在里面。有一天,天气异常库冷,还下着雨,马冻得再忍受不住了,便跑到人那里,请求让它住在屋内避寒。人说除非马同意把它的部分寿命送给人,否则就不让它进门。马高兴地答应了。不久之后,牛也忍受不了寒冬,跑来找人。那人同样地说,除非牛能把部分寿命送给人,不然就不收留它。牛献出了部分寿命后,被收留下来。最后,狗冻得几乎要死了,也跑来把自己的部分寿命送给人,得到住处。这样,人在宙斯所给的年岁内,纯洁而善良;到了马给的年岁,就吹牛说大话,自命不凡;到了牛给的年岁,开始干事业;而到狗给的年岁,便容易发脾气,动不动就大吵大闹。

{\bfseries \color{red}这故事适用于爱发脾气的固执的老人。}

\section{马与兵}

战争期间,一个士兵用大麦精心地喂养他的马。然而战争一结束,那马便被拉去服苦役,搬运沉重的货物。后来战火重燃,军号吹响了,主人备好马鞍,全副武装骑着马去迎敌。这时,马却毫无力气,不断摔倒,他对他主人说:“你还是赶快再去找一匹战马吧。因为你已把我变成一头驴子,又怎么还能把我当战马骑呢?”

{\bfseries \color{red}这是说,和平、舒适的日子里不能忘记了灾难。}

\section{大树和芦苇}

有一天,狂风刮断了大树。大树看见弱小的芦苇没受一点损伤,便问芦苇,为什么我这么粗壮都被风刮断了,而纤细、软弱的你什么事也没有呢?芦苇回答说:“我们感觉到自己的软弱无力,便低下头给风让路,避免了狂风的冲击;你们却仗着自己的粗壮有力,拼命抵抗,结果被狂风刮断了。”

{\bfseries \color{red}这故事是说,遇到风险时,退让也许比硬顶更安全。}

\section{核桃树}

有棵核桃树生在路旁,结了很多核桃,路过的人们都用石头去打树上的果实。核桃树暗自叹息,自言自语:“我真倒霉,每年我给人们带来了果实却为自己招来许多侮辱与苦恼。”

{\bfseries \color{red}这故事是说那些因自己行善而吃苦的人们。}

\section{河里拉屎的骆驼}

有一匹骆驼渡过湍急的河时,在河中拉屎,他看见那粪便一下就被急流的河水冲到了他的前面。他说:“为什么我看见在我后面的却一下到了我的前面?”

{\bfseries \color{red}这故事是说,愚昧无知的人只看到现象却不理解原因。}

\section{蔷薇与鸡冠花}

蔷薇和鸡冠花生长在一起。有一天,鸡冠花对蔷薇说:“你是世上最美丽的花朵,神和人们都十分喜爱你,我真羡慕你有漂亮的颜色和芬芳的香味。”蔷薇回答说:“鸡冠花啊,我仅昙花一现,即使人们不去摘,也会凋零,你却是永久开着花,青春常在。”

{\bfseries \color{red}这是说,事物各有所长,也各有所短,不必羡慕别人有你所没有的东西,你也有别人所没有的东西。}

\section{骆驼、象、猴子}

无知的动物们要选举国王,骆驼和象也积极去参加竞选,一个身材高大,一个力气超群,他们都希望能战胜他人而当选。然而,猴子认为他们俩都不适合,他说:“骆驼一贯温顺,对于做坏事的动物也不生气;而象总害怕那小猪,不像国王。”

{\bfseries \color{red}这故事说明,有许多人都是因小失大。}

\section{跳舞的骆驼}

从前,有一个人逼迫他的骆驼跳舞,骆驼说:“我连走路的姿势都不雅观,又怎能跳舞?”

{\bfseries \color{red}这故事适用于一切不恰当的行为。}

\section{人与骆驼}

人们第一次看见骆驼时,对这些庞然大物感到十分恐惧和震惊,都吓得纷纷逃跑。随着时间的推移,他们渐渐地发现骆驼的脾气温顺,便壮着胆子,勇敢地去接近它。过了不久,人们完全明白骆驼这动物根本没一点脾气,于是便瞧不起它了,还给它们装上缰绳,交给孩子们牵着走。

{\bfseries \color{red}这故事说明,熟悉和了解事物能消除对事物的恐惧。}

\section{蟹与狐狸}

一只螃蟹离开海水,独自住在海岸边。有只饥饿的狐狸正愁没有吃的,看见他后,便跑过去捉住了他。蟹在将要被狐狸吞食之前,说:“我真是活该!我本应该生活在大海里,却偏要到陆上来。”

{\bfseries \color{red}这故事是说,有些人抛弃自己熟知的事情,而去做一无所知的事,结果遇到了不幸。}

\section{狐狸和狮子}

一只狐狸从来没有见过狮子,偶然一次,他在森林里碰到了狮子,被吓得半死。当他第二次遇到狮子时,仍很害怕,但比第一次好得多了。第三次遇到狮子时,他竟有胆量,走了上去,与狮子进行十分亲切的谈话。

{\bfseries \color{red}这故事是说不要害怕不了解的事物,接近它,就会觉得没什么可怕的。}

\section{狐狸和荆棘}

一只狐狸爬越篱笆时,差一点跌了下去,他拼命抓住了一根荆棘。脚被荆棘刺破了,他痛得抱怨荆棘说,自己仅向他求助,而他却比篱笆还坏。荆棘说:“我总是习惯于依附别人,你却来依附我,那实在是太愚蠢了。”

{\bfseries \color{red}这是说千万不要依靠那些不能依靠的人。}

\section{跳蚤和公牛}

有一天,跳蚤问牛:“你这般高大强壮而且勇敢,为什么还终日去为人们耕作?而我这只区区的小虫,却能毫无顾忌地去叮咬人,大口地吸他们的鲜血!”公牛回答说:“我一定要报答人类的恩德,因为他们都喜欢我,经常替我擦洗身体,并抚mo我的额角,我从他们那里得到了爱。”跳蚤说:“你喜欢的这些方式,我都受不了。人们一旦抓住我,用对付你的方法,将会要我的命。”

{\bfseries \color{red}这是说,得到怎样的待遇,就会采取怎样的态度去回报。}

\section{跳蚤和人}

有一天,一只小小的跳蚤在一个人身上跳上跳下,不断地叮咬他,弄得他极其难受。他一把抓住跳蚤,问它:“你是谁?怎么在我身上四处叮咬,使我到处骚痒?”跳蚤说:“请饶恕我,千万别捏死我!我们一直就是这样的活着,虽然不断地骚扰人们,但决不会去干更大的坏事。”那人笑着说:“罪恶不论大小,只要祸及别人,就决不能留情,所以一定要捏死你。”

{\bfseries \color{red}这故事说明,坏人无论大小都应坚决加以惩治。}

\section{两只屎壳郎}

一个小岛上放养着一头牛,有两只屎壳郎靠吃牛粪而生活。冬季来临,一只屎壳郎对另一只说,他想立刻飞到大陆去过冬,让那只单独留在岛上,这样食物就足够充饥了。他还说,如果他发现丰富的食物,将带些回来。他飞到了大陆,发现有许多牛粪,还是稀的,就在那里停留下来过冬。冬天一晃就过去了,他又飞回到岛上。另一只屎壳郎看见他浑身油光发亮,长得又肥又壮,便责怪他不履行诺言,什么都没有带回来。他答道:“你不要责怪我,去责备那地方的自然条件吧,因为只可以在那里吃,一点也不能带回来。”

{\bfseries \color{red}这故事是说不讲信用的朋友是不可信赖的。}

\section{河狸}

河狸是生活在水中的四足动物。据说它的yin部可用于治疗某种病,因此人们一看见它就追赶,要捉住它,割下它的yin部来。海狸知道被追赶的原因,便靠腿的力量竭力逃生,以保护自己的身体。每到它要被捉住时,它便把自己的yin部撕下来,抛出去,这样就能保全自己的生命。

{\bfseries \color{red}这故事是说,聪明的人宁愿抛弃财富,以保全自己的生命。}

\section{苍蝇}

苍蝇掉进一口盛着肉汤的瓦锅里,快要被淹死时,他自言自语说:“我已经吃饱了,喝足了,洗过澡了,即使死了我也不遗憾。”

{\bfseries \color{red}这故事说明,人们容易忍受无痛苦的死。}

\section{蚂蚁}

很久很久以前,蚂蚁本来是人,他们有田可耕,有地可种。但他们不满足于自己劳动所得,不愿辛勤劳作,却很羡慕别人的东西,经常去偷邻居的果实。宙斯对他们的贪婪感到气愤,便把他们变成了现在称之为蚂蚁的小动物。虽然他们的模样改变了,本性却依然如旧,直到现在他们仍在别人的田里走来走去,拾捡小麦和大麦,贮存在自己的窝里。

{\bfseries \color{red}这故事说明,本性恶劣的人,即使受到了最严厉的惩罚,恶习也不会改变。}

\section{蝉与狐狸}

蝉在大树顶上鸣唱。狐狸想要吃掉他,便想出了一个诡计。他站在树下,一会儿赞美蝉的歌声悦耳动听,一会儿羡慕地看着蝉,认真欣赏他的歌声,并劝蝉下来,说他想要看一看是什么样的动物才能发出这么悦耳的声音。蝉识破了他的诡计,便摘了一片树叶抛下去。狐狸以为是蝉,猛扑过去,抓住它。蝉说道:“喂,坏家伙,你若以为我会飞下来,那就大错特错。我自从见到狐狸的粪便里有蝉的翅膀之后,就时刻警惕狐狸。”

{\bfseries \color{red}这是说,聪明的人懂得从邻人的灾难中吸取教训。}

\section{蝉与蚂蚁}



{\bfseries \color{red}}

\section{弹琵琶的人}

有个天生不善弹琵琶的人,常常在声音效果较好的室内弹唱。听着室内回响的声音,他洋洋得意,自以为自己的嗓音非常不错。心想自己完全可以去剧场登台表演了,可他登场之后,唱得极差,台下的人们扔石头把他轰赶下来了。

{\bfseries \color{red}这是说,有些演说家在学校里还有模有样,似乎讲得头头是道,但遇到讨论国家大事时却是毫无价值的人。}

\section{觅食的鸟}

一只鸟在林中的树上觅食,树上的果子酸甜可口,他再不愿离开。捕鸟的人看到鸟喜欢这里,便拿来粘竿把它活捉了。临死时,他说:“我真不幸,因为贪吃,图一时快乐,而把自己的性命都丢掉了。”

{\bfseries \color{red}这故事适用于那些为了一时快乐而丧失生命的人。}

\section{小偷与公鸡}

几个小偷悄悄地溜进一户人家里,什么也没偷到,仅发现一只公鸡,便抓住他偷走了。当小偷们要杀公鸡时,公鸡请求放了他,并说他对人们是有益处的,每天天不亮时,他就把人们叫醒起来去工作。小偷们回答说:“单凭这一点,非要你死不可,你把人们都叫醒就妨碍了我们偷盗。”

{\bfseries \color{red}这故事说明,那些对于好人有益的事正是对于坏人有害的。}

\section{池塘里的蛙}

两只青蛙住在池塘里。夏天,池塘干涸了,他们不得不离开那里,四处寻找安身之处。他们来到一个很深的井旁,其中一只不假思索地对另一只说:“喂,朋友,这里井水多好啊!让我们一起到这井里去住吧。”另一只回答说:“这里的水如果也干了,我们又怎么爬上来呢?”

{\bfseries \color{red}这故事告诫我们,凡事要三思而后行,切不可轻率从事。}

\section{猫和公鸡}

一只猫抓到一只公鸡,并想出吃他的口实。他指责公鸡在夜晚打鸣,使人不能安睡,人们都讨厌公鸡。公鸡辩解说,他是为了人们的利益而啼叫,那样可以使人们按时起床工作。猫回答说:“尽管你说的似乎合理,但我总不能不吃晚餐。”于是,毫不客气地把公鸡吃掉了。

{\bfseries \color{red}这是说,坏人干坏事总是能找到借口的。}

\section{孔雀和白鹤}

孔雀看不起白鹤羽毛的色泽,她一边张开美丽羽毛,一边讥笑他说:“我披挂得金碧辉煌,五彩缤纷;而你的羽毛一片灰暗,十分难看。”鹤说道:“可我翱翔于太空,在星空中歌唱;而你却同公鸡和家禽一般,只能在地上行走罢了。”

{\bfseries \color{red}这故事是说,穿戴简朴而志趣高洁的人远胜于披金戴银而平庸凡俗的人。}

\section{孔雀与寒鸦}

众鸟在一起商议选举国王,孔雀认为他美丽漂亮,应该被拥立为王。众鸟正在准备一致推举孔雀为王时,寒鸦说:“若你做了国王,鹰来攻击我们时,你能保护我们吗?”

{\bfseries \color{red}这故事说明,衡量一个人,不能只看外貌,重要的得看他的能力和智慧。}

\section{狮子、老鼠和狐狸}

酷热的天气使狮子疲惫不堪,他躺在洞中酣睡。一只老鼠从他的鬃毛和耳朵上跑过,将他从梦中吵醒。狮子大怒,爬起来摇摆着身子,四处寻找老鼠。狐狸见到后说:“你是一只威严的狮子,也被老鼠吓怕了。”狮子说:“我并不怕老鼠,只是恨他放肆和无礼。”

{\bfseries \color{red}这是说,有时候一点点小小的自由都是很大的冒犯。}

\section{狮子和鹰}

一只鹰停止飞行,请狮子与他结盟,以谋求他们相互的利益。狮子回答说:“我不反对,但你须原谅我,请你找一个担保你信用的保证人。一个可以随时违约飞去的人,我怎能信任他做朋友呢?”

{\bfseries \color{red}这是说交友时一定要经过慎重的考虑。}

\section{狮子国王}

有只狮子做了国王,他善良、温和,与人一样和平、公正。在他的统治下,惩恶扬善,裁决动物之间的纠纷,使所有的动物和睦相处。胆小的兔子说:“我祈祷能得到这样的日子,那时弱者就不怕被强者伤害了。”

{\bfseries \color{red}这是说,在正义的国家里,一切事都公平处理,那么弱小者的生活也会平安。}

\section{狮子和兔}

狮子发现兔子正在睡觉,便想趁机吃掉他。这时,狮子又看见有只鹿走过,便丢下兔子去追赶鹿。兔子听到声响,马上跳起来逃跑了。狮子使劲追鹿,仍没有追到,于是又回头来找兔子,却发现兔子早已逃之夭夭。狮子说:“我真活该!丢掉已到手的食物,却贪心去追求那更大的希望。”

{\bfseries \color{red}这是说,有些人不满足手中的小利,去追求更大的希望。结果,不但不知不觉地把手中的小利丢掉了,更大的希望也没追到,只留得两手空空。}

\section{狮子、普罗米修斯与象}



{\bfseries \color{red}}

\section{狮子和野猪}

夏季,炎热的酷暑使人很口渴,狮子和野猪一起来到小泉边喝水。他们为谁先喝,彼此争斗得你死我活。当他们喘气时,忽然回过头去,看见有几只秃鹰正在等候,他们知道谁倒下去谁就会被吃掉。因此他们停止了争斗,并说:“我们还是成为朋友吧,总比被秃鹰和大鸦吃掉好得多。”

{\bfseries \color{red}这是说,人们不要相互进行无聊的争斗,否则,会给自己招来灾难。}

\section{疯狮子与鹿}

狮子发疯了。鹿在树林中看着他说:“啊呀,我们真不幸!平时都叫我们受不了,那么他疯后又会弄得我们怎样呢?”

{\bfseries \color{red}这是说,那些一贯作恶多端的人尽管一时得势,但大家都躲避他。}

\section{狮子、狐狸与鹿}

狮子生了病,睡在山洞里。他对一直与他亲密要好的狐狸说道:“你若要我健康,使我能活下去,就请你用花言巧语把森林中最大的鹿骗到这里来,我很想吃他的血和心脏。”狐狸走到树林里,看见树林里欢蹦乱跳的大鹿,便向他问好,并说道:“我告诉你一个喜讯。你知道,国王狮子是我的邻居,他病得很厉害,快要死了。他正在考虑,森林中谁能继承他的王位。他说野猪愚蠢无知,熊懒惰无能,豹子暴躁凶恶,老虎骄傲自大,只有大鹿才最适合当国王,鹿的身材魁悟,年轻力壮,他的角使蛇惧怕。我何必这么罗嗦呢?你一定会成为国王。这消息是我第一个告诉你的,你将怎样回报我呢?如果你信任我的话,我劝你快去为他送终。”经狐狸这么一说,鹿给搞糊涂了,便走进了山洞里,丝毫没有想会发生什么别的事情。狮子猛然朝鹿扑过来,用爪子撕下了他的耳朵。鹿拼命地逃回树林里去狐狸辛辛苦苦白忙一场,他两手一拍,表示毫无办法了。狮子忍着饿,叹惜起来,十分懊丧。狮子请求狐狸再想想办法,用狡计把鹿再骗来。狐狸说:“你吩咐我的事太难办了,但我仍尽力去帮你办。”于是,他像猎狗似地到处嗅,寻找鹿的脚迹,心里不断盘算着坏主意。狐狸问牧人们是否见到一只带血的鹿,他们告诉他鹿在树林里。这时,鹿正在树林里休息,狐狸毫不羞耻地来到他的面前。鹿一见狐狸,气得毛都竖了起来,说:“坏东西,你休想再来骗我了!你再靠近,我就不让你活了。你去欺骗那些没经验的人,叫他们做国王。”狐狸说:“你怎么这样胆小怕事?你难道怀疑我,怀疑你的朋友吗?狮子抓住你的耳朵,只是垂死的他想要告诉你一点关于王位的忠告与指示罢了。你却连那衰弱无力的手抓一抓都受不住。现在狮子对你非常生气,要将王位传给狼。那可是一个坏国王呀!快走吧,不要害怕。我向你起誓,狮子决不会害你。我将来也专伺候你。”狐狸再一次欺骗了可怜的鹿,并说服了他。鹿刚一进洞,就被狮子抓住饱餐了一顿,并把他所有的骨头,脑髓和肚肠都吃光了。狐狸站在一旁看着,鹿的心脏掉下来时,他偷偷地拿过来,把它当作自己辛苦的酬劳吃了。狮子吃完后,仍在寻找鹿的那颗心。狐狸远远地站着说:“鹿真是没有心,你不要再找了。他两次走到狮子家里,送给狮子吃,怎么还会有心呢!”

{\bfseries \color{red}这故事是说,有些人图虚荣,不辨真伪,给自己招来灭顶之灾。}

\section{狮子和青蛙}

狮子听见青蛙大声叫喊,便朝声音发出的方向转过头去细心察看,心想一定是什么大动物。他等了一会儿,看见青蛙从池塘里蹦了出来,便走过去,一脚踩住它,说:“这么一个小东西叫声却那么大。”

{\bfseries \color{red}这故事是说那些多嘴多舌的人,除了说空话,别无所能。}

\section{狮子、狼与狐狸}

年老的狮子重病躺在洞里。除了狐狸之外,动物们都去问候国王。狼便趁机在狮子面前诬陷狐狸,说狐狸胆大包天,藐视大王,竟敢不来问候。正在此时,狐狸进来了,听到了狼所说的最后几句。狮子一见到狐狸就怒吼起来,狐狸马上请求让他解释几句。他说:“在所有向大王问候的动物之中,有谁像我这样忠诚,为你四处奔走,遍访名医,寻找妙方呢?”狮子立即命令他将药方说出来。狐狸说:“将狼的皮活剥下,趁热裹在身上。”狼立刻成为一具尸体,躺在了那里。狐狸得意地笑着说:“你不应当怂恿主人起恶念,而应该诱导他发善心才对呀。”

{\bfseries \color{red}这故事说明,常常算计别人的人,往往会自食其果。}

\section{蚊子与狮子}

有只蚊子飞到狮子那里,说:“我不怕你,你也并不比我强多少。你的力量究竟有多大?是用爪子抓,还是用牙齿咬?仅这几招,女人同男人打架时也会用。可我却比你要厉害得多。你若愿意,我们不妨来比试比试。”蚊子吹着喇叭,猛冲上前去,专咬狮子鼻子周围没有毛的地方。狮子气得用爪子把自己的脸都抓破了,最后终于要求停战。蚊子战胜了狮子,吹着喇叭,唱着凯歌,在空中飞来飞去,不料却被蜘蛛网粘住了。蚊子将被吃掉的时候,悲叹道:“我已战胜了最强大的动物,却被这小小的蜘蛛所消灭。”

{\bfseries \color{red}这故事是说,骄傲是没有好下场的,有些人虽击败过比自己强大的人,也会被比自己弱小的人击败。}

\section{种菜人}

种菜人正在菜园里浇菜,有个人跑过来问他,为什么野菜生得很茂盛,人们栽种的菜却很瘦弱。他回答说:“地是野菜的亲娘,却是家菜的后娘。”

{\bfseries \color{red}这故事适用于那些为自己的懒惰编造借口的人。}

\section{种菜人与狗}

种菜人的狗掉到了井里。他想把狗从井里救上来,于是他自己也下到井里,可狗却以为主人下来是要把它再捺到水里去,尽快淹死它。所以当种菜人挨近狗时,狗便转过身来,咬了他一口。种菜人忍着巨痛一边往上爬,一边说:“我真是活该!为什么我要这么热心去救它呢?”

{\bfseries \color{red}这故事是说那些无情无义、恩将仇报的人。}

\section{两只狗}

有个人养了两只狗,他驯养一只狗狩猎,另一只看家守门。每次猎人带着猎狗出去打猎,获得什么猎物,总是分给守门狗一些。猎狗对此很不高兴,便指责守门狗,说自己每次出去打猎都是四处奔跑,十分辛苦,而他什么都没有做,却坐享其成。守门狗对猎狗说:“你别责怪我,应该去责怪主人,是他教我不去打猎,坐在家中享受别人的劳动果实。”

{\bfseries \color{red}这是说,不要责怪孩子的懒惰,因为是父母把他们惯成这样的。}

\section{狼与狗}

一只白胖白胖的狗套着颈圈,狼见到后,便问他:“你被谁拴住了,养得你这么肥胖?”狗说:“是猎人。但愿你不要受我这样的罪,套着沉重的颈圈比挨饿难受得多。”

{\bfseries \color{red}这故事说明,对于失去自由的人来说,即使最好的美食也都索然无味。}

\section{小狗和青蛙}

炎热的夏天,随同主人赶路的小狗奔跑了一天,到了晚上,他便昏昏沉沉地躺下,在池塘边潮湿的草地上睡着了。小狗睡得真香,池塘旁的青蛙像往常一样,齐声哇哇地叫了起来。小狗被他们的叫声吵醒了,十分恼怒。他心想如果马上跳到塘里,狂叫几下,吓唬吓唬他们,定能使他们不再吵闹了,然后自己再舒适地睡上一觉。他接二连三地喊了几次,毫无作用,只好回到塘边,十分气愤地说:“我真是太愚蠢了,这些天生爱吵闹的东西怎么会变得文质彬彬,体贴他人呢?”

{\bfseries \color{red}这故事是说,那些骄傲自大的人总是目空一切,为所欲为,不顾他人。}

\section{牧羊人与狗}

有个牧羊人养着一条壮实的狗,他常常把那些死了的羊喂给狗吃。有一天,羊群都回到羊圈里,牧羊人却看见狗走近羊群,去抚弄它们,他便说:“喂,伙计,你想要对羊做的事,也许会落在你头上!”

{\bfseries \color{red}这故事适用于那些受到优待而还不知足的人。}

\section{猪与狗}

猪与狗互相谩骂。猪向阿佛洛狄忒发誓,一定要用牙齿把狗撕咬得四分五裂。狗却嘲弄他说:“你向阿佛洛狄忒发誓那太好啦,她最痛恨你们这些愚蠢的猪,决不允许任何吃过猪肉的人进入她的圣庙。”猪回答道:“女神如此规定不是出于恨我,而正是对我的厚爱。她这样做是为了防止有人杀害我,吃我的肉。你们才是被女神痛恨的,不管是死是活,都可以拿去祭祀。”

{\bfseries \color{red}这故事说明,聪明的人将对手的非难巧妙地转化为对他的赞颂。}

\section{鬣狗}

传说鬣狗每年都要变更他们的性别,有时变为雄的,有时又变为雌的。有一次,一头雄鬣狗对雌鬣狗大发淫威。雌鬣狗说:“喂,伙计,你这样地干吧,不久你也会遭受到这种侮辱。”

{\bfseries \color{red}这故事说明,人们做任何事时都必须考虑别人,说不定有些事也会落到自己头上。}

\section{猪与狗争论生产}

猪与狗互相为谁生产顺利大吵大闹,争论不休。狗说:“四只脚的动物中,我生产最短。”猪回答说:“也许你所需时间最短,但你应该明白你生的是瞎子。”

{\bfseries \color{red}这故事说明,判定事物的好坏,不完全取决于速度,而要看本质的好坏。}

\section{小偷和狗}

有只狗从小偷身边走过,小偷连忙将面包分成小块,不停地扔给他吃。狗却对小偷说:“喂,伙计,快滚开些!我非常害怕你这般好意。”

{\bfseries \color{red}这故事是说,那些送厚礼的人必另有所图。}

\section{母狗和她的小狗}

一条母狗将要生产小狗了,她急切地跑去请求牧人给她一处生产的地方。她的要求被答应后,又请求给她一处地方抚养小狗,牧人也允许了。然而,当小狗长得身高体壮后,这条母狗竟对牧人提出对这块地方她有独占的权利,不准别人靠近。

{\bfseries \color{red}这故事是说有些人总是贪得无厌,欲壑难填。}

\section{家狗和狼}

一条饥饿的瘦狼在月光下四处寻食,遇到了喂养得壮实的家狗。他们相互问候后,狼说:“朋友,你怎么这般肥壮,吃了些什么好东西啊?我现在日夜为生计奔波,苦苦地煎熬着。”狗回答说:“你若想像我这样,仅只要学着我干就行。”“真是这样,”狼急切地问,“什么话儿?”狗回答说:“就是给主人看家,夜间防止贼进来。”“什么时候开始干呢?”狼说,“住在森林里,风吹雨打,我都受够了。为了有个暖和的屋子住,不挨饿,做什么我都不在乎。”“那好,”狗说,“跟我走吧!”他们俩一起上路,狼突然注意到狗脖子上有一块伤疤,感到十分奇怪,不禁问狗这是怎么回事。狗说:“没什么。”狼继续问:“到底是怎么回事?”“一点点小事,也许是我脖子上拴铁链子的颈圈弄的。”狗轻描淡写地说。“铁链子!”狼惊奇地说,“难道你是说,你不能自由自在随意地跑来跑去吗?”“不对,也许不能完全随我的心意,”狗说,“白天有时候主人把我拴起来。但我向你保证,在晚上我有绝对的自由;主人把自己盘子中的东西喂给我吃,佣人把残羹剩饭拿给我吃,他们都对我倍加宠爱。”“晚安!”狼说,“你去享用你的美餐吧,至于我,宁可自由自在地挨饿,而不愿套着一条链子过舒适的生活。”

{\bfseries \color{red}这是说,自由比安乐更重要。}

\section{猎狗和狐狸}

有条猎狗看见狮子,便追赶上去。当狮子回过头来大声吼叫时,他却被吓慌了,掉头向后逃跑。狐狸见状,说,“胆小鬼!一声吼都受不住,你还去追什么狮子?”

{\bfseries \color{red}这故事是说,有些人,千方百计表现自己的强大的人,当他们面对强者时,却立刻被吓得落荒而逃。}

\section{狗和屠夫}

狗溜进肉店里,趁屠夫正忙着,偷了一个猪心就跑。屠夫回过头来,看见狗正在逃,便说:“喂,你这畜牲,你记清楚,今后不论你跑到哪里,我都会留心提防着,你偷跑了我一个猪心,却把另一个心给了我。”

{\bfseries \color{red}这故事说明,灾祸常成为人们的学问,也就是说,吃一堑,长一智。}

\section{猎狗与众狗}

有个人养着一条强壮的猎狗,主人想让它去追逐野兽。可他一见到一队队行走的野兽,就拼命将颈圈挣脱,使劲地逃跑。其他狗见到这只壮得像公牛似的猎狗,便问:“你为什么如此仓皇逃窜?”猎狗说:“我知道,我虽吃住不愁,生活舒适;但命我去追逐熊和狮子,那我就离死不远了。”那些狗似乎明白了什么道理,便说:“尽管我们缺衣少食,生活简陋,却不要去和凶猛的狮子或熊拼搏。”

{\bfseries \color{red}这故事是说,荣华富贵舒适享乐往往和危险相连,而清贪简陋的生活却是安全的。}

\section{乌鸦与狗}

有一次,乌鸦祭祀雅典娜,请狗来赴宴。狗对他说:“你怎么舍得花这么多钱办这毫无用处的祭祀呢?那女神不是很厌恶你,使得你的预兆一点都不灵吗?”乌鸦回答说:“正因为这样,我才给她祭祀,我知道她一向不喜欢我,总是跟我过不去,但我以祭祀与她和解。”

{\bfseries \color{red}这是说,许多人恐惧敌人,不惜代价想与他们和解。}

\section{田螺}

农夫的孩子在烧烤田螺吃时,听见田螺吱吱地响,便说:“啊,你们这些可怜的东西,家都被烧了,还有心唱歌。”

{\bfseries \color{red}这故事说明,那些不分场合的人常常受到人们的指责。}

\section{狗和海螺}

有只常常偷吃鸡蛋的狗,看见一只海螺,以为也是鸡蛋,张开大嘴,一口就把它吞下肚去。过了一会儿,他觉得肚子疼得十分难受,便说:“我真是活该,把所有圆的都当成了鸡蛋。”

{\bfseries \color{red}这故事告诉我们,不能单凭直觉和外表去认识事物,否则,往往会不知不觉地陷入困境。}

\section{兔与狐狸}

有一次,兔要与鹰打仗,他去请狐狸助威。狐狸却说:“如果我们既不认识你们,又不知道你们要与谁打仗,我们怎么会去帮助你呢?”

{\bfseries \color{red}这故事说明,有些人不顾自己的生命安全,硬要与比自己强大的对手去争斗。}

\section{狗与狐狸}

几条狗发现了一张狮子皮,便使劲用牙齿把它撕碎。狐狸看见了,说:“如果狮子活着,你们就会明白,你们的牙齿是不能与他的爪子相对抗的。”

{\bfseries \color{red}这是说,有些人风光一时,为人敬仰。一旦他们身败名裂,人们就会藐视他们。}

\section{野猪与狐狸}

有头野猪在路旁的树干上磨他的牙齿。狐狸问他,这里没有猎人,一时也没危险来临,为什么要磨牙齿。他说道:“我这样做是有道理的,一旦危险降临,就没磨牙的工夫了,那时我就可以使用磨好的利牙呀。”

{\bfseries \color{red}这故事是说,人们应当未雨绸缪,防患于未然。}

\section{小猪与狐狸}

农夫赶着驮着山羊、绵羊和小猪的驴进城。小猪一路上不停地拼命号叫,狐狸听见了,便问它:“那些羊为什么都安安静静,只有你这般号叫?”小猪回答说:“我并不是无缘无故地号叫,我十分清楚,主人捉绵羊是要它的毛和奶,捉山羊是要干酪和小羊羔,而捉我却是要杀我去祭祀。”

{\bfseries \color{red}这故事是说那些能预感灾难来临的人。}

\section{狼、狐狸和猿猴}



{\bfseries \color{red}}

\section{狮子和牧羊人}

一头狮子走过树林时,踩着了一根刺。他连忙跑到牧羊人面前,摇着尾巴向他亲热,好像在说请帮帮我。牧羊人壮着胆,仔细检查一番,发现了那根刺。接着,他将狮子的爪子放在膝上,将刺拔了出来,解除了狮子的痛苦。狮子返回树林中。不久牧羊人被他人诬告,关进了牢房,被判决喂狮子吃。狮子认出牧羊人是帮助他的人,不但没扑过去,反而慢慢地走近他,把爪子放在牧羊人的膝上。国王听说这事情后,下令赦免了牧羊人。

{\bfseries \color{red}这是说行善者必有善报。}

\section{披着狮子皮的驴子}

有头驴子披着狮子皮四处游荡,吓唬那些弱小无知的动物。他看见了狐狸,也想去吓唬吓唬他。狐狸正巧以前就听到过他的叫声,便对驴子说:“如果我听不出你的叫声,我也会害怕了。”

{\bfseries \color{red}这是说,有些人看起来神气十足,一表人材,然而一开口就原形毕露了。}

\section{驴子和马}

驴子请求马省一点饲料给他吃。马说:“好,为显示我高贵的尊严,如果我吃不完,剩下的就给你吃。晚上我回自己的厩中时,你若能来,我就给你满满一小袋麦子。”驴子回答说:“谢谢你,我才不会相信。现在你连一点饲料都不给我,过一会儿还能给我更大的好处?”

{\bfseries \color{red}这是说,别相信那些吝啬鬼假惺惺的许诺。}

\section{马和驴}

一匹马在路上炫耀他的精美的马饰,忽然遇到了一头满驮着货物的驴子。驴子因货物太重,只能慢慢地让开路。马傲慢地说:“我恨不得要用脚踢你。”驴子丝毫不予计较,只是默默地祈求神保佑。不久,那匹马患了气喘病,被主人送回农庄来。驴子看见拖着粪车的马,便讥笑他说:“骄横的东西,你那华丽的马饰现在到哪里去了?你怎么变成这样一副倒霉相?”

{\bfseries \color{red}这故事是说人们不能因一时荣华富贵而不可一世。}

\section{苍蝇和拉车的骡子}

一只苍蝇叮在四轮车的车轴上,对拉车的骡子说:“你为什么走得这么慢!干吗不跑快一点?看来需要我来叮咬你的颈部了。”骡子说:“我不怕你的恐吓,我只注意坐在你上面的那个人,他会用鞭子使我加快步伐,用缰拉我的头调整方向。你快滚开些吧,别再啰嗦了,我非常清楚什么时候该快,什么时候该慢。”

{\bfseries \color{red}这故事是说不要自以为是,去做那些超越自己范围的事。那样,只会使别人厌恶。}

\section{顽皮的驴}

一头驴子爬上了屋顶,在那里跳舞,将屋上的瓦片踩得粉碎。主人立刻爬了上去,将它赶下来,并用一根粗木棍狠狠地打了它一顿。驴子说:“怎么啦?我昨天看见猴子也是这样玩的,你们大家很开心,它的表演似乎使你们很高兴。”

{\bfseries \color{red}这故事是说那些不顾自身条件的人会遭到嘲笑。}

\section{买驴子的人}

有人买了一头驴子,想要牵走试一试。他把驴牵到自己的驴马之中,并让他站在马槽前。那驴子来到一头好吃懒做的驴子旁边。于是,买驴的人立刻给那头驴套上辔头,牵去还给驴的卖主。卖主问,你这方法可靠吗?那人答道:“不必怀疑了,依我之见,选择什么样的朋友,自己也就什么样。”

{\bfseries \color{red}这是说,物以类聚,人以群分。}

\section{野驴和家驴}

野驴看见家驴舒适地躺在阳光充足的地方,就走了过去,夸奖他身强力壮,还享受美味的食物。后来,野驴又看见家驴驮着沉重的货物,驴夫还跟在后面用棍棒边打边赶,野驴说:“我现在不再觉得你幸福了,我看得出,不遭受那百般痛苦是得不到那一点点享受的。”

{\bfseries \color{red}这是说,人们不必去羡慕那付出沉重代价所得到的利益。}

\section{狼与驴子}

有条狼被选为狼的首领。为了阻止狼互相争食打架,他制定了法律,规定各自猎得的食物都集中起来,再平均分配给大家,一头驴子走来,慢悠悠地摆着鬃毛说:“从狼的脑袋里竟想出了一个好主意。可你自己为什么不把昨天猎得的食物拿出来一起分呢?”狼被驴子说穿了,便把那法律废弃了。

{\bfseries \color{red}这是说,有些制定公正法律的人,常常自己不遵守所制定的法律。}

\section{驴与骡子}

有一天,驴夫赶着驴子和骡子一起驮货赶路。驴子十分气愤他们俩驮的东西一样多,而骡子认为自己应吃双倍饲料。他们刚走一会儿,驴夫看见驴子有点走不动了,便从他背上拿下一部分货物,加在骡子背上。他们又走了一会儿,驴夫看到驴累得更加不行了,又取了一部分货物,最后把驴驮的所有货物,全加在骡子背上。这时,骡子回过头对驴子说:“喂,朋友,你现在还气愤我吃双倍饲料吗?”

{\bfseries \color{red}这故事是说,不要与别人斤斤计较,各人都有自己该做的事,该得的酬劳。}

\section{驴子、乌鸦与狼}

一天,有头背上受了伤的驴子,在牧场上吃草。乌鸦飞到他背上,啄他的伤口,驴子痛得跳起来大叫。而站在远处的驴夫却若无其事,在那里发笑。有只狼正从这里走过,见到后,自言自语地说:“我真倒霉!只要我望一望驴子,就遭到人们追赶,而乌鸦飞到驴背上,人们还笑。”

{\bfseries \color{red}这故事说明,人们时时刻刻警惕那些专做坏事的人。}

\section{驴子们求宙斯}

有一次,驴子们不甚忍受长年沉重的劳作,派代表去宙斯那里,请求为他们减免些痛苦。宙斯知道这是不容改变的,便说当他们撒尿能成河时,就会免去那些苦难。驴子们心想宙斯的话决无戏言。所以现在只要一头驴撒尿,别的驴也会围在那里撒起尿来。

{\bfseries \color{red}这故事说明,天生的本性是无法改变的。}

\section{病驴和狼}

有一天,驴子生病躺在家,狼跑来看望他。他一边摸着驴子的身体,一边问他什么地方痛,驴子回答说:“你摸到的地方都痛。”

{\bfseries \color{red}这是说,那些假心假意的人表面关心你,实际上却想危害你。}

\section{野驴与家驴}

有一天,家驴背负着沉重的货物,显得格外劳累。野驴看见了,便责备他心甘情愿受欺压,说:“你看我自由自在、无忧无虑地生活着,还可以到山上去吃草,非常幸福。而你却过着痛苦而毫无自由的生活,不但十分劳累,还经常遭受主人的欺压和皮鞭抽打。”这时,狮子来了,他看到家驴和驴夫在一起,便向那孤单的野驴猛扑过去,抓住他吃了。

{\bfseries \color{red}这故事说明,自由固然可贵,但有时也造成生命、生活的毫无保障。}

\section{妄自尊大的狼}

一条狼徘徊在山脚下。落日的余辉使他的影子放得特别长。看着自己的影子,他得意洋洋地对自己说:“我有这么大的身体,几乎大到一亩田那样大,为什么还怕狮子?难道我不该被称为百兽之王吗?”正当他沉醉于其中时,一头狮子向他扑来,将他咬得快死了。此时狼悔恨不已,大声喊道:“我真不幸啊!是狂妄自大毁灭了我。”

{\bfseries \color{red}这是说那些盲目狂妄自大的人,会自食其恶果。}

\section{鹿、狼和羊}

鹿跑去向羊借一斗麦,并说狼可以为他担保。羊怀疑他是存心欺诈,便说:“狼常常抢夺他所要的东西,而你跑得比我快得多。到了偿还时,我怎么能找到你们呢?”

{\bfseries \color{red}这是说不要相信那些不值得相信的人,不要借钱物给那些根本不打算偿还的人。}

\section{牧羊人与小狼}

牧羊人捡到几只小狼崽子,很细心地抚育他们,心想养大了他们,不仅可以保护自己的羊群,还可以去把别人的羊抢来给自己。没想到那些小狼崽长大了,首先趁机咬死了牧羊人自己的羊。牧羊人悲叹地说:“我真活该!狼都该杀死,我为什么还去喂养这些小狼崽呢?”牧羊人与狼牧羊人捡到一只刚刚出生的狼崽子,把它带回家,跟他的狗喂养在一起。小狼长大以后,如有狼来叼羊,它就和狗一起去追赶。有一次,狗没追上,就回去了,那狼却继续追赶,待追上后,和其他狼一起分享了羊肉。从此以后,有时并没有狼来叼羊,它也偷偷地咬死一只羊,和狗一起分享。后来,牧羊人觉察到它的行为,便将它吊死在树上。

{\bfseries \color{red}这故事说明,恶劣的本性难以改变。}

\section{牧羊人与狼崽}

牧羊人发现了一只小狼,带回家喂养。小狼长大后,牧羊人教它去偷抢附近别人家的羊。已驯化的狼却说:“你要我养成了偷抢的习惯,那最好首先请你看守好自己的羊,别丢失了。”

{\bfseries \color{red}这是说,唆使别人干坏事,首先遭殃的是自己。}

\section{野驴和狼}

有一天,野驴的脚被刺扎了,走起路来一瘸一拐,十分痛苦。一条狼见到了受伤的野驴,想要吃掉这唾手可得的猎物。野驴请求他说:“你帮我拔出脚上的刺,消除我的痛苦,使我毫无痛苦地让你吃。”狼用牙齿把刺拔出来,野驴不再脚痛了,顿时,他的脚也有力了,便一脚踢死了狼,逃到别处,保住了自己的性命。

{\bfseries \color{red}这故事说明,对敌人行善,不仅得不到好处,还会遭到不幸。}

\section{小羊羔和狼}

狼追赶小羊羔,小羊羔逃到一座庙中躲藏。狼向他叫喊:“和尚如把你捉住,会把你杀了去祭神。”羊羔回答说:“在庙中祭神,比让你吃掉好得多。”

{\bfseries \color{red}这是说,无论遭到怎样的危险,也比死在恶人手中好。}

\section{狼医生}

驴子在牧场上吃草,看见一只狼向他跑来,便装出瘸腿的样子。狼走过来,问他脚怎么啦。他说越过篱笆时,踩着了刺,扎伤了脚,请狼先把刺拔掉,然后再吃他,免得扎伤喉咙。狼信以为真,便抬起驴的腿来,全神贯注地认真检查驴的蹄子。这时,驴子用脚对准狼的嘴使劲一蹬,踹掉了狼的牙齿。狼十分痛苦地说:“我真活该!父亲教我做屠户,我干嘛要去做医生呢?”

{\bfseries \color{red}这是说,那些不安分守己的人往往会遭到不幸。}

\section{狼与狗打仗}

有一次,狼与狗宣战。一只希腊狗被选为狗将军,他迟迟没有应战,狼却不断地大肆威胁他们。希腊狗说道:“知道我为什么犹豫不决吗?战前谋划至关重要。狼的种类与毛色几乎相同,我们却种类不同,性格不同,加上我们毛色五颜六色,有的黑色,有的红色,还有的是白与灰色。带领了这些完全不能统一的狗,如何能去应战呢?”

{\bfseries \color{red}这是说,人们必须团结一致、一心一意,方能战胜敌人。}

\section{狼、羊群和公羊}

狼派使者到羊那里去,说羊群若把守护他们的狗抓住杀了,便与他们缔结永久的和平。那些愚蠢的羊许诺了狼。这时,有只年老的公羊说:“怎么使我们信任你们并与你们一起生活呢?有狗保护我们时,你们还搅得我们不能平安地吃食呢。”

{\bfseries \color{red}这是说,人们不能相信坏人假惺惺的誓约,而放弃自己的安全保障。}

\section{占卜者}

占卜者坐在市场里收钱算卦,忽然有人赶来告诉他,他家的门被撬,家里的所有东西都被偷走了。占卜者大吃一惊,气得跳了起来,唉声叹气地赶回家中,察看所发生的事。一位旁观者见此便说:“喂,朋友,你不是宣称你能预知别人的祸福吗,怎么连自己的事情都没预测到呢?”

{\bfseries \color{red}这故事适用于那些连自己的事都预料不到,却扬言可以预测未来的人。}

\section{蜜蜂和牧人}

有个牧人发现树洞里有蜂蜜,就连忙上去想偷走。这时,从各处飞回的蜜蜂一下就把他包围了,并准备用毒刺刺他。牧人立刻说:“我走,我走。我一点儿蜂蜜也不要,只要你们别刺我。”

{\bfseries \color{red}这是说,不义之财不可取,否则将危害自己。}

\section{养蜜蜂的人}

有人走进养蜂人家里,见主人不在,便将蜂蜜和蜜粉偷走了。养蜂人回来看见蜂箱空了,就在蜂箱旁寻找不见的东西。这时,采花回来的蜜蜂看见他,都围住他用针刺。那人痛苦地对蜜蜂说:“啊,坏家伙!你们不惩治那偷蜜的人,却一个劲地来刺爱护你们的人。”

{\bfseries \color{red}这是说,愚蠢无知的人不去提防坏人,却戒备朋友,以友为敌。}

\section{僧人}

僧人们有一头驴子,他们常常让驴子驮着行李四处云游。有一天,驴子劳累致死,僧人剥下他的皮,用皮绷了一面鼓,敲打着它来化缘。别的僧人遇见他们,问他们的驴子哪里去了。他们说:“死了。可现在,他遭到更厉害的挨打,如果他还活着决受不了。”

{\bfseries \color{red}这是说,有些仆人即便摆脱了奴役,可改不掉他们的出身。}

\section{年轻人与屠夫}

两个年轻人去一家店铺里买肉。当屠夫转身忙着做事时,一个人偷了一块肉,并把肉放到另一个的怀里。屠夫回过身来,四处寻找那块肉,责怪他们。那偷肉的人发誓说没拿,怀里藏着肉的人发誓说没偷。屠夫识破了他们的诡计,说道:“即使你们发假誓骗过我,也骗不过神明。”

{\bfseries \color{red}这故事说明,骗人的假誓言总是会被识破的。}

\section{年轻的浪子与燕子}

年轻的浪子把传下来的祖业都挥霍一空,仅剩身穿的一件外衣。一天,他看见有一只燕子提早季节飞回,以为春天到了,不再要穿外衣了,便拿去卖了。不久一阵凛冽的北风袭来,非常寒冷,冻得他四处躲藏,碰巧见到燕子冻死在地上,便对他说道:“喂,朋友,你把我俩都毁了。”

{\bfseries \color{red}这故事说明,不按自然规律办事是十分危险的。}

\section{吃肉的小孩}

牧人们在野外祭祀,烹制了一只山羊,请来附近的人们共享。有个贫穷的女人,带着他的孩子也来了。正当大家吃得高兴时,那孩子因吃了太多的肉,肚子痛。他痛苦地说:“妈妈,我要把肉吐出来。”他妈妈说:“孩子,那不是你的,仅仅是你所吃下去的罢了。”

{\bfseries \color{red}这故事是说,有些人随随便便就拿别人的东西,欠别人的债,当被讨还时,却是那么痛苦。}

\section{小孩与乌鸦}

有一天,有个女人去为还不会说话的孩子算命,算命先生预言孩子将会被乌鸦所害死。她非常害怕,便做了一个大箱子,把孩子放在里边保护起来,她定时打开箱子,给孩子送饭菜和水。有一次,她正打开箱子盖给孩子送水,孩子顽皮地把头伸出来,不巧箱子上的鸦嘴形的搭扣砸在孩子的脑门上,把他砸死了。

{\bfseries \color{red}这是说,该来的灾难是躲不掉的,只有提高自己与灾难抗争的能力。}

\section{小孩与画的狮子}

有个胆小的老人有个独生子,他勇敢而且天生喜欢打猎。有一次,老人梦见儿子悲惨地被狮子咬死。他极害怕这梦变为现实,便特别建造了一座悬空的漂亮房子,将儿子锁在里面,把他保护起来。为了让儿子高兴,老人在墙上画了各种各样的动物,其中也画有狮子。然而,那孩子越看画越烦恼。有一次,他站在狮子画的旁边,说道:“喂,你这可恶的野兽,为了你和我父亲荒唐的梦,我才被关在这种像牢房一样的房里。”说着说着,便挥动拳头用力向墙打去,好像要把那狮子打死。不料一根刺钻到他指甲里去了,他疼痛难忍,最后发炎引起高烧不退,没多久便死了。原本是一头画在墙上的狮子,竟把孩子害死了。这位父亲精心的安排对孩子有害无益。

{\bfseries \color{red}这是说,人们要勇敢地去面对困难,而不要用什么心计去回避它。}

\section{人和蝈蝈}

有个穷人捉蝗虫时,捉到了一只叫声嘹亮的蝈蝈。那人正要弄死它时,蝈蝈说:“你为什么无缘无故地要弄死我?我既没危害庄稼,又没破坏森林。我发出悦耳动听的声音,还能让人们高兴,也许是吵闹一点,除此之外,无可挑剔。”那人听后,就把蝈蝈放走了。

{\bfseries \color{red}这故事是说,正确的道理是能说服人的。}

\section{跳蚤与运动员}

有一次,有只跳蚤跳到正在奔跑的运动员脚上,不停地咬他。他十分气愤,准备用手指捏住跳蚤。可跳蚤凭着天生的本领,一窜就逃跑了,保住了小命。运动员叹息地说:“赫拉克勒斯呀,如果你是这样帮助我去对付那小小跳蚤的话,又怎样帮助我去战胜强劲的对手呢?”

{\bfseries \color{red}这故事告诉我们,不要为那些无关紧要的事去求神,而要在遇到重大困难时再去求神。}

\section{骡子和强盗}

两匹满载背包的骡子长途跋涉,一匹驮着装满财宝的背包,另一匹驮着装满谷物的背包。驮着财宝的骡子昂着头,不断地摇动系在颈部的铃,使之发出清脆的声音。他趾高气扬地走着,仿佛知道所载东西的价值。而那一匹驮着谷物的骡子却以恬静、安闲的步伐跟着走。突然,一班强盗从隐蔽的地方冲出来打劫,在格斗中,一个强盗用一把短刀刺伤了那驮财宝的骡子,将财宝抢劫一空,而那驮着谷物的骡子根本没有引起强盗的注意。受伤的骡子哭诉他的不幸,另一匹却说:“我很高兴强盗不看重我,我没一点损失,也没有受伤。”

{\bfseries \color{red}这故事是说,财富并不值得夸耀,倒是要小心它会带来灾难。}

\section{两个士兵和强盗}

两个士兵一起赶路,途中,他们被一个强盗所劫。一个马上逃躲到一边,另一个勇敢地迎上去,与之搏斗,杀死了强盗。这时,那胆子小的士兵跑过来,抽出剑,并将外衣丢开,大声说:“我来对付他,我要让他知道,他所抢劫的是什么人。”这时,那名勇敢的士兵说:“我只愿你刚才能来帮助我,即使只说些话也好。因为我会相信这些话是真的,更会鼓足勇气去抗敌。而现在还是请你将剑插进鞘里,管住你那毫无用处的舌头吧。你只能欺骗那些不知道你的人。我亲眼见到了你逃跑的速度,十分清楚你的勇气是不可靠的。”

{\bfseries \color{red}这故事是说,有些人在事快要成功或已经成功后,企图把自己打扮成英雄,而在夺取成功的过程中,他们却袖手旁观。}

\section{补鞋匠改做医生}



{\bfseries \color{red}}

\section{兄和妹}

父亲有一个儿子和一个女儿,儿子以他的美貌而闻名,而女儿却以奇丑出名。有一天,兄妹俩偶然在镜子里看见了自己的面目。哥哥自夸他的美貌,妹妹十分生气,难以忍受哥哥的自我赞扬,她似乎觉得哥哥的自夸是在嘲笑自己。为了报复哥哥,她便跪倒在父亲跟前,抱怨说哥哥是男孩子,却要了应属于女孩子的美貌。父亲连忙拥抱住兄妹俩,给他们每人亲吻和抚爱,并说:“我愿你们俩每天都去照照镜子。我的儿子,你不可让恶行来污损你的美貌;我的女儿,你可以用你的美德来弥补美貌的不足。”

{\bfseries \color{red}这故事是说外表的美貌和内心的美德合而为一,才能使人真正的美丽。}

\section{乌鸦和羊}

一只讨厌的乌鸦站在羊背上。羊很不情愿地载着他前前后后走了很久,然后说:“你如这样子去对待狗,他早就会用锐利的牙齿来报答你了。”听完这话,乌鸦回答说:“我轻视弱者,服从强者。我知道能欺侮谁,应该奉承谁。我就是这样延长生命一直到老。”

{\bfseries \color{red}这是说那些唯利是图的小人总是欺弱怕强。}

\section{说大话的燕子与乌鸦}

燕子对乌鸦说:“我是漂亮的姑娘,是名雅典人,是公主,是雅典国王的女儿。”她又说忒柔斯强奸了她,还割去了她舌头。乌鸦说:“你舌头割去了,还这么滔滔不绝,口若悬河。你若有舌头,还不知将会怎么样吹牛呢?”

{\bfseries \color{red}这是说,那些好吹牛说大话的人,往往在自己的谎话中原形毕露。}

\section{鸽子与乌鸦}

有只鸽子被人饲养在舒适的鸽舍里,他到处吹嘘自己生了很多小鸽子。乌鸦听后,说:“喂,朋友,你别骄傲了,你生得越多,你就越为他们的生活操心。”

{\bfseries \color{red}这故事是说,多一个孩子多操一份心。}

\section{白嘴鸦与乌鸦}

乌鸦能为人们占卜吉凶,预测未来,被人们视为神鸟。白嘴鸦十分羡慕,也想这样做。他看见有些行人路过时,便飞到一棵树上,大声地叫了起来。行人们惊奇地听到白嘴鸦的声音,转过头来看了看。其中一个人说:“朋友们,我们赶快走吧,这是一只白嘴鸦,他的叫声毫无作用。”

{\bfseries \color{red}这故事是说,无能者嫉妒强者,往往达不到目的,而且还会遭到众人的耻笑。}

\section{乌鸦}

乌鸦非常羡慕天鹅洁白的羽毛。他猜想天鹅一定是经常洗澡,羽毛才变得如此洁白无瑕。于是,他毅然离开了他赖以生存的祭坛,来到江湖边。他天天洗刷自己的羽毛,不但一点都没洗白,反而因缺少食物饥饿而死。

{\bfseries \color{red}这故事是说,人的本性不会随着生活方式的改变而改变。}

\section{乌鸦与狐狸}

有只乌鸦偷到一块肉,衔着站在大树上。路过此地的狐狸看见后,口水直流,很想把肉弄到手。他便站在树下,大肆夸奖乌鸦的身体魁悟、羽毛美丽,还说他应该成为鸟类之王,若能发出声音,那就更当之无愧了。乌鸦为了要显示他能发出声音,便张嘴放声大叫,而那块肉掉到了树下。狐狸跑上去,抢到了那块肉,并嘲笑说:“喂,乌鸦,你若有头脑,真的可以当鸟类之王。”

{\bfseries \color{red}这故事适用于愚蠢的人。}

\section{乌鸦与赫耳墨斯}

有只乌鸦被捕鸟夹夹住了,他祈求阿波罗,说他若能脱险,将供奉贵重物品。阿波罗解救了他,但他把许的愿丢到了脑后。不久,他又被捕鸟夹夹住了,他再不敢求阿波罗,只好向赫耳墨斯许愿。赫耳墨斯对他说:“你这坏东西,你背弃和欺骗别人,我怎么还会相信你呢?”

{\bfseries \color{red}这故事是说,那些忘恩负义的人遇到灾难时,谁也不会去救他。}

\section{蚱蜢和猫头鹰}

一只猫头鹰每到晚上才出来吃东西,白天就睡觉。有一天,正当他睡得很香时,被一只蚱蜢的声音吵醒了,他没法入睡,便急切地请求蚱蜢停止叫声。蚱蜢却根本不理他,仍然叫个不停。猫头鹰越不断地请求,蚱蜢反而越叫得响。猫头鹰被弄得无可奈何,烦燥不安。突然他想到一个好计策,便对蚱蜢说:“听到你动听的歌声,我已睡不着了。你的歌声如同阿波罗神的七弦琴一样动听。我将把青春女神赫柏刚送给我的仙酒拿出来,痛痛快快地畅饮一场。你若不反对,就请上来一起喝吧。”蚱蜢这时正很渴,又被这赞美辞弄得高兴得忘乎所以,什么也没想就急忙地飞了上去。结果,猫头鹰从洞中冲出来,把蚱蜢弄死了。

{\bfseries \color{red}这故事是说有些人有一点点本事就飘飘然,忘乎所以,忘记了自己的地位和处境,结果,自找苦吃。}

\section{黄蜂和蛇}

一只黄蜂坐在一条蛇头上,不停地用刺去叮扎他,差一点置蛇于死地。忍受着极大痛苦的蛇不知所措,他无法回避这个小小的仇敌,怎么也吓不跑黄蜂。这时,一辆满载笨重木材的货车驶来,蛇便有意地将头放到车轮底下,并说:“让我和仇敌同归于尽吧!”

{\bfseries \color{red}这故事说明,与其备受敌人的折磨,不如与他们同归于尽。}

\section{行人与乌鸦}

几个人外出办事,正赶着路,偶遇一只独眼乌鸦迎面飞来。他们抬起头看了看乌鸦,其中一人说这是一种凶兆,劝大家赶快回去。另一个人却说:“乌鸦如何能预示未来呢?他若能预知,为什么不事先防止自己不瞎眼呢?”

{\bfseries \color{red}这是说,那些对于自己的事都考虑不周的人,也就没资格教训他人。}

\section{蝙蝠、荆棘与水鸟}

蝙蝠、荆棘、水鸟商定,合伙经商为生。于是蝙蝠借来钱作为资金,荆棘带来了他自己的衣服,水鸟带着赤铜,然后,他们装好货,扬帆远航。在海上不巧碰到了强大的风暴,船翻了,所有的货物全沉没了。幸运的是,他们被海浪冲到岸上,未被淹死。从此以后,水鸟总是站到水中,想把丢失的赤铜找回来;蝙蝠怕见债主,白天不敢出来,只有夜间才出来觅食;荆棘则到处寻找衣服,总把过路人的衣服抓住,看是否是自己的。

{\bfseries \color{red}这故事说明,许多人在一件事上失败过,以后凡做这事就格外地仔细认真。}

\section{蝙蝠与黄鼠狼}

蝙蝠掉落在地上,被黄鼠狼叼去,他请求饶命。黄鼠狼说绝不会放过他,自己生来痛恨鸟类。蝙蝠说他是老鼠,不是鸟,便被放了。后来蝙蝠又掉落了下来,被另一只黄鼠狼叼住,他再三请求不要吃他。这只黄鼠狼说他恨一切鼠类。蝙蝠改口说自己是鸟类,并非老鼠,又被放了。这样,蝙蝠两次改变了自己的名字,终于死里逃生。

{\bfseries \color{red}这故事说明,我们遇事要随机应变方能避免危险。}

\section{寒鸦与乌鸦}

有只寒鸦身体格外强壮,比其他寒鸦大得多。于是,他就瞧不起自己的同伴,自以为是地跑到乌鸦那里,想与他们共同生活。乌鸦们很快从他的形状和声音中认出他是寒鸦,并一齐啄赶他,把他驱逐出来。被赶出来后,他又只好回到寒鸦那里。然而曾受到他的侮辱的寒鸦们十分愤慨,一致不同意收留他。结果,这只寒鸦就变得无家可归了。

{\bfseries \color{red}这故事是说,那些看不起自己的亲人和同伴的人,既不会受到外人的欢迎,又会被同胞们所不齿。}

\section{橄榄树和无花果树}

冬天,橄榄树讥笑无花果树说:“我一年四季常青,永远漂亮,而你的树叶到冬天就会凋落,仅在夏天美丽。”正当他夸夸其谈时,天突然下起了大雪,雪花飘飘而来。雪花全都压在枝繁叶茂的橄榄树上,一会儿就把他压垮了,美丽也随之消失了。而光秃秃的无花果树,一点也没被雪伤害。

{\bfseries \color{red}这是说,有时候美丽外表会给人们带来危害。}

\section{冬天与春天}

冬天讥笑春天,专挑他的毛病,并责备他说,只要春天一到,人们就不再安静了,有的走进原野山林观赏风景,高兴地把采集来的鲜花插在头上;有的扬帆远航,漂洋过海到别的国家游玩,毫不担心什么*。他又说:“我却如同一个威严的帝王,我对天发令,使人们害怕*和大雪;我对地发令,使人们害怕天寒地冻;我强迫人们老老实实地只呆在家里度日。”春天说道:“正因如此,人们希望尽早地告别冬天。人们认为我的名字就是美丽。宙斯也说,春天是所有名字中最美的。因此,人们总是盼望春天来到。”

{\bfseries \color{red}这是说,威逼强迫只能使人产生反感,和煦温馨却使人向往。}

\section{强盗与桑树}



{\bfseries \color{red}}

\section{百灵鸟葬父亲}

据古代的传说,百灵鸟生于地球未出现之前。她父亲死于一场大病,因当时还没有地球,她找不到地方为父亲做坟墓。停丧五天后,她心中慌乱,就将父亲葬在自己的头上。从此以后,她头上就有了冠毛,人们传说那是她父亲的坟山。

{\bfseries \color{red}这故事是说青年人的第一责任是孝敬父母。}

\section{麻雀和野兔}

一只野兔被老鹰抓住了,他十分悲伤和痛苦,他的哭叫声如同孩子哭一般。这时,一只麻雀指责他说:“你飞快的速度怎么不见了?你的脚为什么跑得这么慢?”麻雀正说着,一只老鹰飞来,突然抓住他,将他吃掉了。野兔见后,心安地说:“唉!他刚才还幸灾乐祸,现在自己也遭到同样不幸的命运了。”

{\bfseries \color{red}这是说,见人遭受危难时,切不可幸灾乐祸。}

\section{鹦鹉与猫}

有人买回一只鹦鹉,精心地饲养,让它自由自在地生活。这只被驯养的鹦鹉,高兴得不停地叫唤。一只家猫看见了它,问它是谁,哪来的。它答道:“我是主人刚买来的。”猫说:“你这胆大的东西,怎么刚来就这么叽叽喳喳地叫。我是这里长大的,主人都不允许我这样做。有时若这么做了,他会大发脾气,赶我出去。”鹦鹉回答说:“好管家太太,你最好赶快出去。主人喜爱我悦耳的声音,而讨厌你的叫声。”

{\bfseries \color{red}这故事适用于那些总是对别人妄加评论的人。}

\section{燕子与鸟类}

从前,有一种能产生粘鸟胶的树,当这种树刚发芽时,燕子预感到鸟类将大难临头。于是,她召集鸟类,劝说他们一定要把所有这种树弄死。如果做不到,就马上飞到人那里去,向他们求助,请求他们不要用粘鸟胶来捕捉鸟类。所有的鸟都取笑燕子,认为她是说傻话。燕子无奈便独自飞到人那里,请求保护。人们认为她聪明、机智、善良,便答应了她的请求,允许她和人们住在一起。结果,别的鸟类都常被人捕捉,成为人们的美食。唯独燕子幸免于难,在人们家里平平安安地筑窝,无忧无虑地生活。

{\bfseries \color{red}这故事说明,未雨绸缪的人能避免危险。}

\section{天鹅}

有个富人养着家鹅和天鹅,他们的用处却不一样:养天鹅完全是因为他善于唱歌,养家鹅仅为吃肉。有一次,主人准备将家鹅派上用场,时值夜晚,辨别不出哪是家鹅哪是天鹅,天鹅被作为家鹅抓了出去。这时,他唱起歌来,以表他的悲哀。歌声道明了天鹅的本性,使他幸免于死难。

{\bfseries \color{red}这故事说明,音乐常常使生命延续。}

\section{天鹅与主人}

传说天鹅临死前才唱歌。有人偶然遇见市场上有天鹅出售,还听说这只天鹅的歌声非常悦耳动听,便买了带回家。有一天,他设宴请客,让天鹅在席间唱歌,天鹅却始终没吭一声。后来,天鹅老了,知道自己死到临头,这才为自己唱起了挽歌。主人听到后说:“如果你真是除临死之外,其余别的时间都不肯唱歌,那么我就是太傻了,那天叫你唱歌时,就应该把你杀了。”

{\bfseries \color{red}这是说,许多人不愿意自愿去做某些事,总是在迫不得已时,才勉强去做。}

\section{冠雀}

有只冠雀被捕鸟夹夹住了,他悲哀地说:“我真是最不幸的鸟呀!我没偷别人的金子、银子,更没偷别的贵重的东西,仅仅一颗小谷子却使我丧失了性命。”

{\bfseries \color{red}这故事是说那些贪小便宜而招来巨大灾难的人。}

\section{猿猴和两个人}

从前有两个人,一个总爱说实话,一个却只说谎话。有一次,他们偶然来到了猿猴国。一只自称为国王的猿猴吩咐手下捉住这两个人,他要询问这两人对他的看法。同时他还下令,所有的猿猴都要像人类的朝廷仪式那样,将在他左右分列成两行,中间给他放一个王位。一切准备妥当后,他发令,将那两人带到面前来,对那两个人说:“先生们,你们看,我是怎样的国王?”说谎的人回答说:“在我看来,你就像一个最有权力的国王。”“那旁边的这些猿猴呢?”那人连忙说:“他们都是你的栋梁之材,至少都能做大使和将帅。”那猿猴国王和他的手下听到这番谎话,十分得意,高兴地吩咐将美好的礼物送给这个阿谀奉承的人。那位说真话的人见到这般情形,心想:“一番谎话可得这般丰厚的报酬,那么,若我依照习惯,说了真话,又将怎样呢?”这时,那猿猴国王转过身来问他:“请问你觉得我和我的这些朋友怎么样呢?”他说道:“你是一只最优秀的猿猴,依此类推,你的所有同伴都是优秀的猿猴。”猿猴国王听到这些真话后,恼羞成怒,将说真话的人扔给手下去处置。

{\bfseries \color{red}这故事是说,许多人宁愿相信謟媚的假话,却不爱听道出实质的真话。}

\section{猴子与骆驼}

在动物们的集会上,猴子登台跳舞,深受欢迎,赢得大家的称赞,个个为之喝彩。骆驼却十分嫉妒猴子,他也想获得大家的喝彩。于是,他站了起来,自我得意地显示自己的舞技,结果,他那怪模怪样的舞姿,洋相百出,使动物们大为扫兴,他们用棍棒打他,把他赶跑了。

{\bfseries \color{red}这故事适用于那些不顾自身条件盲目模仿他人的人。}

\section{猴子与小猴}

一个猴子生了双胞胎,她只宠爱其中的一个,细心抚养,特别爱护,而对另一个却十分嫌弃,毫不经心。可不知是什么神的力量,那个为母亲宠爱、细心抚养的小猴,被紧紧抱在怀里而窒息死了,那个被嫌弃的却茁壮成长。

{\bfseries \color{red}这故事说明,过分的关心宠爱对孩子的成长不利。}

\section{狼与狗}

狼对狗说:“你们和我们几乎完全一样,咱们为什么就不能亲如兄弟?我们和你们其他方面毫无差别,可是你们却要屈服于主人,被套上颈圈,保护羊群。尽管你们劳累工作,甘心做奴隶,但仍免不了遭鞭打。你们若认为我说得对,那羊群就都归我们了。”那些狗同意了,狼走进羊圈里,首先把狗全咬死了。

{\bfseries \color{red}这是说,那些背叛朋友的人,都会受到严厉的惩罚。}

\section{驴子与狗}

驴子与狗一起外出赶路,发现地上有一封密封好的信。驴子捡起来,撕开封印,展开信纸大声朗读。信里谈到饲料、干草、大麦以及糠麸。狗听到驴子读的这些,很不舒服,不耐烦地对驴说:“好朋友,快读下去,看有没有提到肉和骨头。”驴子将信全部读完后,仍没有发现信中提到狗所想要的东西,狗就说:“把它扔了吧,朋友,都是些没有什么兴趣的东西。”

{\bfseries \color{red}这是说,有些人总是以自己的意愿代替他人的意愿。}

\section{狗和狼}

有只狗自认为自己有劲,跑得快,便拼命去追赶一只狼。毕竟狗的心里还是有点畏惧,不时躲躲闪闪。狼回过头来对他说:“你并不可怕,你身后的主人来袭击我那才真正可怕。”

{\bfseries \color{red}这是说那些狗仗人势的人。}

\section{睡着的狗与狼}

有条狗睡在羊圈前面。狼窥见后,冲上去袭击他,想把他吃掉。狗请求暂不要吃他,说道:“我现在还骨瘦如柴,你再等几天,我的主人要举行婚礼,那时我将吃得饱饱的,定会变得肥肥胖胖的,你再来吃不是更香些吗。”狼听信了狗的话,便放了他。过了几天狼再来时,发现狗已睡到了屋顶上,他便站在下面喊狗,提醒他记住以前的诺言。狗却说:“喂,狼呀,你若以后看见我睡在那羊圈前面,用不着再等婚礼了。”

{\bfseries \color{red}这故事说明,聪明的人一旦脱离险境后,他终生都会防范这种危险。}

\section{牧羊人与狗}

有一天,牧羊人把羊群赶进圈时,一条狼跑来,混入羊群中。牧羊人差一点儿把狼与在羊群关在一起。狗看见了,连忙对他说道:“你若想要这群羊,怎能把狼和羊群关在一起呢?”

{\bfseries \color{red}这是说,与恶人同居必将引来灾难和死亡。}

\section{寒鸦与狐狸}

有只饥饿的寒鸦站在一棵无花果树上。他发现无花果又硬又青,便一直守在那里等候它们长大成熟。狐狸看见寒鸦老是站在那里,就去问明其中的原因,随后说:“唉呀,朋友,你真糊涂,你只知一味等待是没用的,那只能欺骗你自己,而决不能填饱你的肚子。”

{\bfseries \color{red}这故事是说那些一味等待却不知努力行动的人。}

\section{寒鸦与鸽子}

寒鸦看见一群不愁吃喝的鸽子舒适地住在鸽舍里,便将自己的羽毛全都涂成白色,跑到鸽舍里,与他们一起过活。寒鸦一直不敢出声,鸽子便以为他也是只鸽子,允许他在—起生活,可是,有一次,他不留心,发出了一声叫声,鸽子们立刻辨认出了他的本来面目,将他啄赶出来。寒鸦在鸽子那里再也吃不到食了,只好又回到他的同类那里。然而他的毛色与以前不同了,寒鸦们不认识他,不让他与他们一起生活。这样,这只寒鸦因想贪得两份,最后却一份都没得到。

{\bfseries \color{red}这故事是说,人们应该满足于自己所有的东西,贪得无厌,最后会一无所获。}

\section{逃走的寒鸦}

有人捕捉到一只寒鸦,用麻绳拴住他的一只脚,给自己的孩子玩。那寒鸦很不愿意被小孩作弄,便逃回自己的窠里。可脚上的绳索却缠住了树枝,他再也飞不起来了。他临死时自言自语地说:“我真倒霉!因不愿忍受人的奴役,却丧失了自己的生命。”

{\bfseries \color{red}这故事是说,有些人为逃避普通的危险,反而遇到更大的灾祸。}

\section{吃饱了的狼与羊}

有只吃饱的狼,看见一只羊倒在地上,他以为羊是害怕他而瘫倒了,便走上前去鼓励他,说只要羊能对自己说三句真话,就放了他。羊开始说,第一,他不希望遇见狼;第二,若是不幸遇到了,希望遇到的是一只瞎眼狼;第三,他说,愿所有恶狼都死光,因为恶狼不断地伤害他们,而他们却从来没做过伤害狼的事。狼认为他的话一点不假,便放了羊。

{\bfseries \color{red}这故事说明,有时真话也能在敌人面前显示出力量。}

\section{牧羊人与羊}

牧羊人赶着一群羊来到橡树林里,看见一棵高大橡树上长满了橡子,十分招人喜爱,便高兴地脱下外衣,铺在地上,再爬上树去,使劲摇落橡子。羊群跑过来尽情享受这些橡子,不知不觉把牧羊人的外衣也啃吃完了。牧人从树上下来,见到如此情形,说道:“这些没用的坏家伙,你们把羊毛给他人做衣服穿,而我辛辛苦苦的喂养你们,你们却把我的外衣吃掉了。”

{\bfseries \color{red}这故事是说,有些糊涂无知的人善待外人,却损害自己人的利益。}

\section{公牛与野山羊}

有头公牛被狮子追赶,逃进了一个山洞,洞里住着一群野山羊。尽管野山羊朝他又踢又顶,公牛还是忍着痛对他们说:“我在这里忍辱负重,并不是害怕你们,而是害怕那站在洞口的狮子。”

{\bfseries \color{red}这故事说的是,为了逃避大灾难,必须忍受小痛苦。俗话说,小不忍,则乱大谋。}

\section{公牛、狮子和猎人}

公牛见一只小狮子睡得正香,便趁机用牛角把他顶死了。母狮走过来看到自己的孩子死了,十分悲伤,痛哭流涕。一头野猪站在远处对悲伤的狮子说:“你知道有多少人为他们的孩子被你们咬死而伤心落泪吗?”

{\bfseries \color{red}这故事是说,只有当自己也遭到同样不幸时,才会反省自己给别人带来的不幸。}

\section{老鼠和公牛}



{\bfseries \color{red}}

\section{公牛和小牛犊}

一头公牛竭尽全力要挤过一条小路,到牛栏里去。这时,一头小牛犊走了过来,争着要先走,并告诉公牛如何才能通过这条小路。公牛说:“不用劳驾你了,你还没出世前,我就早已知道那办法了。”

{\bfseries \color{red}这是说,年青人千万不要在老人面前逞能。}

\section{行人与真理}

有个人在荒凉的野外赶路,他看见一个女人眼睛盯着地下,独自站在路旁,便问她,“你是谁?”她说:“我是真理。”他又问道:“你为什么不住在那繁华热闹的城市,而住在这荒凉的野外呢?”她答道:“古时候,谬误只在少数人哪里,可是现在你无论到哪里都听到谬误。”

{\bfseries \color{red}这是说,当到处都充满谬误时,真理就远离人们无处存身了。}

\section{行人与赫耳墨斯}

有个行人经过长途跋涉后,发誓说若是发现了什么财富,一定将一半献给赫耳墨斯。他果然发现了一只装有杏仁与干枣的袋子,心想袋中一定有银子,立刻捡了起来。他把袋里所有的东西都倒了出来,发现袋里只有一些吃的东西后,便吃了起来,然后抓起杏仁壳和枣子核放到祭坛上,说道:“赫耳墨斯,请接受我所许诺的东西吧!现在我把它们连里带外全都献给你了。”

{\bfseries \color{red}这故事是说,那些贪心不足的人连神都要欺骗。}

\section{行人与幸运女神}

有个行人长途跋涉后,精疲力尽地倒在井边睡着了。当他差一点掉到井里时,幸运女神叫醒了他,说:“喂,朋友,你若掉到井里,一定会责怪我,决不会怨自己的疏忽。”

{\bfseries \color{red}这是说,许多人把由于自己造成的不幸,常常归之于命运。}

\section{宣誓之神}

曾经有这样一个人,信任他的朋友将钱寄存在他那里,请他代为保管后,他却想把这笔钱据为己有。朋友让他去宣誓,他心惊胆颤,借故离开家,向城外走去。走到城门口,遇见一个跛脚的人出城,便问:“你是谁?去哪里?”那人回答:“我是宣誓神,到那些不尊敬神的人那里去。”他又问:“你多长时间再回到城里来?”他回答说:“每四十年来一次,高兴时就三十年来一次。”第二天,那人毫不犹豫地跑去宣誓,说从没有代为保管过什么朋友的钱。这时,他忽然遇见了那宣誓神。当他被送到断头台上时,他责备宣誓神:“你明明说了每三四十年回来一次,现在却宽容一天都不肯。”宣誓神回答说:“但你该清楚,只要有谁惹怒了我,当天我就会赶回来。”

{\bfseries \color{red}这故事说明,对于不敬神的人们来说,神的惩罚是不按期的。}

\section{普罗米修斯与人}

普罗米修斯遵照宙斯的命令,创造了人和动物。宙斯看见动物太多,又命令他毁掉一些,以便多创造些人。普罗米修斯执行了宙斯的命令。因此,有些原本不是人的动物虽经改做,仅具有人的外形,内心却仍与动物一样。

{\bfseries \color{red}这故事说的是那些人面兽心的家伙。}

\section{老鹰、猫和野猪}

一只老鹰飞到一棵大橡树上筑起了巢。一只猫跑到这棵树的树干上找到一个树洞,在那里生下小猫。一只母野猪带着小猪住在这棵树树根的洞里。猫想独占这块地方,便实行她的诡计。她先爬到老鹰巢边说:“你们真不幸啊!不久将要被毁灭,我们也很危险。你不妨看看,那树下的野猪天天挖土,想把这棵树连根拔掉。树一倒下,他就可以轻而易举地把我们的孩子抓去,喂给他的孩子吃。”吓得老鹰心惊胆颤,惊惶失措。然后,猫又爬下来,来到野猪洞里说:“你的孩子们非常危险,只要你出去为小猪找食,树上的老鹰就会把他们叼了去。”猫狠狠地吓唬了野猪一番后,假装自己也很害怕,躲进了她的树洞。到了晚上,她偷偷地跑出去为自己和孩子寻找食物。白天,她仍装出一副恐惧的样子,整天躲在洞口守望着。于是,老鹰害怕野猪,静静地坐在枝头,不敢乱走;野猪也害怕老鹰,不敢走出洞来。这样,老鹰和野猪以及他们的孩子都饿死了。猫和她的孩子便把老鹰和野猪作为自己的食物了。

{\bfseries \color{red}这故事说的是那些两面三刀、挑拨离间的恶人。}

\section{乌鸦与蛇}

有只饥饿的乌鸦四处觅食,看见有一条蛇正熟睡在温暖的阳光里,便猛扑下去把他抓住。惊醒的蛇回过头来,咬了他一大口。乌鸦临死时说:“我真不幸!我虽找到了这样可口的好食物,却丢掉了生命。”

{\bfseries \color{red}这故事是说,有些人为了寻财找宝,不惜用生命去冒险。}

\section{战争与残暴}

各路神明都一一抓阄结了婚。剩下的最后一个阄被战争之神抓了去,与他相配的是残暴女神。于是,他们相爱,结为夫妻。从此以后,无论在什么地方,他俩总在一起。

{\bfseries \color{red}这是说,无论何时何地,只要有战争,就会有残暴。}

\section{河水与皮革}

河水对漂流在水中的皮革说:“你是谁?”皮革答道:“我叫坚硬。”湍急的河水拍打着他,说道:“你还是赶快另找个名字吧,我马上就能使你变软。”

{\bfseries \color{red}这是说,恢复事物的本性是轻而易举的。}

\section{墙壁与钉子}

墙壁被钉子猛烈地钉坏了,便大声叫道:“你为何要钉坏我,我与你无冤无仇,什么坏事我都没做。”钉子辩解道:“这一切都不是我的责任,你应该去责怪那个狠狠敲打我的人。”

{\bfseries \color{red}这就是说,责任要归之于罪魁祸首。}

\section{蚯蚓和蟒蛇}

有一天,路旁的一条蚯蚓看见一条长长的蟒蛇正在睡觉,蟒蛇修长的身材使他羡慕不已,他心想自己若有那样漂亮的身材该多好啊。于是,蚯蚓便爬到蟒蛇旁边,使劲地将自己拉长,不料用力过大,终于把自己的身体弄断了。

{\bfseries \color{red}这故事是说自不量力,不根据实际情况,盲目去模仿强者,对自己没有好处。}

\section{贼和旅馆老板}

一个贼在旅馆租住了一间房,一连住了几天,希望偷一点东西足够付房钱和饭钱,可他白白等了几天,一无所获。这天,贼看见旅馆老板穿着一件漂亮的新衣坐在门口,便走上前去,与他闲谈。谈了一会儿,他们都觉得疲倦了,贼打了一个呵欠,并像狼叫似的大吼了一声。旅馆老板说:“你怎么叫得这么吓人呢?”贼说:“我愿告诉你。但先请抓住我的衣服,我愿意把衣服放在你手中。先生,我自己也不知道我到底什么时候是这样打呵欠,也不知道这种可怕的嚎叫传染到我身上来是惩罚我的罪孽,还是其他别的原因。可有一点我是知道的,我若第三次打呵欠时,就会变成一只狼,去扑咬人。”说完之后,他又打了第二个呵欠,并和第一次一样,像狼一般的嚎叫。旅馆老板听完贼的故事,信以为真,非常恐惧,站起身来,准备逃走。贼扯住他的外衣,请他留步,并说:“先生,请等一等,扯住我的衣服,不然我变成狼时,就会暴怒地撕破它。”刚一说完,又像狼嚎叫一样打了第三个呵欠。旅馆老板害怕被贼伤害,便赶紧脱下新衣交给他,逃进旅馆躲藏起来。贼带着新衣连忙逃离旅馆,不再返回。

{\bfseries \color{red}这是说有些人为了达到某种目的,信口雌黄。如果相信其鬼话,肯定要吃亏。}

\section{神射手和狮子}

从前,有一个神射手。他到山里去寻找猎物,森林里的野兽见他来了,全都逃得无影无踪,只有高傲的狮子向他挑战。神射手朝狮子射出一箭,说:“这仅是给你一个消息,你可以从中知道我本人来攻打你的情形。”狮子被射中受了伤,吓得惊慌逃窜。狐狸劝狮子要勇敢些,不要轻易示弱。狮子回答说:“他的一枝箭都这么厉害,我还怎么能经受得住他本人的打击呢?”

{\bfseries \color{red}这故事是说,要善于借助外物去攻击那些不便直接攻击的强大敌人。}

\section{船主和船夫们}

有一天,人们乘船出海,天公不作美,海面掀起了狂风巨浪,船主一筹莫展,感到十分疲倦和烦躁。船夫们仍顶着风浪拼命地划船,累得几乎精疲力尽了。船主却严厉地对他们说:“你们再不划快点,我就用石头砸死你们。”其中一个船夫说:“但愿能到有石头的地方。”

{\bfseries \color{red}这故事告诉我们在生活中遇到危险时,要避重就轻,宁愿忍受小一点的危险,而躲避致命的危险。}

\section{人、马和小驹}

有个人骑着一匹已怀孕的母马上路。路途中,母马产下了小马。刚生下的小马驹跟着妈妈走一会儿,就感到全身乏力,他只好对骑在他妈妈背上的人说:“我这么一点点小,不能走多远。你若把我扔下,我马上就会死掉。如果你能把我放到什么地方喂养,日后我定将让你骑着我走。”

{\bfseries \color{red}这故事说明,行善会有好报,尽管这种好报很难很快实现。}

\section{猎人和骑马的人}

有个猎人扛着一只兔子打猎归来。路上一个骑马的人看见了他,便停下来假装要买兔子。骑马人一拿到兔子,就纵马飞奔而去。猎人拼命地在后面追赶,但终究没追上,他们相隔越来越远。猎人望着那远去的骑马人,无可奈何地说:“你走吧!那只兔子送给你了。”

{\bfseries \color{red}这故事是说,许多人迫于无奈才假装乐意,把自己舍不得的东西送给他人。}

\section{野猪、马与猎人}

从前,野猪和马常常同在一处吃草,野猪时常使坏,不是践踏青草,就是把水搅浑。马十分恼怒,一心想要报复他,便跑去请求猎人帮忙。猎人说除非马愿套上辔头让他骑,才帮助马惩治野猪。马报复心切,便答应了猎人的要求。于是,猎人骑在马上打败了野猪,随后又把马牵回去,拴在马槽边。马悲叹地说:“我真傻!为了一点小事不能容忍他人,现在却招致终身被奴役。”

{\bfseries \color{red}这故事是说,人们在生活中一定要对他人宽容,不要因为小事就想去报复他人,否则会给自己带来不幸。}

\section{黄蜂、鹧鸪与农夫}

有一次,黄蜂与鹧鸪因口渴难忍,飞到农夫那里求水喝,他们许诺将报答农夫,鹧鸪许诺在葡萄园松土,以便结出累累硕果;黄蜂许诺守护葡萄园,用毒刺驱逐偷吃的人。农夫说道:“我有两头牛,他们从不许诺什么,但什么活都干,因此,我把你们要的水给他们喝,那不更好吗!”

{\bfseries \color{red}这故事说的是那些随便许诺却并不打算实干的人。}

\section{行人与浮木}

几个行人一同沿着海边走,来到一处高地,看见大海的远处漂浮的木头,心想一定是一艘大海船。于是,他们等着它靠岸,想要搭乘这一艘船。当迎面而来的风把浮木吹到离岸边较近时,他们认为这不是艘大船,可能是一条小船,仍满怀希望地在那里等待。一个大浪把那木头送到岸上,他们才发现原来是一根木头,互相说道:“这无聊的东西使我们白等了一场!”

{\bfseries \color{red}这故事说明,有些人对不完全了解的东西,抱有很大的希望,但一经了解,却大失所望。}

\section{航海者}

有几个人乘船出海。大海的气候变化万千,船刚驶入海面时,恰恰碰上了狂风巨浪,船几乎被巨浪吞没。有个人撕破衣服,大声悲惨地痛哭,祈求庇护神,许愿说如能得救,定当还愿报恩。过了不久,风暴过去了,大海恢复了往日的平静,大家为幸免于难而互相祝福,手舞足蹈,高兴极了。老实的舵工却对他们说道:“朋友们,幸免于难确实值得高兴庆贺。但我们还必须勇敢地去面对说不定还会再来的狂风巨浪。”

{\bfseries \color{red}这故事告诫人们要知道天有不测风云,风平浪静时仍要警惕随时可能降临的惊涛骇浪。}

\section{富人与制皮匠}

有个富人与制皮匠相邻而住。那富人受不了皮革的臭气,多次逼迫制皮匠搬家。制皮匠总是说,马上就搬,却老是拖延不搬。这样一直拖来拖去,随着时间的推移,富人已闻惯了皮革的臭气,也就不再非难制皮匠了。

{\bfseries \color{red}这故事说明,习惯能消除对事物的恶感。俗话说习惯成自然。}

\section{富人与哭丧女}

富人有两个女儿,一个死了,便请来些哭丧女为女儿哭丧。另一个女儿对母亲说:“真是不幸!我们有丧事,却不知道怎么尽哀,而她们这些无亲无故的人却能这样悲痛欲绝,嚎啕大哭。”母亲回答说:“好孩子,不要大惊小怪,她们这样嚎啕痛哭,不是出自于内心的悲哀,而是为了金钱装出来的。”

{\bfseries \color{red}这故事说的是那些不惜借别人的不幸来牟取利益的人。}

\section{驴子与青蛙}

驴子驮着木料走过池塘,脚滑了一下,掉到水里,便失声痛哭。池塘里的青蛙听见他的哭声,说道:“喂,朋友,你摔倒一下就这么悲伤;如果像我们这样长久在这里生活又该怎么办呢?”

{\bfseries \color{red}这故事是说,有些人没有受过较大的困苦,一点小小的挫折都难以忍受。}

\section{病人与医生}

有个人生了病,医生问他怎么样,他说出汗过多。医生说:“这很好。”第二次又问他怎么样,他说畏寒怕冷,抖得十分厉害。医生说:“这也很好。”第三次医生再来问他的病情时,他说现在泻肚子。医生说:“这仍很好。”病人有一个亲戚来看他,问他怎么样,他说:“我就因这些很好而快丧命了。”

{\bfseries \color{red}这故事是说,只讲好话的人会给人们带来危险。}

